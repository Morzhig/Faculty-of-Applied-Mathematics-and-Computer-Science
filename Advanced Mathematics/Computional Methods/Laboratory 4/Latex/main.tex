\DeclareUnicodeCharacter{041B}{\CYRL}



\documentclass[11pt]{article}

    
    
    \usepackage[breakable]{tcolorbox}
    \usepackage{parskip} % Stop auto-indenting (to mimic markdown behaviour)
    \usepackage[english,russian]{babel}
    \usepackage{graphicx}
    \usepackage{subcaption}
    

    % Basic figure setup, for now with no caption control since it's done
    % automatically by Pandoc (which extracts ![](path) syntax from Markdown).
    \usepackage{graphicx}
    % Maintain compatibility with old templates. Remove in nbconvert 6.0
    \let\Oldincludegraphics\includegraphics
    % Ensure that by default, figures have no caption (until we provide a
    % proper Figure object with a Caption API and a way to capture that
    % in the conversion process - todo).
    \usepackage{caption}
    \DeclareCaptionFormat{nocaption}{}
    \captionsetup{format=nocaption,aboveskip=0pt,belowskip=0pt}

    \usepackage{float}
    \floatplacement{figure}{H} % forces figures to be placed at the correct location
    \usepackage{xcolor} % Allow colors to be defined
    \usepackage{enumerate} % Needed for markdown enumerations to work
    \usepackage{geometry} % Used to adjust the document margins
    \usepackage{amsmath} % Equations
    \usepackage{amssymb} % Equations
    \usepackage{textcomp} % defines textquotesingle
    % Hack from http://tex.stackexchange.com/a/47451/13684:
    \AtBeginDocument{%
        \def\PYZsq{\textquotesingle}% Upright quotes in Pygmentized code
    }
    \usepackage{upquote} % Upright quotes for verbatim code
    \usepackage{eurosym} % defines \euro

    \usepackage{iftex}
    \ifPDFTeX
        \usepackage[T1]{fontenc}
        \IfFileExists{alphabeta.sty}{
              \usepackage{alphabeta}
          }{
              \usepackage[mathletters]{ucs}
              \usepackage[utf8x]{inputenc}
          }
    \else
        \usepackage{fontspec}
        \usepackage{unicode-math}
    \fi

    \usepackage{fancyvrb} % verbatim replacement that allows latex
    \usepackage{grffile} % extends the file name processing of package graphics
                         % to support a larger range
    \makeatletter % fix for old versions of grffile with XeLaTeX
    \@ifpackagelater{grffile}{2019/11/01}
    {
      % Do nothing on new versions
    }
    {
      \def\Gread@@xetex#1{%
        \IfFileExists{"\Gin@base".bb}%
        {\Gread@eps{\Gin@base.bb}}%
        {\Gread@@xetex@aux#1}%
      }
    }
    \makeatother
    \usepackage[Export]{adjustbox} % Used to constrain images to a maximum size
    \adjustboxset{max size={0.9\linewidth}{0.9\paperheight}}

    % The hyperref package gives us a pdf with properly built
    % internal navigation ('pdf bookmarks' for the table of contents,
    % internal cross-reference links, web links for URLs, etc.)
    \usepackage{hyperref}
    % The default LaTeX title has an obnoxious amount of whitespace. By default,
    % titling removes some of it. It also provides customization options.
    \usepackage{titling}
    \usepackage{longtable} % longtable support required by pandoc >1.10
    \usepackage{booktabs}  % table support for pandoc > 1.12.2
    \usepackage{array}     % table support for pandoc >= 2.11.3
    \usepackage{calc}      % table minipage width calculation for pandoc >= 2.11.1
    \usepackage[inline]{enumitem} % IRkernel/repr support (it uses the enumerate* environment)
    \usepackage[normalem]{ulem} % ulem is needed to support strikethroughs (\sout)
                                % normalem makes italics be italics, not underlines
    \usepackage{mathrsfs}
    

    
    % Colors for the hyperref package
    \definecolor{urlcolor}{rgb}{0,.145,.698}
    \definecolor{linkcolor}{rgb}{.71,0.21,0.01}
    \definecolor{citecolor}{rgb}{.12,.54,.11}

    % ANSI colors
    \definecolor{ansi-black}{HTML}{3E424D}
    \definecolor{ansi-black-intense}{HTML}{282C36}
    \definecolor{ansi-red}{HTML}{E75C58}
    \definecolor{ansi-red-intense}{HTML}{B22B31}
    \definecolor{ansi-green}{HTML}{00A250}
    \definecolor{ansi-green-intense}{HTML}{007427}
    \definecolor{ansi-yellow}{HTML}{DDB62B}
    \definecolor{ansi-yellow-intense}{HTML}{B27D12}
    \definecolor{ansi-blue}{HTML}{208FFB}
    \definecolor{ansi-blue-intense}{HTML}{0065CA}
    \definecolor{ansi-magenta}{HTML}{D160C4}
    \definecolor{ansi-magenta-intense}{HTML}{A03196}
    \definecolor{ansi-cyan}{HTML}{60C6C8}
    \definecolor{ansi-cyan-intense}{HTML}{258F8F}
    \definecolor{ansi-white}{HTML}{C5C1B4}
    \definecolor{ansi-white-intense}{HTML}{A1A6B2}
    \definecolor{ansi-default-inverse-fg}{HTML}{FFFFFF}
    \definecolor{ansi-default-inverse-bg}{HTML}{000000}

    % common color for the border for error outputs.
    \definecolor{outerrorbackground}{HTML}{FFDFDF}

    % commands and environments needed by pandoc snippets
    % extracted from the output of `pandoc -s`
    \providecommand{\tightlist}{%
      \setlength{\itemsep}{0pt}\setlength{\parskip}{0pt}}
    \DefineVerbatimEnvironment{Highlighting}{Verbatim}{commandchars=\\\{\}}
    % Add ',fontsize=\small' for more characters per line
    \newenvironment{Shaded}{}{}
    \newcommand{\KeywordTok}[1]{\textcolor[rgb]{0.00,0.44,0.13}{\textbf{{#1}}}}
    \newcommand{\DataTypeTok}[1]{\textcolor[rgb]{0.56,0.13,0.00}{{#1}}}
    \newcommand{\DecValTok}[1]{\textcolor[rgb]{0.25,0.63,0.44}{{#1}}}
    \newcommand{\BaseNTok}[1]{\textcolor[rgb]{0.25,0.63,0.44}{{#1}}}
    \newcommand{\FloatTok}[1]{\textcolor[rgb]{0.25,0.63,0.44}{{#1}}}
    \newcommand{\CharTok}[1]{\textcolor[rgb]{0.25,0.44,0.63}{{#1}}}
    \newcommand{\StringTok}[1]{\textcolor[rgb]{0.25,0.44,0.63}{{#1}}}
    \newcommand{\CommentTok}[1]{\textcolor[rgb]{0.38,0.63,0.69}{\textit{{#1}}}}
    \newcommand{\OtherTok}[1]{\textcolor[rgb]{0.00,0.44,0.13}{{#1}}}
    \newcommand{\AlertTok}[1]{\textcolor[rgb]{1.00,0.00,0.00}{\textbf{{#1}}}}
    \newcommand{\FunctionTok}[1]{\textcolor[rgb]{0.02,0.16,0.49}{{#1}}}
    \newcommand{\RegionMarkerTok}[1]{{#1}}
    \newcommand{\ErrorTok}[1]{\textcolor[rgb]{1.00,0.00,0.00}{\textbf{{#1}}}}
    \newcommand{\NormalTok}[1]{{#1}}

    % Additional commands for more recent versions of Pandoc
    \newcommand{\ConstantTok}[1]{\textcolor[rgb]{0.53,0.00,0.00}{{#1}}}
    \newcommand{\SpecialCharTok}[1]{\textcolor[rgb]{0.25,0.44,0.63}{{#1}}}
    \newcommand{\VerbatimStringTok}[1]{\textcolor[rgb]{0.25,0.44,0.63}{{#1}}}
    \newcommand{\SpecialStringTok}[1]{\textcolor[rgb]{0.73,0.40,0.53}{{#1}}}
    \newcommand{\ImportTok}[1]{{#1}}
    \newcommand{\DocumentationTok}[1]{\textcolor[rgb]{0.73,0.13,0.13}{\textit{{#1}}}}
    \newcommand{\AnnotationTok}[1]{\textcolor[rgb]{0.38,0.63,0.69}{\textbf{\textit{{#1}}}}}
    \newcommand{\CommentVarTok}[1]{\textcolor[rgb]{0.38,0.63,0.69}{\textbf{\textit{{#1}}}}}
    \newcommand{\VariableTok}[1]{\textcolor[rgb]{0.10,0.09,0.49}{{#1}}}
    \newcommand{\ControlFlowTok}[1]{\textcolor[rgb]{0.00,0.44,0.13}{\textbf{{#1}}}}
    \newcommand{\OperatorTok}[1]{\textcolor[rgb]{0.40,0.40,0.40}{{#1}}}
    \newcommand{\BuiltInTok}[1]{{#1}}
    \newcommand{\ExtensionTok}[1]{{#1}}
    \newcommand{\PreprocessorTok}[1]{\textcolor[rgb]{0.74,0.48,0.00}{{#1}}}
    \newcommand{\AttributeTok}[1]{\textcolor[rgb]{0.49,0.56,0.16}{{#1}}}
    \newcommand{\InformationTok}[1]{\textcolor[rgb]{0.38,0.63,0.69}{\textbf{\textit{{#1}}}}}
    \newcommand{\WarningTok}[1]{\textcolor[rgb]{0.38,0.63,0.69}{\textbf{\textit{{#1}}}}}


    % Define a nice break command that doesn't care if a line doesn't already
    % exist.
    \def\br{\hspace*{\fill} \\* }
    % Math Jax compatibility definitions
    \def\gt{>}
    \def\lt{<}
    \let\Oldtex\TeX
    \let\Oldlatex\LaTeX
    \renewcommand{\TeX}{\textrm{\Oldtex}}
    \renewcommand{\LaTeX}{\textrm{\Oldlatex}}
    % Document parameters
    % Document title
    \title{CHM\_Lab1}
    
    
    
    
    
% Pygments definitions
\makeatletter
\def\PY@reset{\let\PY@it=\relax \let\PY@bf=\relax%
    \let\PY@ul=\relax \let\PY@tc=\relax%
    \let\PY@bc=\relax \let\PY@ff=\relax}
\def\PY@tok#1{\csname PY@tok@#1\endcsname}
\def\PY@toks#1+{\ifx\relax#1\empty\else%
    \PY@tok{#1}\expandafter\PY@toks\fi}
\def\PY@do#1{\PY@bc{\PY@tc{\PY@ul{%
    \PY@it{\PY@bf{\PY@ff{#1}}}}}}}
\def\PY#1#2{\PY@reset\PY@toks#1+\relax+\PY@do{#2}}

\@namedef{PY@tok@w}{\def\PY@tc##1{\textcolor[rgb]{0.73,0.73,0.73}{##1}}}
\@namedef{PY@tok@c}{\let\PY@it=\textit\def\PY@tc##1{\textcolor[rgb]{0.24,0.48,0.48}{##1}}}
\@namedef{PY@tok@cp}{\def\PY@tc##1{\textcolor[rgb]{0.61,0.40,0.00}{##1}}}
\@namedef{PY@tok@k}{\let\PY@bf=\textbf\def\PY@tc##1{\textcolor[rgb]{0.00,0.50,0.00}{##1}}}
\@namedef{PY@tok@kp}{\def\PY@tc##1{\textcolor[rgb]{0.00,0.50,0.00}{##1}}}
\@namedef{PY@tok@kt}{\def\PY@tc##1{\textcolor[rgb]{0.69,0.00,0.25}{##1}}}
\@namedef{PY@tok@o}{\def\PY@tc##1{\textcolor[rgb]{0.40,0.40,0.40}{##1}}}
\@namedef{PY@tok@ow}{\let\PY@bf=\textbf\def\PY@tc##1{\textcolor[rgb]{0.67,0.13,1.00}{##1}}}
\@namedef{PY@tok@nb}{\def\PY@tc##1{\textcolor[rgb]{0.00,0.50,0.00}{##1}}}
\@namedef{PY@tok@nf}{\def\PY@tc##1{\textcolor[rgb]{0.00,0.00,1.00}{##1}}}
\@namedef{PY@tok@nc}{\let\PY@bf=\textbf\def\PY@tc##1{\textcolor[rgb]{0.00,0.00,1.00}{##1}}}
\@namedef{PY@tok@nn}{\let\PY@bf=\textbf\def\PY@tc##1{\textcolor[rgb]{0.00,0.00,1.00}{##1}}}
\@namedef{PY@tok@ne}{\let\PY@bf=\textbf\def\PY@tc##1{\textcolor[rgb]{0.80,0.25,0.22}{##1}}}
\@namedef{PY@tok@nv}{\def\PY@tc##1{\textcolor[rgb]{0.10,0.09,0.49}{##1}}}
\@namedef{PY@tok@no}{\def\PY@tc##1{\textcolor[rgb]{0.53,0.00,0.00}{##1}}}
\@namedef{PY@tok@nl}{\def\PY@tc##1{\textcolor[rgb]{0.46,0.46,0.00}{##1}}}
\@namedef{PY@tok@ni}{\let\PY@bf=\textbf\def\PY@tc##1{\textcolor[rgb]{0.44,0.44,0.44}{##1}}}
\@namedef{PY@tok@na}{\def\PY@tc##1{\textcolor[rgb]{0.41,0.47,0.13}{##1}}}
\@namedef{PY@tok@nt}{\let\PY@bf=\textbf\def\PY@tc##1{\textcolor[rgb]{0.00,0.50,0.00}{##1}}}
\@namedef{PY@tok@nd}{\def\PY@tc##1{\textcolor[rgb]{0.67,0.13,1.00}{##1}}}
\@namedef{PY@tok@s}{\def\PY@tc##1{\textcolor[rgb]{0.73,0.13,0.13}{##1}}}
\@namedef{PY@tok@sd}{\let\PY@it=\textit\def\PY@tc##1{\textcolor[rgb]{0.73,0.13,0.13}{##1}}}
\@namedef{PY@tok@si}{\let\PY@bf=\textbf\def\PY@tc##1{\textcolor[rgb]{0.64,0.35,0.47}{##1}}}
\@namedef{PY@tok@se}{\let\PY@bf=\textbf\def\PY@tc##1{\textcolor[rgb]{0.67,0.36,0.12}{##1}}}
\@namedef{PY@tok@sr}{\def\PY@tc##1{\textcolor[rgb]{0.64,0.35,0.47}{##1}}}
\@namedef{PY@tok@ss}{\def\PY@tc##1{\textcolor[rgb]{0.10,0.09,0.49}{##1}}}
\@namedef{PY@tok@sx}{\def\PY@tc##1{\textcolor[rgb]{0.00,0.50,0.00}{##1}}}
\@namedef{PY@tok@m}{\def\PY@tc##1{\textcolor[rgb]{0.40,0.40,0.40}{##1}}}
\@namedef{PY@tok@gh}{\let\PY@bf=\textbf\def\PY@tc##1{\textcolor[rgb]{0.00,0.00,0.50}{##1}}}
\@namedef{PY@tok@gu}{\let\PY@bf=\textbf\def\PY@tc##1{\textcolor[rgb]{0.50,0.00,0.50}{##1}}}
\@namedef{PY@tok@gd}{\def\PY@tc##1{\textcolor[rgb]{0.63,0.00,0.00}{##1}}}
\@namedef{PY@tok@gi}{\def\PY@tc##1{\textcolor[rgb]{0.00,0.52,0.00}{##1}}}
\@namedef{PY@tok@gr}{\def\PY@tc##1{\textcolor[rgb]{0.89,0.00,0.00}{##1}}}
\@namedef{PY@tok@ge}{\let\PY@it=\textit}
\@namedef{PY@tok@gs}{\let\PY@bf=\textbf}
\@namedef{PY@tok@gp}{\let\PY@bf=\textbf\def\PY@tc##1{\textcolor[rgb]{0.00,0.00,0.50}{##1}}}
\@namedef{PY@tok@go}{\def\PY@tc##1{\textcolor[rgb]{0.44,0.44,0.44}{##1}}}
\@namedef{PY@tok@gt}{\def\PY@tc##1{\textcolor[rgb]{0.00,0.27,0.87}{##1}}}
\@namedef{PY@tok@err}{\def\PY@bc##1{{\setlength{\fboxsep}{\string -\fboxrule}\fcolorbox[rgb]{1.00,0.00,0.00}{1,1,1}{\strut ##1}}}}
\@namedef{PY@tok@kc}{\let\PY@bf=\textbf\def\PY@tc##1{\textcolor[rgb]{0.00,0.50,0.00}{##1}}}
\@namedef{PY@tok@kd}{\let\PY@bf=\textbf\def\PY@tc##1{\textcolor[rgb]{0.00,0.50,0.00}{##1}}}
\@namedef{PY@tok@kn}{\let\PY@bf=\textbf\def\PY@tc##1{\textcolor[rgb]{0.00,0.50,0.00}{##1}}}
\@namedef{PY@tok@kr}{\let\PY@bf=\textbf\def\PY@tc##1{\textcolor[rgb]{0.00,0.50,0.00}{##1}}}
\@namedef{PY@tok@bp}{\def\PY@tc##1{\textcolor[rgb]{0.00,0.50,0.00}{##1}}}
\@namedef{PY@tok@fm}{\def\PY@tc##1{\textcolor[rgb]{0.00,0.00,1.00}{##1}}}
\@namedef{PY@tok@vc}{\def\PY@tc##1{\textcolor[rgb]{0.10,0.09,0.49}{##1}}}
\@namedef{PY@tok@vg}{\def\PY@tc##1{\textcolor[rgb]{0.10,0.09,0.49}{##1}}}
\@namedef{PY@tok@vi}{\def\PY@tc##1{\textcolor[rgb]{0.10,0.09,0.49}{##1}}}
\@namedef{PY@tok@vm}{\def\PY@tc##1{\textcolor[rgb]{0.10,0.09,0.49}{##1}}}
\@namedef{PY@tok@sa}{\def\PY@tc##1{\textcolor[rgb]{0.73,0.13,0.13}{##1}}}
\@namedef{PY@tok@sb}{\def\PY@tc##1{\textcolor[rgb]{0.73,0.13,0.13}{##1}}}
\@namedef{PY@tok@sc}{\def\PY@tc##1{\textcolor[rgb]{0.73,0.13,0.13}{##1}}}
\@namedef{PY@tok@dl}{\def\PY@tc##1{\textcolor[rgb]{0.73,0.13,0.13}{##1}}}
\@namedef{PY@tok@s2}{\def\PY@tc##1{\textcolor[rgb]{0.73,0.13,0.13}{##1}}}
\@namedef{PY@tok@sh}{\def\PY@tc##1{\textcolor[rgb]{0.73,0.13,0.13}{##1}}}
\@namedef{PY@tok@s1}{\def\PY@tc##1{\textcolor[rgb]{0.73,0.13,0.13}{##1}}}
\@namedef{PY@tok@mb}{\def\PY@tc##1{\textcolor[rgb]{0.40,0.40,0.40}{##1}}}
\@namedef{PY@tok@mf}{\def\PY@tc##1{\textcolor[rgb]{0.40,0.40,0.40}{##1}}}
\@namedef{PY@tok@mh}{\def\PY@tc##1{\textcolor[rgb]{0.40,0.40,0.40}{##1}}}
\@namedef{PY@tok@mi}{\def\PY@tc##1{\textcolor[rgb]{0.40,0.40,0.40}{##1}}}
\@namedef{PY@tok@il}{\def\PY@tc##1{\textcolor[rgb]{0.40,0.40,0.40}{##1}}}
\@namedef{PY@tok@mo}{\def\PY@tc##1{\textcolor[rgb]{0.40,0.40,0.40}{##1}}}
\@namedef{PY@tok@ch}{\let\PY@it=\textit\def\PY@tc##1{\textcolor[rgb]{0.24,0.48,0.48}{##1}}}
\@namedef{PY@tok@cm}{\let\PY@it=\textit\def\PY@tc##1{\textcolor[rgb]{0.24,0.48,0.48}{##1}}}
\@namedef{PY@tok@cpf}{\let\PY@it=\textit\def\PY@tc##1{\textcolor[rgb]{0.24,0.48,0.48}{##1}}}
\@namedef{PY@tok@c1}{\let\PY@it=\textit\def\PY@tc##1{\textcolor[rgb]{0.24,0.48,0.48}{##1}}}
\@namedef{PY@tok@cs}{\let\PY@it=\textit\def\PY@tc##1{\textcolor[rgb]{0.24,0.48,0.48}{##1}}}

\def\PYZbs{\char`\\}
\def\PYZus{\char`\_}
\def\PYZob{\char`\{}
\def\PYZcb{\char`\}}
\def\PYZca{\char`\^}
\def\PYZam{\char`\&}
\def\PYZlt{\char`\<}
\def\PYZgt{\char`\>}
\def\PYZsh{\char`\#}
\def\PYZpc{\char`\%}
\def\PYZdl{\char`\$}
\def\PYZhy{\char`\-}
\def\PYZsq{\char`\'}
\def\PYZdq{\char`\"}
\def\PYZti{\char`\~}
% for compatibility with earlier versions
\def\PYZat{@}
\def\PYZlb{[}
\def\PYZrb{]}
\makeatother


    % For linebreaks inside Verbatim environment from package fancyvrb.
    \makeatletter
        \newbox\Wrappedcontinuationbox
        \newbox\Wrappedvisiblespacebox
        \newcommand*\Wrappedvisiblespace {\textcolor{red}{\textvisiblespace}}
        \newcommand*\Wrappedcontinuationsymbol {\textcolor{red}{\llap{\tiny$\m@th\hookrightarrow$}}}
        \newcommand*\Wrappedcontinuationindent {3ex }
        \newcommand*\Wrappedafterbreak {\kern\Wrappedcontinuationindent\copy\Wrappedcontinuationbox}
        % Take advantage of the already applied Pygments mark-up to insert
        % potential linebreaks for TeX processing.
        %        {, <, #, %, $, ' and ": go to next line.
        %        _, }, ^, &, >, - and ~: stay at end of broken line.
        % Use of \textquotesingle for straight quote.
        \newcommand*\Wrappedbreaksatspecials {%
            \def\PYGZus{\discretionary{\char`\_}{\Wrappedafterbreak}{\char`\_}}%
            \def\PYGZob{\discretionary{}{\Wrappedafterbreak\char`\{}{\char`\{}}%
            \def\PYGZcb{\discretionary{\char`\}}{\Wrappedafterbreak}{\char`\}}}%
            \def\PYGZca{\discretionary{\char`\^}{\Wrappedafterbreak}{\char`\^}}%
            \def\PYGZam{\discretionary{\char`\&}{\Wrappedafterbreak}{\char`\&}}%
            \def\PYGZlt{\discretionary{}{\Wrappedafterbreak\char`\<}{\char`\<}}%
            \def\PYGZgt{\discretionary{\char`\>}{\Wrappedafterbreak}{\char`\>}}%
            \def\PYGZsh{\discretionary{}{\Wrappedafterbreak\char`\#}{\char`\#}}%
            \def\PYGZpc{\discretionary{}{\Wrappedafterbreak\char`\%}{\char`\%}}%
            \def\PYGZdl{\discretionary{}{\Wrappedafterbreak\char`\$}{\char`\$}}%
            \def\PYGZhy{\discretionary{\char`\-}{\Wrappedafterbreak}{\char`\-}}%
            \def\PYGZsq{\discretionary{}{\Wrappedafterbreak\textquotesingle}{\textquotesingle}}%
            \def\PYGZdq{\discretionary{}{\Wrappedafterbreak\char`\"}{\char`\"}}%
            \def\PYGZti{\discretionary{\char`\~}{\Wrappedafterbreak}{\char`\~}}%
        }
        % Some characters . , ; ? ! / are not pygmentized.
        % This macro makes them "active" and they will insert potential linebreaks
        \newcommand*\Wrappedbreaksatpunct {%
            \lccode`\~`\.\lowercase{\def~}{\discretionary{\hbox{\char`\.}}{\Wrappedafterbreak}{\hbox{\char`\.}}}%
            \lccode`\~`\,\lowercase{\def~}{\discretionary{\hbox{\char`\,}}{\Wrappedafterbreak}{\hbox{\char`\,}}}%
            \lccode`\~`\;\lowercase{\def~}{\discretionary{\hbox{\char`\;}}{\Wrappedafterbreak}{\hbox{\char`\;}}}%
            \lccode`\~`\:\lowercase{\def~}{\discretionary{\hbox{\char`\:}}{\Wrappedafterbreak}{\hbox{\char`\:}}}%
            \lccode`\~`\?\lowercase{\def~}{\discretionary{\hbox{\char`\?}}{\Wrappedafterbreak}{\hbox{\char`\?}}}%
            \lccode`\~`\!\lowercase{\def~}{\discretionary{\hbox{\char`\!}}{\Wrappedafterbreak}{\hbox{\char`\!}}}%
            \lccode`\~`\/\lowercase{\def~}{\discretionary{\hbox{\char`\/}}{\Wrappedafterbreak}{\hbox{\char`\/}}}%
            \catcode`\.\active
            \catcode`\,\active
            \catcode`\;\active
            \catcode`\:\active
            \catcode`\?\active
            \catcode`\!\active
            \catcode`\/\active
            \lccode`\~`\~
        }
    \makeatother

    \let\OriginalVerbatim=\Verbatim
    \makeatletter
    \renewcommand{\Verbatim}[1][1]{%
        %\parskip\z@skip
        \sbox\Wrappedcontinuationbox {\Wrappedcontinuationsymbol}%
        \sbox\Wrappedvisiblespacebox {\FV@SetupFont\Wrappedvisiblespace}%
        \def\FancyVerbFormatLine ##1{\hsize\linewidth
            \vtop{\raggedright\hyphenpenalty\z@\exhyphenpenalty\z@
                \doublehyphendemerits\z@\finalhyphendemerits\z@
                \strut ##1\strut}%
        }%
        % If the linebreak is at a space, the latter will be displayed as visible
        % space at end of first line, and a continuation symbol starts next line.
        % Stretch/shrink are however usually zero for typewriter font.
        \def\FV@Space {%
            \nobreak\hskip\z@ plus\fontdimen3\font minus\fontdimen4\font
            \discretionary{\copy\Wrappedvisiblespacebox}{\Wrappedafterbreak}
            {\kern\fontdimen2\font}%
        }%

        % Allow breaks at special characters using \PYG... macros.
        \Wrappedbreaksatspecials
        % Breaks at punctuation characters . , ; ? ! and / need catcode=\active
        \OriginalVerbatim[#1,codes*=\Wrappedbreaksatpunct]%
    }
    \makeatother

    % Exact colors from NB
    \definecolor{incolor}{HTML}{303F9F}
    \definecolor{outcolor}{HTML}{D84315}
    \definecolor{cellborder}{HTML}{CFCFCF}
    \definecolor{cellbackground}{HTML}{F7F7F7}

    % prompt
    \makeatletter
    \newcommand{\boxspacing}{\kern\kvtcb@left@rule\kern\kvtcb@boxsep}
    \makeatother
    \newcommand{\prompt}[4]{
        {\ttfamily\llap{{\color{#2}[#3]:\hspace{3pt}#4}}\vspace{-\baselineskip}}
    }
    

    
    % Prevent overflowing lines due to hard-to-break entities
    \sloppy
    % Setup hyperref package
    \hypersetup{
      breaklinks=true,  % so long urls are correctly broken across lines
      colorlinks=true,
      urlcolor=urlcolor,
      linkcolor=linkcolor,
      citecolor=citecolor,
      }
    % Slightly bigger margins than the latex defaults
    
    \geometry{verbose,tmargin=1in,bmargin=1in,lmargin=1in,rmargin=1in}
    
\newtheorem{theorem}{Теорема}

\begin{document}
    
    \begin{titlepage}
    \newpage
    
    \begin{center}
    МИНИСТЕРСТВО ОБРАЗОВАНИЯ РЕСПУБЛИКИ БЕЛАРУСЬ БЕЛОРУССКИЙ ГОСУДАРСТВЕННЫЙ УНИВЕРСИТЕТ \\
    Факультет прикладной математики и инворматики \\ Кафедра вычислительной математики
 
    \end{center}
    
    \vspace{8em}
    
    \vspace{2em}
    
    \begin{center}
    \textsc{\textbf{Отчет по лабораторной работе 4 \\ "Численное интегрирование" \linebreak Вариант 5}}
    \end{center}
    
    \vspace{6em}
    
    \begin{flushright}
        Выполнил:\\
        Карпович Артём Дмитриевич\\
        студент 3 курса 7 группы
    \end{flushright}
    
    \begin{flushright}
        Преподаватель:\\
        Репников Василий Иванович
    \end{flushright}
    
    \vspace{\fill}
    
    \vspace{\fill}
    
    \begin{center}
    Минск, 2024
    \end{center}
    
    \end{titlepage}
    
    

    
    \section*{Численное интегрирование}

\subsection*{Постановка задач} 1. Построить квадратурную формулу
максимально возможной степени точности вида
\[\int\limits_a^b f(x)dx \approx A_0 f(a) + A_1 f(b) + A_2 f'(a) + A_3 f'(b).\]
2. Определить алгебраическую степень точности указанной квадратурной
формулы
\[\int\limits_{-1}^1 f(x)dx \approx \dfrac{1}{6}[f(-1) + f(1)] + \dfrac{5}{6}[f(-\dfrac{1}{\sqrt5}) + f(\dfrac{1}{\sqrt5})].\]
3. Используя правило Рунге, провести сравнительный анализ квадратурных
формул средних прямоугольников и трапеций на примере вычисления
интеграла \[I = \int\limits_1^3 \dfrac{\ln(\sin^2x + 3)}{x^2+2x-1}dx.\]
4. Вычислить с точностью \(\epsilon = 10^{-4}\) интеграл
\[I = \int\limits_{-1}^1 \frac{1}{\sqrt{(1-x^2)}} \dfrac{\sin x^2}{1 + \ln^2(x+1)}dx.\]
5. Найти с точностью до \(\epsilon = 10^{-4}\) решение уравнения
\[\int\limits_{0}^X (t-1)^6 (\lg \sqrt{t^2 + 1} + 2)dt = 5. \]

    \subsection*{Задача 1}

Для построения квадратурной формы с алгребраической степенью точности
\(m\) необходимо составить соотношения $$
\begin{cases}
\int\limits_a^b \rho(x) x^idx = \sum\limits_{k=0}^{n}A_kx^i_k,\quad i=\overline{0,m},\\
\int\limits_a^b \rho(x) x^{m+1}dx \ne \sum\limits_{k=0}^{n}A_kx^i_k;
\end{cases}$$ Из этих соотношений можно составить систему для
нахождения коэффициентов \(A_0, A_1, A_2, A_3\)

$$\begin{cases}
x^0: \int\limits_a^b 1 dx = A_0 + A_1,\\
x^1: \int\limits_a^b x dx = A_0a + A_1b + A_2 + A_3, \\
x^2: \int\limits_a^b x^2 dx = A_0a^2 + A_1b^2 + 2A_2a + 2A_3b,\\
x^3: \int\limits_a^b x^3 dx = A_0a^3 + A_1b^3 + 3A_2a^2 + 3A_3b^2.
\end{cases}$$

Раскроем интегралы и получим

$$\begin{cases}
b-a = A_0 + A_1,\\
\frac{(b-a)^2}{2} = A_0a + A_1b + A_2 + A_3, \\
\frac{(b-a)^3}{3} = A_0a^2 + A_1b^2 + 2A_2a + 2A_3b,\\
\frac{(b-a)^4}{4} = A_0a^3 + A_1b^3 + 3A_2a^2 + 3A_3b^2.
\end{cases}$$

Найдем решение системы программно, используя инструменты языка Python

    \begin{tcolorbox}[breakable, size=fbox, boxrule=1pt, pad at break*=1mm,colback=cellbackground, colframe=cellborder]
\prompt{In}{incolor}{4}{\boxspacing}
\begin{Verbatim}[commandchars=\\\{\}]
\PY{k+kn}{import} \PY{n+nn}{math}
\PY{k+kn}{import} \PY{n+nn}{numpy} \PY{k}{as} \PY{n+nn}{np}
\PY{k+kn}{import} \PY{n+nn}{sympy} \PY{k}{as} \PY{n+nn}{sp}
\PY{k+kn}{import} \PY{n+nn}{pandas} \PY{k}{as} \PY{n+nn}{pd}

\PY{n}{a}\PY{p}{,} \PY{n}{b}\PY{p}{,} \PY{n}{A0}\PY{p}{,} \PY{n}{A1}\PY{p}{,} \PY{n}{A2}\PY{p}{,} \PY{n}{A3} \PY{o}{=} \PY{n}{sp}\PY{o}{.}\PY{n}{symbols}\PY{p}{(}\PY{l+s+s1}{\PYZsq{}}\PY{l+s+s1}{a b A0 A1 A2 A3}\PY{l+s+s1}{\PYZsq{}}\PY{p}{)}

\PY{n}{eq1} \PY{o}{=} \PY{n}{sp}\PY{o}{.}\PY{n}{Eq}\PY{p}{(}\PY{n}{b} \PY{o}{\PYZhy{}} \PY{n}{a}\PY{p}{,} \PY{n}{A0} \PY{o}{+} \PY{n}{A1}\PY{p}{)}
\PY{n}{eq2} \PY{o}{=} \PY{n}{sp}\PY{o}{.}\PY{n}{Eq}\PY{p}{(}\PY{p}{(}\PY{p}{(}\PY{n}{b} \PY{o}{\PYZhy{}} \PY{n}{a}\PY{p}{)}\PY{o}{*}\PY{o}{*}\PY{l+m+mi}{2}\PY{p}{)}\PY{o}{/}\PY{l+m+mi}{2}\PY{p}{,} \PY{n}{A0}\PY{o}{*}\PY{n}{a} \PY{o}{+} \PY{n}{A1}\PY{o}{*}\PY{n}{b} \PY{o}{+} \PY{n}{A2} \PY{o}{+} \PY{n}{A3}\PY{p}{)}
\PY{n}{eq3} \PY{o}{=} \PY{n}{sp}\PY{o}{.}\PY{n}{Eq}\PY{p}{(}\PY{p}{(}\PY{p}{(}\PY{n}{b} \PY{o}{\PYZhy{}} \PY{n}{a}\PY{p}{)}\PY{o}{*}\PY{o}{*}\PY{l+m+mi}{3}\PY{p}{)}\PY{o}{/}\PY{l+m+mi}{3}\PY{p}{,} \PY{n}{A0}\PY{o}{*}\PY{n}{a}\PY{o}{*}\PY{o}{*}\PY{l+m+mi}{2} \PY{o}{+} \PY{n}{A1}\PY{o}{*}\PY{n}{b}\PY{o}{*}\PY{o}{*}\PY{l+m+mi}{2} \PY{o}{+} \PY{l+m+mi}{2}\PY{o}{*}\PY{n}{A2}\PY{o}{*}\PY{n}{a} \PY{o}{+} \PY{l+m+mi}{2}\PY{o}{*}\PY{n}{A3}\PY{o}{*}\PY{n}{b}\PY{p}{)}
\PY{n}{eq4} \PY{o}{=} \PY{n}{sp}\PY{o}{.}\PY{n}{Eq}\PY{p}{(}\PY{p}{(}\PY{p}{(}\PY{n}{b} \PY{o}{\PYZhy{}} \PY{n}{a}\PY{p}{)}\PY{o}{*}\PY{o}{*}\PY{l+m+mi}{4}\PY{p}{)}\PY{o}{/}\PY{l+m+mi}{4}\PY{p}{,} \PY{n}{A0}\PY{o}{*}\PY{n}{a}\PY{o}{*}\PY{o}{*}\PY{l+m+mi}{3} \PY{o}{+} \PY{n}{A1}\PY{o}{*}\PY{n}{b}\PY{o}{*}\PY{o}{*}\PY{l+m+mi}{3} \PY{o}{+} \PY{l+m+mi}{3}\PY{o}{*}\PY{n}{A2}\PY{o}{*}\PY{n}{a}\PY{o}{*}\PY{o}{*}\PY{l+m+mi}{2} \PY{o}{+} \PY{l+m+mi}{3}\PY{o}{*}\PY{n}{A3}\PY{o}{*}\PY{n}{b}\PY{o}{*}\PY{o}{*}\PY{l+m+mi}{2}\PY{p}{)}

\PY{n}{solution} \PY{o}{=} \PY{n}{sp}\PY{o}{.}\PY{n}{solve}\PY{p}{(}\PY{p}{(}\PY{n}{eq1}\PY{p}{,} \PY{n}{eq2}\PY{p}{,} \PY{n}{eq3}\PY{p}{,} \PY{n}{eq4}\PY{p}{)}\PY{p}{,} \PY{p}{(}\PY{n}{A0}\PY{p}{,} \PY{n}{A1}\PY{p}{,} \PY{n}{A2}\PY{p}{,} \PY{n}{A3}\PY{p}{)}\PY{p}{)}

\PY{n}{simplified\PYZus{}solution} \PY{o}{=} \PY{p}{\PYZob{}}\PY{n}{key}\PY{p}{:} \PY{n}{sp}\PY{o}{.}\PY{n}{simplify}\PY{p}{(}\PY{n}{value}\PY{p}{)} \PY{k}{for} \PY{n}{key}\PY{p}{,} \PY{n}{value} \PY{o+ow}{in} \PY{n}{solution}\PY{o}{.}\PY{n}{items}\PY{p}{(}\PY{p}{)}\PY{p}{\PYZcb{}}

\PY{k}{for} \PY{n}{key}\PY{p}{,} \PY{n}{value} \PY{o+ow}{in} \PY{n}{simplified\PYZus{}solution}\PY{o}{.}\PY{n}{items}\PY{p}{(}\PY{p}{)}\PY{p}{:}
    \PY{n+nb}{print}\PY{p}{(}\PY{l+s+sa}{f}\PY{l+s+s2}{\PYZdq{}}\PY{l+s+si}{\PYZob{}}\PY{n}{key}\PY{l+s+si}{\PYZcb{}}\PY{l+s+s2}{: }\PY{l+s+si}{\PYZob{}}\PY{n}{value}\PY{l+s+si}{\PYZcb{}}\PY{l+s+s2}{\PYZdq{}}\PY{p}{)}
\end{Verbatim}
\end{tcolorbox}

    \begin{Verbatim}[commandchars=\\\{\}]
A0: (-3*a**3 - a**2*b - a*b**2 + b**3)/(2*(a**2 - 2*a*b + b**2))
A1: (a**3 + 7*a**2*b - 5*a*b**2 + b**3)/(2*(a**2 - 2*a*b + b**2))
A2: (7*a**3 + 3*a**2*b + 3*a*b**2 - b**3)/(12*(a - b))
A3: (17*a**3 - 3*a**2*b - 3*a*b**2 + b**3)/(12*(a - b))
    \end{Verbatim}

    Получили, что \[A_0 = \frac{(-3) a^3 - a^2b-ab^2+b^3}{2(a^2-2ab+b^2)},\]
\[A_1 = \frac{a^3+7a^2b-5ab^2+b^3}{2(a^2-2ab+b^2)},\]
\[A_2 = \frac{7a^3+3a^2b+3ab^2-b^3}{12(a-b)},\]
\[A_3 = \frac{17a^3-3a^2b-3ab^2+b^3}{12(a-b)}.\] Проверим полученное
решение с помощью тех же инструментов Python

    \begin{tcolorbox}[breakable, size=fbox, boxrule=1pt, pad at break*=1mm,colback=cellbackground, colframe=cellborder]
\prompt{In}{incolor}{6}{\boxspacing}
\begin{Verbatim}[commandchars=\\\{\}]
\PY{n}{eq1\PYZus{}sub} \PY{o}{=} \PY{n}{sp}\PY{o}{.}\PY{n}{simplify}\PY{p}{(}\PY{n}{eq1}\PY{o}{.}\PY{n}{subs}\PY{p}{(}\PY{n}{solution}\PY{p}{)}\PY{p}{)}
\PY{n}{eq2\PYZus{}sub} \PY{o}{=} \PY{n}{sp}\PY{o}{.}\PY{n}{simplify}\PY{p}{(}\PY{n}{eq2}\PY{o}{.}\PY{n}{subs}\PY{p}{(}\PY{n}{solution}\PY{p}{)}\PY{p}{)}
\PY{n}{eq3\PYZus{}sub} \PY{o}{=} \PY{n}{sp}\PY{o}{.}\PY{n}{simplify}\PY{p}{(}\PY{n}{eq3}\PY{o}{.}\PY{n}{subs}\PY{p}{(}\PY{n}{solution}\PY{p}{)}\PY{p}{)}
\PY{n}{eq4\PYZus{}sub} \PY{o}{=} \PY{n}{sp}\PY{o}{.}\PY{n}{simplify}\PY{p}{(}\PY{n}{eq4}\PY{o}{.}\PY{n}{subs}\PY{p}{(}\PY{n}{solution}\PY{p}{)}\PY{p}{)}

\PY{n+nb}{print}\PY{p}{(}\PY{n}{eq1\PYZus{}sub}\PY{p}{)}
\PY{n+nb}{print}\PY{p}{(}\PY{n}{eq2\PYZus{}sub}\PY{p}{)}
\PY{n+nb}{print}\PY{p}{(}\PY{n}{eq3\PYZus{}sub}\PY{p}{)}
\PY{n+nb}{print}\PY{p}{(}\PY{n}{eq4\PYZus{}sub}\PY{p}{)}
\end{Verbatim}
\end{tcolorbox}

    \begin{Verbatim}[commandchars=\\\{\}]
True
True
True
True
    \end{Verbatim}

    Таким образом, решение получено правильно и, соответственно,
квадратурная форма будет иметь вид:
$$\int\limits_a^b f(x)dx \approx \frac{(-3) a^3 - a^2b-ab^2+b^3}{2(a^2-2ab+b^2)} f(a) + \frac{a^3+7a^2b-5ab^2+b^3}{2(a^2-2ab+b^2)} f(b) + $$
$$+\frac{7a^3+3a^2b+3ab^2-b^3}{12(a-b)} f'(a) + \frac{17a^3-3a^2b-3ab^2+b^3}{12(a-b)} f'(b)$$
Найдем АСТ для получившейся квадратурной формулы, для этого построим
соотношение \(x^4:\)
\[\int\limits_a^b x^4 dx = A_0a^4 + A_1b^4 + 4A_2a^3 + 4A_3b^3.\]
Подставим коэффициенты и вычислим интеграл
$$\frac{(b-a)^5}{5} = \frac{(-3) a^3 - a^2b-ab^2+b^3}{2(a^2-2ab+b^2)} a^4 + \frac{a^3+7a^2b-5ab^2+b^3}{2(a^2-2ab+b^2)} b^4 + $$
$$+4 \frac{7a^3+3a^2b+3ab^2-b^3}{12(a-b)} a^3 + 4\frac{17a^3-3a^2b-3ab^2+b^3}{12(a-b)} b^3$$
Посмотрим, выполняется ли равенство
    При подстановке мы получаем следующее
\[\frac{(a-b)^5}{5} \overset{?}{=} -\frac{10a^6-12a^5b-18a^4b^2+23a^3b^3-27a^2b^4+15ab^5-3b^6}{12(a-b)}.\]
Раскроем левую часть
\[\frac{(a-b)^5}{5} = \frac{a^5-5a^4b+10a^3b^2-10a^2b^3+5ab^4-b^5}{5}\]
\[\frac{a^5-5a^4b+10a^3b^2-10a^2b^3+5ab^4-b^5}{5} \neq -\frac{10a^6-12a^5b-18a^4b^2+23a^3b^3-27a^2b^4+15ab^5-3b^6}{12(a-b)}\]
Таким образом, алгебраическая степень точности равна тому $i$, для которого равенство выполнялось, в нашем случае это значение равно 3.

    \subsection*{Задача 2}

Рассмотрим обший вид квадратурной формулы
\[I(f) = \int\limits_a^b \rho(x)f(x)dx \approx A_0 f(x_0) + A_1 f(x_1) + A_2 f(x_2) + A_3 f(x_3).\]

Для построения квадратурной формы с алгребраической степенью точности
\(m\) необходимо составить соотношения 
$$\begin{cases}
\int\limits_a^b \rho(x) x^idx = \sum\limits_{k=0}^{n}A_kx^i_k,\quad i=\overline{0,m},\\
\int\limits_a^b \rho(x) x^{m+1}dx \ne \sum\limits_{k=0}^{n}A_kx^i_k;
\end{cases}$$
Таким образом, в нашем примере мы имеем
\[[a, b] = [-1, 1], \ \rho(x) = 1,\]
\[A_0 = A_1 = \frac{1}{6},\  A_2 = A_3 = \frac{5}{6},\]
\[x_0 = -1, \ x_1 = 1, \ x_2 = -\frac{1}{\sqrt{5}},\  x_3 = \frac{1}{\sqrt{5}}.\]
Для определения алгебраической степени точности, необходимо строить по
одному уравнению из нашего соотношения до тех пор, пока равенство не
обратится в неравенство.

Найдем решение интеграла для любого \(i:\)
\[\int\limits_{-1}^1 x^i dx = \dfrac{x^{i+1}}{i+1}\bigg|^{1}_{-1} = \dfrac{1 - (-1)^{i+1}}{i+1}\]
Подставим соотношение для \(i-\)го порядка:
\[\dfrac{1 - (-1)^{i+1}}{i+1} = \frac{1}{6} \cdot (-1)^{i} + \frac{1}{6} \cdot 1 + \frac{5}{6} \cdot \Bigr(-\frac{1}{\sqrt{5}} \Bigl)^i + \frac{5}{6} \cdot \Bigr(\frac{1}{\sqrt{5}} \Bigl)^i\]
Нетрудно заметить, что при нечетных \(i\) и левая, и правая части будут
равны \(0\), поэтому сразу будем рассматривать четные \(i\), при которых
получим \[\dfrac{2}{i+1} = \frac{1}{3} + \frac{5}{3 \cdot(\sqrt{5})^i}\]

Начнем с \(i=0:\)
\[x^0: 2 \overset{?}{=} \frac{1}{6} + \frac{1}{6} + \frac{5}{6} + \frac{5}{6} = 2 \Rightarrow \text{Равенство выполняется}.\]
\[x^2: \dfrac{2}{3} \overset{?}{=} \frac{1}{3} + \frac{5}{3 \cdot 5} = \dfrac{2}{3}\Rightarrow \text{Равенство выполняется}.\]
\[x^4: \dfrac{2}{5} \overset{?}{=} \frac{1}{3} + \frac{5}{3 \cdot 5 \cdot {\sqrt5}} = \frac{1}{3} + \frac{1}{3 \cdot {\sqrt5}} \neq \dfrac{2}{5} \Rightarrow \text{Равенство не выполняется}.\]

Таким образом, АСТ нашей квадратурной формулы равна 3, поскольку, как ранее было отмечено, при $i=3$ мы получим равенство $0=0$.

\subsection*{Задача 3}

Зададим нашу функцию \(f(x)=\dfrac{\ln(\sin^2x + 3)}{x^2+2x-1}\)
программно

    \begin{tcolorbox}[breakable, size=fbox, boxrule=1pt, pad at break*=1mm,colback=cellbackground, colframe=cellborder]
\prompt{In}{incolor}{11}{\boxspacing}
\begin{Verbatim}[commandchars=\\\{\}]
\PY{k}{def} \PY{n+nf}{f}\PY{p}{(}\PY{n}{x}\PY{p}{)}\PY{p}{:}
    \PY{k}{return} \PY{n}{math}\PY{o}{.}\PY{n}{log}\PY{p}{(}\PY{n}{math}\PY{o}{.}\PY{n}{sin}\PY{p}{(}\PY{n}{x}\PY{p}{)}\PY{o}{*}\PY{o}{*}\PY{l+m+mi}{2} \PY{o}{+} \PY{l+m+mi}{3}\PY{p}{)} \PY{o}{/} \PY{p}{(}\PY{n}{x}\PY{o}{*}\PY{o}{*}\PY{l+m+mi}{2} \PY{o}{+} \PY{l+m+mi}{2}\PY{o}{*}\PY{n}{x} \PY{o}{\PYZhy{}} \PY{l+m+mi}{1}\PY{p}{)}

\PY{n}{a}\PY{p}{,} \PY{n}{b}\PY{p}{,} \PY{n}{n} \PY{o}{=} \PY{l+m+mi}{1}\PY{p}{,} \PY{l+m+mi}{3}\PY{p}{,} \PY{l+m+mi}{10}
\end{Verbatim}
\end{tcolorbox}

    Рассмотрим общий вид составной квадратурной формулы средних
прямоугольников
\[I_{\text{сс}}=h \sum_{k=0}^{N-1} f(a+(k + \dfrac{1}{2})h),\] и зададим
его программно.

    \begin{tcolorbox}[breakable, size=fbox, boxrule=1pt, pad at break*=1mm,colback=cellbackground, colframe=cellborder]
\prompt{In}{incolor}{13}{\boxspacing}
\begin{Verbatim}[commandchars=\\\{\}]
\PY{k}{def} \PY{n+nf}{mean\PYZus{}rect}\PY{p}{(}\PY{n}{a}\PY{p}{,} \PY{n}{b}\PY{p}{,} \PY{n}{f}\PY{p}{,} \PY{n}{h}\PY{p}{)}\PY{p}{:}
    \PY{n}{I} \PY{o}{=} \PY{l+m+mi}{0}
    \PY{n}{N} \PY{o}{=} \PY{n+nb}{int}\PY{p}{(}\PY{p}{(}\PY{n}{b} \PY{o}{\PYZhy{}} \PY{n}{a}\PY{p}{)} \PY{o}{/} \PY{n}{h}\PY{p}{)}
    
    \PY{k}{for} \PY{n}{k} \PY{o+ow}{in} \PY{n+nb}{range}\PY{p}{(}\PY{n}{N}\PY{p}{)}\PY{p}{:}
        \PY{n}{I} \PY{o}{+}\PY{o}{=} \PY{n}{f}\PY{p}{(}\PY{n}{a} \PY{o}{+} \PY{p}{(}\PY{n}{k} \PY{o}{+} \PY{l+m+mi}{1}\PY{o}{/}\PY{l+m+mi}{2}\PY{p}{)} \PY{o}{*} \PY{n}{h}\PY{p}{)}
        
    \PY{k}{return} \PY{n}{h} \PY{o}{*} \PY{n}{I}
\end{Verbatim}
\end{tcolorbox}

    Рассмотрим так же составную квадратурную формулу трапеций
\[I_{\text{тс}} = h \Bigr[ \dfrac{f(a) + f(b)}{2} + \sum_{k=0}^{N-1}f(x_k)  \Bigl]\]

    \begin{tcolorbox}[breakable, size=fbox, boxrule=1pt, pad at break*=1mm,colback=cellbackground, colframe=cellborder]
\prompt{In}{incolor}{15}{\boxspacing}
\begin{Verbatim}[commandchars=\\\{\}]
\PY{k}{def} \PY{n+nf}{trap}\PY{p}{(}\PY{n}{a}\PY{p}{,} \PY{n}{b}\PY{p}{,} \PY{n}{f}\PY{p}{,} \PY{n}{h}\PY{p}{)}\PY{p}{:}
    \PY{n}{I} \PY{o}{=} \PY{l+m+mi}{0}
    \PY{n}{N} \PY{o}{=} \PY{n+nb}{int}\PY{p}{(}\PY{p}{(}\PY{n}{b} \PY{o}{\PYZhy{}} \PY{n}{a}\PY{p}{)} \PY{o}{/} \PY{n}{h}\PY{p}{)}
    
    \PY{n}{x} \PY{o}{=} \PY{n}{np}\PY{o}{.}\PY{n}{linspace}\PY{p}{(}\PY{n}{a}\PY{p}{,} \PY{n}{b}\PY{p}{,} \PY{n}{N}\PY{p}{)}
    
    \PY{k}{for} \PY{n}{k} \PY{o+ow}{in} \PY{n+nb}{range}\PY{p}{(}\PY{n}{N}\PY{p}{)}\PY{p}{:}
        \PY{n}{I} \PY{o}{+}\PY{o}{=} \PY{n}{f}\PY{p}{(}\PY{n}{x}\PY{p}{[}\PY{n}{k}\PY{p}{]}\PY{p}{)}

    \PY{k}{return} \PY{n}{h} \PY{o}{*} \PY{p}{(}\PY{p}{(}\PY{n}{f}\PY{p}{(}\PY{n}{a}\PY{p}{)} \PY{o}{+} \PY{n}{f}\PY{p}{(}\PY{n}{b}\PY{p}{)}\PY{p}{)} \PY{o}{/} \PY{l+m+mi}{2} \PY{o}{+} \PY{n}{I}\PY{p}{)}
\end{Verbatim}
\end{tcolorbox}

    Для использования правила Рунге используем выражение для главной части
остатка квадратурной формулы
\[R(h, f) \approx \dfrac{I_{h_2} - I_{h_1}}{1 - \Bigr( \dfrac{h_2}{h_1} \Bigl)^m}\]

\(m \ -\) алгебраическая степень точности методов, которая в случае состовной квадратурной формулы средних прямоугольников равна 2, для составной квадратурной формулы трапеций $-$ 2.

Подбирать шаги будем следующим образом
\[h_1 = \dfrac{b-a}{N}, \ h_2 = \dfrac{h_1}{2}\]

Посмотрим, для какой квадратурной формулы мы быстрее сможем подобрать
такие шаги \(h_1, h_2\), при которых \[|R(h,f)| \leq \epsilon.\]
Погрешность в этом задании возьмем \(\epsilon = 10^{-4}\), начальные
шаги \[h_1=b-a, h_2 = \frac{h_1}{2}\] подбор будем делать по правилу
\[|R(h,f)| \nleq \epsilon \Rightarrow h_1 = h_2, h_2 = \frac{h_1}{2}\]
Если же неравенство будет выполняться, то мы подобрали шаг при котором,
достигается нужная точность. Зададим правило Рунге программно

    \begin{tcolorbox}[breakable, size=fbox, boxrule=1pt, pad at break*=1mm,colback=cellbackground, colframe=cellborder]
\prompt{In}{incolor}{17}{\boxspacing}
\begin{Verbatim}[commandchars=\\\{\}]
\PY{k}{def} \PY{n+nf}{runge\PYZus{}rule}\PY{p}{(}\PY{n}{m}\PY{p}{,} \PY{n}{a}\PY{p}{,} \PY{n}{b}\PY{p}{,} \PY{n}{I}\PY{p}{,} \PY{n}{f}\PY{p}{,} \PY{n}{epsilon} \PY{o}{=} \PY{l+m+mf}{1e\PYZhy{}4}\PY{p}{)}\PY{p}{:}
    \PY{n}{h1} \PY{o}{=} \PY{n}{b} \PY{o}{\PYZhy{}} \PY{n}{a}
    \PY{n}{h2} \PY{o}{=} \PY{n}{h1} \PY{o}{/} \PY{l+m+mi}{2}
    
    \PY{n}{R} \PY{o}{=} \PY{p}{(}\PY{n}{I}\PY{p}{(}\PY{n}{a}\PY{p}{,} \PY{n}{b}\PY{p}{,} \PY{n}{f}\PY{p}{,} \PY{n}{h1}\PY{p}{)} \PY{o}{\PYZhy{}} \PY{n}{I}\PY{p}{(}\PY{n}{a}\PY{p}{,} \PY{n}{b}\PY{p}{,} \PY{n}{f}\PY{p}{,} \PY{n}{h2}\PY{p}{)}\PY{p}{)} \PY{o}{/} \PY{p}{(}\PY{l+m+mi}{1} \PY{o}{\PYZhy{}} \PY{p}{(}\PY{n}{h2} \PY{o}{/} \PY{n}{h1}\PY{p}{)}\PY{o}{*}\PY{o}{*}\PY{n}{m}\PY{p}{)}
    \PY{n}{array} \PY{o}{=} \PY{p}{[}\PY{n}{R}\PY{p}{]}
    \PY{n}{h\PYZus{}1} \PY{o}{=} \PY{p}{[}\PY{n}{h1}\PY{p}{]}
    \PY{n}{h\PYZus{}2} \PY{o}{=} \PY{p}{[}\PY{n}{h2}\PY{p}{]}
    
    \PY{k}{while} \PY{n+nb}{abs}\PY{p}{(}\PY{n}{R}\PY{p}{)} \PY{o}{\PYZgt{}} \PY{n}{epsilon}\PY{p}{:}
        \PY{n}{h1} \PY{o}{=} \PY{n}{h2}
        \PY{n}{h2} \PY{o}{=} \PY{n}{h1} \PY{o}{/} \PY{l+m+mi}{2}
        
        \PY{n}{R} \PY{o}{=} \PY{p}{(}\PY{n}{I}\PY{p}{(}\PY{n}{a}\PY{p}{,} \PY{n}{b}\PY{p}{,} \PY{n}{f}\PY{p}{,} \PY{n}{h1}\PY{p}{)} \PY{o}{\PYZhy{}} \PY{n}{I}\PY{p}{(}\PY{n}{a}\PY{p}{,} \PY{n}{b}\PY{p}{,} \PY{n}{f}\PY{p}{,} \PY{n}{h2}\PY{p}{)}\PY{p}{)} \PY{o}{/} \PY{p}{(}\PY{l+m+mi}{1} \PY{o}{\PYZhy{}} \PY{p}{(}\PY{n}{h2} \PY{o}{/} \PY{n}{h1}\PY{p}{)}\PY{o}{*}\PY{o}{*}\PY{n}{m}\PY{p}{)}
        
        \PY{n}{array}\PY{o}{.}\PY{n}{append}\PY{p}{(}\PY{n}{R}\PY{p}{)}
        
        \PY{n}{h\PYZus{}1}\PY{o}{.}\PY{n}{append}\PY{p}{(}\PY{n}{h1}\PY{p}{)}
        \PY{n}{h\PYZus{}2}\PY{o}{.}\PY{n}{append}\PY{p}{(}\PY{n}{h2}\PY{p}{)}
        
    \PY{k}{return} \PY{n}{h\PYZus{}1}\PY{p}{,} \PY{n}{h\PYZus{}2}\PY{p}{,} \PY{n}{array}\PY{p}{,} \PY{p}{(}\PY{n}{I}\PY{p}{(}\PY{n}{a}\PY{p}{,} \PY{n}{b}\PY{p}{,} \PY{n}{f}\PY{p}{,} \PY{n}{h1}\PY{p}{)} \PY{o}{+} \PY{n}{R}\PY{p}{)}
\end{Verbatim}
\end{tcolorbox}

    \begin{tcolorbox}[breakable, size=fbox, boxrule=1pt, pad at break*=1mm,colback=cellbackground, colframe=cellborder]
\prompt{In}{incolor}{18}{\boxspacing}
\begin{Verbatim}[commandchars=\\\{\}]
\PY{n}{m} \PY{o}{=} \PY{n+nb}{max}\PY{p}{(}\PY{n+nb}{len}\PY{p}{(}\PY{n}{runge\PYZus{}rule}\PY{p}{(}\PY{l+m+mi}{2}\PY{p}{,} \PY{n}{a}\PY{p}{,} \PY{n}{b}\PY{p}{,} \PY{n}{mean\PYZus{}rect}\PY{p}{,} \PY{n}{f}\PY{p}{)}\PY{p}{[}\PY{l+m+mi}{2}\PY{p}{]}\PY{p}{)}\PY{p}{,} \PY{n+nb}{len}\PY{p}{(}\PY{n}{runge\PYZus{}rule}\PY{p}{(}\PY{l+m+mi}{2}\PY{p}{,} \PY{n}{a}\PY{p}{,} \PY{n}{b}\PY{p}{,} \PY{n}{trap}\PY{p}{,} \PY{n}{f}\PY{p}{)}\PY{p}{[}\PY{l+m+mi}{2}\PY{p}{]}\PY{p}{)}\PY{p}{)}

\PY{n}{h1} \PY{o}{=} \PY{n}{runge\PYZus{}rule}\PY{p}{(}\PY{l+m+mi}{2}\PY{p}{,} \PY{n}{a}\PY{p}{,} \PY{n}{b}\PY{p}{,} \PY{n}{trap}\PY{p}{,} \PY{n}{f}\PY{p}{)}\PY{p}{[}\PY{l+m+mi}{0}\PY{p}{]}

\PY{n}{df} \PY{o}{=} \PY{n}{pd}\PY{o}{.}\PY{n}{DataFrame}\PY{p}{(}\PY{n}{columns} \PY{o}{=} \PY{p}{[}\PY{l+s+s2}{\PYZdq{}}\PY{l+s+s2}{Количество узлов}\PY{l+s+s2}{\PYZdq{}}\PY{p}{,} \PY{l+s+s2}{\PYZdq{}}\PY{l+s+s2}{Трапеции}\PY{l+s+s2}{\PYZdq{}}\PY{p}{]}\PY{p}{)}

\PY{n}{N} \PY{o}{=} \PY{p}{[}\PY{l+m+mi}{0}\PY{p}{]} \PY{o}{*} \PY{n}{m}

\PY{k}{for} \PY{n}{i} \PY{o+ow}{in} \PY{n+nb}{range}\PY{p}{(}\PY{n}{m}\PY{p}{)}\PY{p}{:}
    \PY{n}{N}\PY{p}{[}\PY{n}{i}\PY{p}{]} \PY{o}{=} \PY{n+nb}{int}\PY{p}{(}\PY{p}{(}\PY{n}{b} \PY{o}{\PYZhy{}} \PY{n}{a}\PY{p}{)} \PY{o}{/} \PY{n}{h1}\PY{p}{[}\PY{n}{i}\PY{p}{]}\PY{p}{)}
    
\PY{n}{df}\PY{p}{[}\PY{l+s+s2}{\PYZdq{}}\PY{l+s+s2}{Количество узлов}\PY{l+s+s2}{\PYZdq{}}\PY{p}{]} \PY{o}{=} \PY{n}{N}  
\PY{n}{df}\PY{p}{[}\PY{l+s+s2}{\PYZdq{}}\PY{l+s+s2}{Трапеции}\PY{l+s+s2}{\PYZdq{}}\PY{p}{]} \PY{o}{=} \PY{n}{runge\PYZus{}rule}\PY{p}{(}\PY{l+m+mi}{2}\PY{p}{,} \PY{n}{a}\PY{p}{,} \PY{n}{b}\PY{p}{,} \PY{n}{trap}\PY{p}{,} \PY{n}{f}\PY{p}{)}\PY{p}{[}\PY{l+m+mi}{2}\PY{p}{]}

\PY{n}{df} \PY{o}{=} \PY{n}{pd}\PY{o}{.}\PY{n}{concat}\PY{p}{(}\PY{p}{[}\PY{n}{df}\PY{p}{,} \PY{n}{pd}\PY{o}{.}\PY{n}{DataFrame}\PY{p}{(}\PY{n}{runge\PYZus{}rule}\PY{p}{(}\PY{l+m+mi}{2}\PY{p}{,} \PY{n}{a}\PY{p}{,} \PY{n}{b}\PY{p}{,} \PY{n}{mean\PYZus{}rect}\PY{p}{,} \PY{n}{f}\PY{p}{)}\PY{p}{[}\PY{l+m+mi}{2}\PY{p}{]}\PY{p}{,} \PY{n}{columns}\PY{o}{=}\PY{p}{[}\PY{l+s+s2}{\PYZdq{}}\PY{l+s+s2}{Средние прямоугольники}\PY{l+s+s2}{\PYZdq{}}\PY{p}{]}\PY{p}{)}\PY{p}{]}\PY{p}{,} \PY{n}{axis}\PY{o}{=}\PY{l+m+mi}{1}\PY{p}{,} \PY{n}{verify\PYZus{}integrity}\PY{o}{=}\PY{k+kc}{True}\PY{p}{)}\PY{o}{.}\PY{n}{fillna}\PY{p}{(}\PY{l+s+s1}{\PYZsq{}}\PY{l+s+s1}{\PYZsq{}}\PY{p}{)}
\PY{n}{df}
\end{Verbatim}
\end{tcolorbox}

            \begin{tcolorbox}[breakable, size=fbox, boxrule=.5pt, pad at break*=1mm, opacityfill=0]
\prompt{Out}{outcolor}{18}{\boxspacing}
\begin{Verbatim}[commandchars=\\\{\}]
    Количество узлов  Трапеции Средние прямоугольники
0                  1  1.257883              -0.121278
1                  2  0.461537              -0.048862
2                  4  0.194290              -0.015571
3                  8  0.089161              -0.004272
4                 16  0.042803              -0.001098
5                 32  0.020992              -0.000277
6                 64  0.010399              -0.000069
7                128  0.005176
8                256  0.002582
9                512  0.001290
10              1024  0.000644
11              2048  0.000322
12              4096  0.000161
13              8192  0.000081
\end{Verbatim}
\end{tcolorbox}
        
    Как можно заметить, с увеличением количества узлов значение остатка по
правилу Рунге уменьшается как при использовании составной КФ средних
прямоугольников, так и составной КФ трапеций, однако можно увидеть, что
для КФ средних прямоугольников мы изначально начинали с меньшего по
модулю значения остатка и практически в два раза быстрее нам удалось
достигнуть нужной точности.

    \subsection*{Задача 4}

Для вычисления значения интеграла
\[I = \int\limits_{-1}^1 \frac{1}{\sqrt{(1-x^2)}} \dfrac{\sin x^2}{1 + \ln^2(x+1)}dx\]
с точностью \(\epsilon=10^{-4}\) воспользуемся квадратурными формулами
наивысшей алгебраической степени точности, или квадратурные формулы типа
Гаусса.

Поскольку мы сразу можем выделить весовую функцию
\(p(x)=\frac{1}{\sqrt{(1-x^2)}}\) и мы имеем отрезок интегрирования
\([a,b]=[-1,1]\), то мы сразу можем перейти к использованию квадратурной
формулы Эрмита
\[I(f)=\int_{-1}^1 \frac{1}{\sqrt{1-x^2}}f(x)dx \approx \sum_{k=0}^nA_kf(x_k)\]
В качестве узлов \(x_k\) выбираются \[x_k=\cos{\frac{2k+1}{2n+2} \pi}\]
Для \[A_k=\frac{\pi}{n+1}\] Остаток имеет вид
\[R_n(f) = \frac{\pi}{2^{2n+1}(2n+2)!} \cdot f^{2n+2}(\eta), \eta \in [-1,1].\]
Таким образом, получаем, что КФ имеет вид
\[I(f)=\int_{-1}^1 \frac{1}{\sqrt{1-x^2}}f(x)dx \approx \frac{\pi}{n+1}\sum_{k=0}^nf(\cos{\frac{2k+1}{2n+2} \pi})\]
Подставим нашу функцию \[f(x) =  \dfrac{\sin x^2}{1 + \ln^2(x+1)}\] и
получим
\[I(f)=\int_{-1}^1 \frac{1}{\sqrt{1-x^2}}f(x)dx \approx \frac{\pi}{n+1}\sum_{k=0}^n\dfrac{\sin (\cos{\frac{2k+1}{2n+2} \pi})^2}{1 + \ln^2(\cos{\frac{2k+1}{2n+2} \pi}+1)}\]
Погрешность будем оценивать следующим образом
\[\Delta = |I(f,n)-I(f,n-1)| < \epsilon = 10^{-4}\] Из этого неравенства
путем перебора найдем подходящее \(n\).

Перейдем к программной реализации.

    \begin{tcolorbox}[breakable, size=fbox, boxrule=1pt, pad at break*=1mm,colback=cellbackground, colframe=cellborder]
\prompt{In}{incolor}{21}{\boxspacing}
\begin{Verbatim}[commandchars=\\\{\}]
\PY{k}{def} \PY{n+nf}{u}\PY{p}{(}\PY{n}{x}\PY{p}{)}\PY{p}{:}
    \PY{k}{return} \PY{n}{np}\PY{o}{.}\PY{n}{sin}\PY{p}{(}\PY{n}{x}\PY{o}{*}\PY{o}{*}\PY{l+m+mi}{2}\PY{p}{)}\PY{o}{/}\PY{p}{(}\PY{l+m+mi}{1}\PY{o}{+}\PY{n}{np}\PY{o}{.}\PY{n}{log}\PY{p}{(}\PY{n}{x}\PY{o}{+}\PY{l+m+mi}{1}\PY{p}{)}\PY{o}{*}\PY{o}{*}\PY{l+m+mi}{2}\PY{p}{)}
\end{Verbatim}
\end{tcolorbox}

    \begin{tcolorbox}[breakable, size=fbox, boxrule=1pt, pad at break*=1mm,colback=cellbackground, colframe=cellborder]
\prompt{In}{incolor}{22}{\boxspacing}
\begin{Verbatim}[commandchars=\\\{\}]
\PY{k}{def} \PY{n+nf}{hermite\PYZus{}quadrature\PYZus{}formula}\PY{p}{(}\PY{n}{f}\PY{p}{,} \PY{n}{epsilon}\PY{o}{=}\PY{l+m+mf}{10e\PYZhy{}4}\PY{p}{)}\PY{p}{:}
    \PY{n}{n} \PY{o}{=} \PY{l+m+mi}{1}
    
    \PY{n}{I\PYZus{}new} \PY{o}{=} \PY{n}{np}\PY{o}{.}\PY{n}{inf}
    \PY{n}{I\PYZus{}last} \PY{o}{=} \PY{l+m+mi}{0}
    
    \PY{k}{while} \PY{n+nb}{abs}\PY{p}{(}\PY{n}{I\PYZus{}new} \PY{o}{\PYZhy{}} \PY{n}{I\PYZus{}last}\PY{p}{)} \PY{o}{\PYZgt{}}\PY{o}{=} \PY{n}{epsilon}\PY{p}{:}
        \PY{n}{I\PYZus{}last} \PY{o}{=} \PY{n}{I\PYZus{}new}
        \PY{n}{I\PYZus{}new} \PY{o}{=} \PY{l+m+mi}{0}
        
        \PY{k}{for} \PY{n}{k} \PY{o+ow}{in} \PY{n+nb}{range}\PY{p}{(}\PY{n}{n} \PY{o}{+} \PY{l+m+mi}{1}\PY{p}{)}\PY{p}{:}
            \PY{n}{x\PYZus{}k} \PY{o}{=} \PY{n}{np}\PY{o}{.}\PY{n}{cos}\PY{p}{(}\PY{p}{(}\PY{l+m+mi}{2} \PY{o}{*} \PY{n}{k} \PY{o}{+} \PY{l+m+mi}{1}\PY{p}{)} \PY{o}{/} \PY{p}{(}\PY{l+m+mi}{2} \PY{o}{*} \PY{n}{n} \PY{o}{+} \PY{l+m+mi}{2}\PY{p}{)} \PY{o}{*} \PY{n}{np}\PY{o}{.}\PY{n}{pi}\PY{p}{)}
            
            \PY{n}{A\PYZus{}k} \PY{o}{=} \PY{n}{np}\PY{o}{.}\PY{n}{pi} \PY{o}{/} \PY{p}{(}\PY{n}{n} \PY{o}{+} \PY{l+m+mi}{1}\PY{p}{)}
            
            \PY{n}{I\PYZus{}new} \PY{o}{+}\PY{o}{=} \PY{n}{A\PYZus{}k} \PY{o}{*} \PY{n}{f}\PY{p}{(}\PY{n}{x\PYZus{}k}\PY{p}{)}
            
        \PY{n}{n} \PY{o}{+}\PY{o}{=} \PY{l+m+mi}{1}
        
    \PY{k}{return} \PY{n}{I\PYZus{}new}\PY{p}{,} \PY{n}{n}
\end{Verbatim}
\end{tcolorbox}

    \begin{tcolorbox}[breakable, size=fbox, boxrule=1pt, pad at break*=1mm,colback=cellbackground, colframe=cellborder]
\prompt{In}{incolor}{23}{\boxspacing}
\begin{Verbatim}[commandchars=\\\{\}]
\PY{n}{hermite\PYZus{}quadrature\PYZus{}formula}\PY{p}{(}\PY{n}{u}\PY{p}{)}
\end{Verbatim}
\end{tcolorbox}

            \begin{tcolorbox}[breakable, size=fbox, boxrule=.5pt, pad at break*=1mm, opacityfill=0]
\prompt{Out}{outcolor}{23}{\boxspacing}
\begin{Verbatim}[commandchars=\\\{\}]
(0.6831692972144928, 7)
\end{Verbatim}
\end{tcolorbox}
        
    Таким образом, получаем, что приближенное значение интеграла равно
\[I(f) \approx 0.6832\] и получено оно было при \(n=7\) узлах.

    \subsection*{Задача 5}

Перепишем наше уравнение
\[\int\limits_{0}^X (t-1)^6 (\lg \sqrt{t^2 + 1} + 2)dt = 5\] в виде

\[\int\limits_{0}^X (t-1)^6 (\lg \sqrt{t^2 + 1} + 2)dt - 5 = f(x). \]

Таким образом, решение этого уравнения совпадает с решением нелинейного
уравнения \[f(x)=0,\] которое будем решать методом Ньютона.

    Вспомним первую лабораторную работу, уравнение будет иметь единственный
корень на отрезке \([a, b]\), если функция на концах отрезка имеет
разные по знаку значения и является монотонной.

Начнем с монотонности функции, для этого рассмотрим ее производную,
вычисленную по теореме Барроу
\[f'(x) = (x-1)^6 (\lg \sqrt{x^2 + 1} + 2).\]

Поскольку функции \((x-1)^6\) и \(\lg \sqrt{x^2 + 1} + 2\) являются
строго положительными на отрезке \([0, +\infty),\) тогда и
\[f'(x)>0, \ x \in [0, +\infty],\] следовательно, наша исходная функци
монотонна на нашем множестве.

Для определения знака на концах отрезка, необходимо приближенно
вычислить интеграл
\[\int\limits_{0}^X (t-1)^6 (\lg \sqrt{t^2 + 1} + 2)dt.\]

Для этого определим подынтегральную функцию
\[g(x) = (t-1)^6 (\lg \sqrt{t^2 + 1} + 2).\]

    \begin{tcolorbox}[breakable, size=fbox, boxrule=1pt, pad at break*=1mm,colback=cellbackground, colframe=cellborder]
\prompt{In}{incolor}{27}{\boxspacing}
\begin{Verbatim}[commandchars=\\\{\}]
\PY{k}{def} \PY{n+nf}{g}\PY{p}{(}\PY{n}{x}\PY{p}{)}\PY{p}{:}
    \PY{k}{return} \PY{p}{(}\PY{n}{x} \PY{o}{\PYZhy{}} \PY{l+m+mi}{1}\PY{p}{)}\PY{o}{*}\PY{o}{*}\PY{l+m+mi}{6} \PY{o}{*} \PY{p}{(}\PY{n}{np}\PY{o}{.}\PY{n}{log}\PY{p}{(}\PY{n}{np}\PY{o}{.}\PY{n}{sqrt}\PY{p}{(}\PY{n}{x}\PY{o}{*}\PY{o}{*}\PY{l+m+mi}{2} \PY{o}{+} \PY{l+m+mi}{1}\PY{p}{)}\PY{p}{)} \PY{o}{+} \PY{l+m+mi}{2}\PY{p}{)}
\end{Verbatim}
\end{tcolorbox}

    Для вычисления интеграла воспользуемся правилом Рунге, а именно
приближенное значение будем искать в виде
\[\int\limits_{0}^X (t-1)^6 (\lg \sqrt{t^2 + 1} + 2)dt \approx I_h(f) + R(h,f),\]
где \(I_h(f)\) - квадратурная формула, \(R(h,f)\) - остаток составной
квадратурной формулы.

Опираясь на заданее 3, будем использовать составную формулу средних
прямоугольников, поскольку этот метод сошелся быстрее в рассматриваемом
случае, для \(\epsilon = 10^{-4}\).

Возьмем значение \(x=2\)

    \begin{tcolorbox}[breakable, size=fbox, boxrule=1pt, pad at break*=1mm,colback=cellbackground, colframe=cellborder]
\prompt{In}{incolor}{29}{\boxspacing}
\begin{Verbatim}[commandchars=\\\{\}]
\PY{k}{def} \PY{n+nf}{f}\PY{p}{(}\PY{n}{x}\PY{p}{)}\PY{p}{:}
    \PY{k}{return} \PY{n}{runge\PYZus{}rule}\PY{p}{(}\PY{l+m+mi}{2}\PY{p}{,} \PY{l+m+mi}{0}\PY{p}{,} \PY{n}{x}\PY{p}{,} \PY{n}{mean\PYZus{}rect}\PY{p}{,} \PY{n}{g}\PY{p}{)}\PY{p}{[}\PY{l+m+mi}{3}\PY{p}{]} \PY{o}{\PYZhy{}} \PY{l+m+mi}{5}
\end{Verbatim}
\end{tcolorbox}

    \begin{tcolorbox}[breakable, size=fbox, boxrule=1pt, pad at break*=1mm,colback=cellbackground, colframe=cellborder]
\prompt{In}{incolor}{30}{\boxspacing}
\begin{Verbatim}[commandchars=\\\{\}]
\PY{n+nb}{print}\PY{p}{(}\PY{n}{f}\PY{p}{(}\PY{l+m+mi}{2}\PY{p}{)}\PY{p}{)}
\end{Verbatim}
\end{tcolorbox}

    \begin{Verbatim}[commandchars=\\\{\}]
-4.319155993134725
    \end{Verbatim}

    То есть \[f(2) < 0.\] Так как мы помним, что наша функция является
момнотонно возрастающей, то все значения слева будут точно
отрицательными.

Для рассмотрения правого конца отрезка возьмем значение из отрезка
\((2, +\infty)\), например, \(x=2.5\)

    \begin{tcolorbox}[breakable, size=fbox, boxrule=1pt, pad at break*=1mm,colback=cellbackground, colframe=cellborder]
\prompt{In}{incolor}{32}{\boxspacing}
\begin{Verbatim}[commandchars=\\\{\}]
\PY{n+nb}{print}\PY{p}{(}\PY{n}{f}\PY{p}{(}\PY{l+m+mf}{2.5}\PY{p}{)}\PY{p}{)}
\end{Verbatim}
\end{tcolorbox}

    \begin{Verbatim}[commandchars=\\\{\}]
2.420954617125342
    \end{Verbatim}

    Таким образом, \[f(2.5) > 0,\] что, опираясь на монотонность функции
гарантирует, что во всех точках находящихся правее, значения функции
будут положительны.

Следовательно, мы получили, что на отрезке \([2, 2.5]\) наша функция
монотонна и имеет разные по знаку значения на его концах, что говорит о
том, что на данном отрезке есть единственный корень.

Сведем наш отрезок к минимум, применив метод деления отрезка пополам

    \begin{tcolorbox}[breakable, size=fbox, boxrule=1pt, pad at break*=1mm,colback=cellbackground, colframe=cellborder]
\prompt{In}{incolor}{34}{\boxspacing}
\begin{Verbatim}[commandchars=\\\{\}]
\PY{n}{a}\PY{p}{,} \PY{n}{b} \PY{o}{=} \PY{l+m+mi}{2}\PY{p}{,} \PY{l+m+mf}{2.5}
\PY{n}{epsilon\PYZus{}2} \PY{o}{=} \PY{l+m+mf}{1e\PYZhy{}1}

\PY{n}{dichotomy\PYZus{}table} \PY{o}{=} \PY{p}{[}\PY{p}{[}\PY{n}{a}\PY{p}{,} \PY{n}{b}\PY{p}{,} \PY{n}{b} \PY{o}{\PYZhy{}} \PY{n}{a}\PY{p}{,} \PY{n}{f}\PY{p}{(}\PY{n}{a}\PY{p}{)}\PY{p}{,} \PY{n}{f}\PY{p}{(}\PY{n}{b}\PY{p}{)} \PY{p}{,} \PY{p}{(}\PY{n}{a} \PY{o}{+} \PY{n}{b}\PY{p}{)}\PY{o}{/}\PY{l+m+mi}{2}\PY{p}{]}\PY{p}{]}
\PY{n}{c} \PY{o}{=} \PY{l+m+mi}{0}

\PY{k}{while} \PY{n}{b} \PY{o}{\PYZhy{}} \PY{n}{a} \PY{o}{\PYZgt{}} \PY{n}{epsilon\PYZus{}2}\PY{p}{:}
    \PY{n}{c} \PY{o}{=} \PY{p}{(}\PY{n}{a} \PY{o}{+} \PY{n}{b}\PY{p}{)} \PY{o}{/} \PY{l+m+mi}{2}
    
    \PY{k}{if} \PY{n}{f}\PY{p}{(}\PY{n}{c}\PY{p}{)} \PY{o}{*} \PY{n}{f}\PY{p}{(}\PY{n}{a}\PY{p}{)} \PY{o}{\PYZgt{}}\PY{o}{=} \PY{l+m+mi}{0}\PY{p}{:}
        \PY{n}{a} \PY{o}{=} \PY{n}{c}
        
    \PY{k}{else}\PY{p}{:}
        \PY{n}{b} \PY{o}{=} \PY{n}{c}
        
    \PY{n}{dichotomy\PYZus{}table}\PY{o}{.}\PY{n}{append}\PY{p}{(}\PY{p}{[}\PY{n}{a}\PY{p}{,} \PY{n}{b}\PY{p}{,} \PY{n}{b} \PY{o}{\PYZhy{}} \PY{n}{a}\PY{p}{,} \PY{n}{f}\PY{p}{(}\PY{n}{a}\PY{p}{)}\PY{p}{,} \PY{n}{f}\PY{p}{(}\PY{n}{b}\PY{p}{)} \PY{p}{,} \PY{p}{(}\PY{n}{a} \PY{o}{+} \PY{n}{b}\PY{p}{)} \PY{o}{/} \PY{l+m+mi}{2}\PY{p}{]}\PY{p}{)}
    
\PY{n}{pd}\PY{o}{.}\PY{n}{DataFrame}\PY{p}{(}\PY{n}{dichotomy\PYZus{}table}\PY{p}{,} \PY{n}{columns} \PY{o}{=} \PY{p}{[}\PY{l+s+s1}{\PYZsq{}}\PY{l+s+s1}{a}\PY{l+s+s1}{\PYZsq{}}\PY{p}{,} \PY{l+s+s1}{\PYZsq{}}\PY{l+s+s1}{b}\PY{l+s+s1}{\PYZsq{}}\PY{p}{,} \PY{l+s+s1}{\PYZsq{}}\PY{l+s+s1}{b\PYZhy{}a}\PY{l+s+s1}{\PYZsq{}}\PY{p}{,} \PY{l+s+s1}{\PYZsq{}}\PY{l+s+s1}{f(a)}\PY{l+s+s1}{\PYZsq{}}\PY{p}{,} \PY{l+s+s1}{\PYZsq{}}\PY{l+s+s1}{f(b)}\PY{l+s+s1}{\PYZsq{}}\PY{p}{,} \PY{l+s+s1}{\PYZsq{}}\PY{l+s+s1}{(a+b)/2}\PY{l+s+s1}{\PYZsq{}}\PY{p}{]}\PY{p}{)}
\end{Verbatim}
\end{tcolorbox}

            \begin{tcolorbox}[breakable, size=fbox, boxrule=.5pt, pad at break*=1mm, opacityfill=0]
\prompt{Out}{outcolor}{34}{\boxspacing}
\begin{Verbatim}[commandchars=\\\{\}]
       a       b     b-a      f(a)      f(b)  (a+b)/2
0  2.000  2.5000  0.5000 -4.319156  2.420955  2.25000
1  2.250  2.5000  0.2500 -2.777547  2.420955  2.37500
2  2.375  2.5000  0.1250 -0.886425  2.420955  2.43750
3  2.375  2.4375  0.0625 -0.886425  0.546916  2.40625
\end{Verbatim}
\end{tcolorbox}
        
    Таким образом, получаем, что корень находится на отрезке
\([2.375, 2,4375]\).

Вспоминая метод Ньютона, имеем следующую формулу
\[x^{k+1} = x^k - \dfrac{f(x^k)}{f'(x^0)},\quad k = 0,1,\ldots;\quad x_0\]
Так же рассмотрим теорему о сходимости метода Ньютона

\begin{theorem}[о сходимости метода Ньютона]
Пусть выполняются следующие условия:
\begin{enumerate}
\item Функция $f(x)$ определена и дважды непрерывно дифференцируема на отрезке $$s_0 = [x^0; x^0 + 2h_0],\quad h_0 =- \dfrac{f(x^0)}{f'(x^0)}.$$ При этом на концах отрезка $f(x)f'(x)\ne 0$.
\item Для начального приближения $x^0$ выполняется неравенство $$2|h_0|M \leq |f'(x_0)|,\quad M = \underset{x\in s_0}{\max}|f''(x)|.$$
\end{enumerate}
Тогда справедливы следующие утверждения: 
\begin{enumerate}
\item Внутри отрезка $s_0$ уравнение $f(x) = 0$ имеет корень $x^*$ и при этом этот корень единственный.
\item Последовательность приближений $x^k$, $k=1,2,\ldots$ может быть построена по формуле с заданным приближением $x^0$.
\item Последовательность $x^k$ сходится к корню $x^*$, то есть $x^k \xrightarrow[k\to\infty]{}x^*$.
\item Скорость сходимости характеризуется неравенством $$|x^* - x^{k+1}|\leq |x^{k+1} - x^k|\leq \dfrac{M}{2|f'(x^*)|}\cdot (x^k-x^{k-1})^2,\quad k=0,1,2,\ldots\eqno(4)$$
\end{enumerate}
\end{theorem}

Выберем \(x_0\) из нашего отрезка, \(x_0 = 2.4\) и перейдем к проверке
условий теоремы

    \begin{tcolorbox}[breakable, size=fbox, boxrule=1pt, pad at break*=1mm,colback=cellbackground, colframe=cellborder]
\prompt{In}{incolor}{36}{\boxspacing}
\begin{Verbatim}[commandchars=\\\{\}]
\PY{n}{x0} \PY{o}{=} \PY{l+m+mf}{2.4}
\PY{n+nb}{print}\PY{p}{(}\PY{l+s+sa}{f}\PY{l+s+s1}{\PYZsq{}}\PY{l+s+s1}{Начальное приближение: }\PY{l+s+si}{\PYZob{}}\PY{n}{x0}\PY{l+s+si}{\PYZcb{}}\PY{l+s+s1}{\PYZsq{}}\PY{p}{)}

\PY{n}{h0} \PY{o}{=} \PY{o}{\PYZhy{}}\PY{n}{f}\PY{p}{(}\PY{n}{x0}\PY{p}{)} \PY{o}{/} \PY{n}{g}\PY{p}{(}\PY{n}{x0}\PY{p}{)}
\PY{n+nb}{print}\PY{p}{(}\PY{l+s+sa}{f}\PY{l+s+s1}{\PYZsq{}}\PY{l+s+s1}{h\PYZus{}0: }\PY{l+s+si}{\PYZob{}}\PY{n}{h0}\PY{l+s+si}{\PYZcb{}}\PY{l+s+s1}{\PYZsq{}}\PY{p}{)}

\PY{n}{s0} \PY{o}{=} \PY{n}{np}\PY{o}{.}\PY{n}{linspace}\PY{p}{(}\PY{n}{x0}\PY{p}{,} \PY{n}{x0} \PY{o}{+} \PY{l+m+mi}{2} \PY{o}{*} \PY{n}{h0}\PY{p}{,} \PY{l+m+mi}{1000}\PY{p}{)}
\PY{n+nb}{print}\PY{p}{(}\PY{l+s+s1}{\PYZsq{}}\PY{l+s+s1}{s\PYZus{}0 = [}\PY{l+s+s1}{\PYZsq{}}\PY{p}{,} \PY{n}{s0}\PY{p}{[}\PY{l+m+mi}{0}\PY{p}{]}\PY{p}{,} \PY{l+s+s1}{\PYZsq{}}\PY{l+s+s1}{;}\PY{l+s+s1}{\PYZsq{}}\PY{p}{,} \PY{n}{s0}\PY{p}{[}\PY{o}{\PYZhy{}}\PY{l+m+mi}{1}\PY{p}{]}\PY{p}{,} \PY{l+s+s1}{\PYZsq{}}\PY{l+s+s1}{]}\PY{l+s+s1}{\PYZsq{}}\PY{p}{)}
\end{Verbatim}
\end{tcolorbox}

    \begin{Verbatim}[commandchars=\\\{\}]
Начальное приближение: 2.4
h\_0: 0.016168261001675316
s\_0 = [ 2.4 ; 2.4323365220033506 ]
    \end{Verbatim}

    Проверяем, чтобы на концах отрезка значение функции и ее производной не
обращались в ноль одновременно:

    \begin{tcolorbox}[breakable, size=fbox, boxrule=1pt, pad at break*=1mm,colback=cellbackground, colframe=cellborder]
\prompt{In}{incolor}{38}{\boxspacing}
\begin{Verbatim}[commandchars=\\\{\}]
\PY{n}{f}\PY{p}{(}\PY{n}{s0}\PY{p}{[}\PY{l+m+mi}{0}\PY{p}{]}\PY{p}{)} \PY{o}{*} \PY{n}{g}\PY{p}{(}\PY{n}{s0}\PY{p}{[}\PY{l+m+mi}{0}\PY{p}{]}\PY{p}{)}
\end{Verbatim}
\end{tcolorbox}

            \begin{tcolorbox}[breakable, size=fbox, boxrule=.5pt, pad at break*=1mm, opacityfill=0]
\prompt{Out}{outcolor}{38}{\boxspacing}
\begin{Verbatim}[commandchars=\\\{\}]
-8.00691153839716
\end{Verbatim}
\end{tcolorbox}
        
    \begin{tcolorbox}[breakable, size=fbox, boxrule=1pt, pad at break*=1mm,colback=cellbackground, colframe=cellborder]
\prompt{In}{incolor}{39}{\boxspacing}
\begin{Verbatim}[commandchars=\\\{\}]
\PY{n}{f}\PY{p}{(}\PY{n}{s0}\PY{p}{[}\PY{o}{\PYZhy{}}\PY{l+m+mi}{1}\PY{p}{]}\PY{p}{)} \PY{o}{*} \PY{n}{g}\PY{p}{(}\PY{n}{s0}\PY{p}{[}\PY{l+m+mi}{0}\PY{p}{]}\PY{p}{)}
\end{Verbatim}
\end{tcolorbox}

            \begin{tcolorbox}[breakable, size=fbox, boxrule=.5pt, pad at break*=1mm, opacityfill=0]
\prompt{Out}{outcolor}{39}{\boxspacing}
\begin{Verbatim}[commandchars=\\\{\}]
9.194062522206684
\end{Verbatim}
\end{tcolorbox}
        
    Функция на концах отрезка не обращается в ноль.

Вычислим вторую производную для функции \(f(x)\):

\[f''(x) = 3(x-1)^5 \ln{(x^2+1)} + \frac{(x-1)^5(12x^2+12)+(x-1)^6x}{x^2+1}\]

    \begin{tcolorbox}[breakable, size=fbox, boxrule=1pt, pad at break*=1mm,colback=cellbackground, colframe=cellborder]
\prompt{In}{incolor}{41}{\boxspacing}
\begin{Verbatim}[commandchars=\\\{\}]
\PY{k}{def} \PY{n+nf}{f\PYZus{}second\PYZus{}derivative}\PY{p}{(}\PY{n}{x}\PY{p}{)}\PY{p}{:}
    \PY{k}{return} \PY{l+m+mi}{3} \PY{o}{*} \PY{p}{(}\PY{n}{x} \PY{o}{\PYZhy{}} \PY{l+m+mi}{1}\PY{p}{)}\PY{o}{*}\PY{o}{*}\PY{l+m+mi}{5} \PY{o}{*} \PY{n}{np}\PY{o}{.}\PY{n}{log}\PY{p}{(}\PY{n}{x}\PY{o}{*}\PY{o}{*}\PY{l+m+mi}{2} \PY{o}{+} \PY{l+m+mi}{1}\PY{p}{)} \PY{o}{+} \PY{p}{(}\PY{p}{(}\PY{n}{x} \PY{o}{\PYZhy{}} \PY{l+m+mi}{1}\PY{p}{)}\PY{o}{*}\PY{o}{*}\PY{l+m+mi}{5} \PY{o}{*} \PY{p}{(}\PY{l+m+mi}{12} \PY{o}{*} \PY{n}{x}\PY{o}{*}\PY{o}{*}\PY{l+m+mi}{2} \PY{o}{+} \PY{l+m+mi}{12}\PY{p}{)} \PY{o}{+} \PY{p}{(}\PY{n}{x} \PY{o}{\PYZhy{}} \PY{l+m+mi}{1}\PY{p}{)}\PY{o}{*}\PY{o}{*}\PY{l+m+mi}{6} \PY{o}{*} \PY{n}{x}\PY{p}{)} \PY{o}{/} \PY{p}{(}\PY{n}{x}\PY{o}{*}\PY{o}{*}\PY{l+m+mi}{2} \PY{o}{+} \PY{l+m+mi}{1}\PY{p}{)}
\end{Verbatim}
\end{tcolorbox}

    Найдем \[M = \max_{x \in s_0}|f''(x)|\] и сразу проверим выполнение
условия

    \begin{tcolorbox}[breakable, size=fbox, boxrule=1pt, pad at break*=1mm,colback=cellbackground, colframe=cellborder]
\prompt{In}{incolor}{43}{\boxspacing}
\begin{Verbatim}[commandchars=\\\{\}]
\PY{n}{M} \PY{o}{=} \PY{n}{np}\PY{o}{.}\PY{n}{max}\PY{p}{(}\PY{n}{np}\PY{o}{.}\PY{n}{absolute}\PY{p}{(}\PY{n}{f\PYZus{}second\PYZus{}derivative}\PY{p}{(}\PY{n}{s0}\PY{p}{)}\PY{p}{)}\PY{p}{)}
\PY{n+nb}{print}\PY{p}{(}\PY{l+s+sa}{f}\PY{l+s+s1}{\PYZsq{}}\PY{l+s+s1}{M: }\PY{l+s+si}{\PYZob{}}\PY{n}{M}\PY{l+s+si}{\PYZcb{}}\PY{l+s+s1}{\PYZsq{}}\PY{p}{)}

\PY{l+m+mi}{2} \PY{o}{*} \PY{n}{np}\PY{o}{.}\PY{n}{absolute}\PY{p}{(}\PY{n}{h0}\PY{p}{)} \PY{o}{*} \PY{n}{M} \PY{o}{\PYZlt{}}\PY{o}{=} \PY{n}{np}\PY{o}{.}\PY{n}{absolute}\PY{p}{(}\PY{n}{g}\PY{p}{(}\PY{n}{x0}\PY{p}{)}\PY{p}{)}
\end{Verbatim}
\end{tcolorbox}

    \begin{Verbatim}[commandchars=\\\{\}]
M: 110.35791665977399
    \end{Verbatim}

            \begin{tcolorbox}[breakable, size=fbox, boxrule=.5pt, pad at break*=1mm, opacityfill=0]
\prompt{Out}{outcolor}{43}{\boxspacing}
\begin{Verbatim}[commandchars=\\\{\}]
True
\end{Verbatim}
\end{tcolorbox}
        
    Таким образом, получаем \(M = 110.35\), и необходимое уловие
выполняется. Что говорит о том, что все условия теоремы выполняются,
следовательно, метод Ньютона на отрезке \([2.375, 2.475]\) сойдется.

Перейдем непосредственно к реализации метода Ньютона

    \begin{tcolorbox}[breakable, size=fbox, boxrule=1pt, pad at break*=1mm,colback=cellbackground, colframe=cellborder]
\prompt{In}{incolor}{45}{\boxspacing}
\begin{Verbatim}[commandchars=\\\{\}]
\PY{k}{def} \PY{n+nf}{newton\PYZus{}method}\PY{p}{(}\PY{n}{x0}\PY{p}{,} \PY{n}{epsilon}\PY{o}{=}\PY{l+m+mf}{1e\PYZhy{}4}\PY{p}{,} \PY{n}{max\PYZus{}iterations}\PY{o}{=}\PY{l+m+mi}{100}\PY{p}{)}\PY{p}{:}
    \PY{n}{x\PYZus{}prev} \PY{o}{=} \PY{n}{x0}
    \PY{n}{x\PYZus{}next} \PY{o}{=} \PY{n}{x\PYZus{}prev} \PY{o}{\PYZhy{}} \PY{n}{f}\PY{p}{(}\PY{n}{x\PYZus{}prev}\PY{p}{)} \PY{o}{/} \PY{n}{g}\PY{p}{(}\PY{n}{x\PYZus{}prev}\PY{p}{)}
    \PY{n}{iterations} \PY{o}{=} \PY{l+m+mi}{1}
    
    \PY{n}{x\PYZus{}k} \PY{o}{=} \PY{p}{[}\PY{p}{]}
    
    \PY{k}{while} \PY{n+nb}{abs}\PY{p}{(}\PY{n}{x\PYZus{}next} \PY{o}{\PYZhy{}} \PY{n}{x\PYZus{}prev}\PY{p}{)} \PY{o}{\PYZgt{}} \PY{n}{epsilon} \PY{o+ow}{and} \PY{n}{iterations} \PY{o}{\PYZlt{}} \PY{n}{max\PYZus{}iterations}\PY{p}{:}
        \PY{n}{x\PYZus{}k}\PY{o}{.}\PY{n}{append}\PY{p}{(}\PY{p}{[}\PY{n}{x\PYZus{}next}\PY{p}{,} \PY{n+nb}{abs}\PY{p}{(}\PY{n}{x\PYZus{}next} \PY{o}{\PYZhy{}} \PY{n}{x\PYZus{}prev}\PY{p}{)}\PY{p}{]}\PY{p}{)}
        
        \PY{n}{x\PYZus{}prev} \PY{o}{=} \PY{n}{x\PYZus{}next}
        \PY{n}{x\PYZus{}next} \PY{o}{=} \PY{n}{x\PYZus{}prev} \PY{o}{\PYZhy{}} \PY{n}{f}\PY{p}{(}\PY{n}{x\PYZus{}prev}\PY{p}{)} \PY{o}{/} \PY{n}{g}\PY{p}{(}\PY{n}{x\PYZus{}prev}\PY{p}{)}
        \PY{n}{iterations} \PY{o}{+}\PY{o}{=} \PY{l+m+mi}{1}

    \PY{k}{if} \PY{n}{iterations} \PY{o}{==} \PY{n}{max\PYZus{}iterations}\PY{p}{:}
        \PY{n+nb}{print}\PY{p}{(}\PY{l+s+s2}{\PYZdq{}}\PY{l+s+s2}{Максимальное количество итераций достигнуто!}\PY{l+s+s2}{\PYZdq{}}\PY{p}{)}
    
    \PY{n}{x\PYZus{}k}\PY{o}{.}\PY{n}{append}\PY{p}{(}\PY{p}{[}\PY{n}{x\PYZus{}next}\PY{p}{,} \PY{n+nb}{abs}\PY{p}{(}\PY{n}{x\PYZus{}next} \PY{o}{\PYZhy{}} \PY{n}{x\PYZus{}prev}\PY{p}{)}\PY{p}{]}\PY{p}{)}
    \PY{k}{return} \PY{n}{x\PYZus{}k}\PY{p}{,} \PY{n}{x\PYZus{}next}

\PY{n}{x\PYZus{}k}\PY{p}{,} \PY{n}{x} \PY{o}{=} \PY{n}{newton\PYZus{}method}\PY{p}{(}\PY{n}{x0}\PY{p}{)}

\PY{n}{df} \PY{o}{=} \PY{n}{pd}\PY{o}{.}\PY{n}{DataFrame}\PY{p}{(}\PY{n}{x\PYZus{}k}\PY{p}{,} \PY{n}{columns}\PY{o}{=}\PY{p}{[}\PY{l+s+s1}{\PYZsq{}}\PY{l+s+s1}{Решение}\PY{l+s+s1}{\PYZsq{}}\PY{p}{,} \PY{l+s+s1}{\PYZsq{}}\PY{l+s+s1}{Погрешность}\PY{l+s+s1}{\PYZsq{}}\PY{p}{]}\PY{p}{)}
\PY{n}{df}
\end{Verbatim}
\end{tcolorbox}

            \begin{tcolorbox}[breakable, size=fbox, boxrule=.5pt, pad at break*=1mm, opacityfill=0]
\prompt{Out}{outcolor}{45}{\boxspacing}
\begin{Verbatim}[commandchars=\\\{\}]
    Решение   Погрешность
0  2.416168  1.616826e-02
1  2.415621  5.471388e-04
2  2.415620  6.594857e-07
\end{Verbatim}
\end{tcolorbox}
        
    Таким образом, мы смогли достичь нужной степени точности за три
итерации, и корнем нашего уравнения является значение
\[x \approx 2.415620.\]

Выполним проверку

    \begin{tcolorbox}[breakable, size=fbox, boxrule=1pt, pad at break*=1mm,colback=cellbackground, colframe=cellborder]
\prompt{In}{incolor}{73}{\boxspacing}
\begin{Verbatim}[commandchars=\\\{\}]
\PY{n}{f}\PY{p}{(}\PY{l+m+mf}{2.415620}\PY{p}{)}
\end{Verbatim}
\end{tcolorbox}

            \begin{tcolorbox}[breakable, size=fbox, boxrule=.5pt, pad at break*=1mm, opacityfill=0]
\prompt{Out}{outcolor}{73}{\boxspacing}
\begin{Verbatim}[commandchars=\\\{\}]
-1.1026360226651377e-05
\end{Verbatim}
\end{tcolorbox}
        
    Полученное решение действительно является решением уравнения при
заданной точности \(\epsilon = 10^{-4}\).


    % Add a bibliography block to the postdoc
    
    
    
\end{document}