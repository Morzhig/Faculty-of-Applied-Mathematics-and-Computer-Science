\DeclareUnicodeCharacter{041B}{\CYRL}



\documentclass[11pt]{article}

    
    
    \usepackage[breakable]{tcolorbox}
    \usepackage{parskip} % Stop auto-indenting (to mimic markdown behaviour)
    \usepackage[english,russian]{babel}
    \usepackage{graphicx}
    \usepackage{subcaption}
    

    % Basic figure setup, for now with no caption control since it's done
    % automatically by Pandoc (which extracts ![](path) syntax from Markdown).
    \usepackage{graphicx}
    % Maintain compatibility with old templates. Remove in nbconvert 6.0
    \let\Oldincludegraphics\includegraphics
    % Ensure that by default, figures have no caption (until we provide a
    % proper Figure object with a Caption API and a way to capture that
    % in the conversion process - todo).
    \usepackage{caption}
    \DeclareCaptionFormat{nocaption}{}
    \captionsetup{format=nocaption,aboveskip=0pt,belowskip=0pt}

    \usepackage{float}
    \floatplacement{figure}{H} % forces figures to be placed at the correct location
    \usepackage{xcolor} % Allow colors to be defined
    \usepackage{enumerate} % Needed for markdown enumerations to work
    \usepackage{geometry} % Used to adjust the document margins
    \usepackage{amsmath} % Equations
    \usepackage{amssymb} % Equations
    \usepackage{textcomp} % defines textquotesingle
    % Hack from http://tex.stackexchange.com/a/47451/13684:
    \AtBeginDocument{%
        \def\PYZsq{\textquotesingle}% Upright quotes in Pygmentized code
    }
    \usepackage{upquote} % Upright quotes for verbatim code
    \usepackage{eurosym} % defines \euro

    \usepackage{iftex}
    \ifPDFTeX
        \usepackage[T1]{fontenc}
        \IfFileExists{alphabeta.sty}{
              \usepackage{alphabeta}
          }{
              \usepackage[mathletters]{ucs}
              \usepackage[utf8x]{inputenc}
          }
    \else
        \usepackage{fontspec}
        \usepackage{unicode-math}
    \fi

    \usepackage{fancyvrb} % verbatim replacement that allows latex
    \usepackage{grffile} % extends the file name processing of package graphics
                         % to support a larger range
    \makeatletter % fix for old versions of grffile with XeLaTeX
    \@ifpackagelater{grffile}{2019/11/01}
    {
      % Do nothing on new versions
    }
    {
      \def\Gread@@xetex#1{%
        \IfFileExists{"\Gin@base".bb}%
        {\Gread@eps{\Gin@base.bb}}%
        {\Gread@@xetex@aux#1}%
      }
    }
    \makeatother
    \usepackage[Export]{adjustbox} % Used to constrain images to a maximum size
    \adjustboxset{max size={0.9\linewidth}{0.9\paperheight}}

    % The hyperref package gives us a pdf with properly built
    % internal navigation ('pdf bookmarks' for the table of contents,
    % internal cross-reference links, web links for URLs, etc.)
    \usepackage{hyperref}
    % The default LaTeX title has an obnoxious amount of whitespace. By default,
    % titling removes some of it. It also provides customization options.
    \usepackage{titling}
    \usepackage{longtable} % longtable support required by pandoc >1.10
    \usepackage{booktabs}  % table support for pandoc > 1.12.2
    \usepackage{array}     % table support for pandoc >= 2.11.3
    \usepackage{calc}      % table minipage width calculation for pandoc >= 2.11.1
    \usepackage[inline]{enumitem} % IRkernel/repr support (it uses the enumerate* environment)
    \usepackage[normalem]{ulem} % ulem is needed to support strikethroughs (\sout)
                                % normalem makes italics be italics, not underlines
    \usepackage{mathrsfs}
    

    
    % Colors for the hyperref package
    \definecolor{urlcolor}{rgb}{0,.145,.698}
    \definecolor{linkcolor}{rgb}{.71,0.21,0.01}
    \definecolor{citecolor}{rgb}{.12,.54,.11}

    % ANSI colors
    \definecolor{ansi-black}{HTML}{3E424D}
    \definecolor{ansi-black-intense}{HTML}{282C36}
    \definecolor{ansi-red}{HTML}{E75C58}
    \definecolor{ansi-red-intense}{HTML}{B22B31}
    \definecolor{ansi-green}{HTML}{00A250}
    \definecolor{ansi-green-intense}{HTML}{007427}
    \definecolor{ansi-yellow}{HTML}{DDB62B}
    \definecolor{ansi-yellow-intense}{HTML}{B27D12}
    \definecolor{ansi-blue}{HTML}{208FFB}
    \definecolor{ansi-blue-intense}{HTML}{0065CA}
    \definecolor{ansi-magenta}{HTML}{D160C4}
    \definecolor{ansi-magenta-intense}{HTML}{A03196}
    \definecolor{ansi-cyan}{HTML}{60C6C8}
    \definecolor{ansi-cyan-intense}{HTML}{258F8F}
    \definecolor{ansi-white}{HTML}{C5C1B4}
    \definecolor{ansi-white-intense}{HTML}{A1A6B2}
    \definecolor{ansi-default-inverse-fg}{HTML}{FFFFFF}
    \definecolor{ansi-default-inverse-bg}{HTML}{000000}

    % common color for the border for error outputs.
    \definecolor{outerrorbackground}{HTML}{FFDFDF}

    % commands and environments needed by pandoc snippets
    % extracted from the output of `pandoc -s`
    \providecommand{\tightlist}{%
      \setlength{\itemsep}{0pt}\setlength{\parskip}{0pt}}
    \DefineVerbatimEnvironment{Highlighting}{Verbatim}{commandchars=\\\{\}}
    % Add ',fontsize=\small' for more characters per line
    \newenvironment{Shaded}{}{}
    \newcommand{\KeywordTok}[1]{\textcolor[rgb]{0.00,0.44,0.13}{\textbf{{#1}}}}
    \newcommand{\DataTypeTok}[1]{\textcolor[rgb]{0.56,0.13,0.00}{{#1}}}
    \newcommand{\DecValTok}[1]{\textcolor[rgb]{0.25,0.63,0.44}{{#1}}}
    \newcommand{\BaseNTok}[1]{\textcolor[rgb]{0.25,0.63,0.44}{{#1}}}
    \newcommand{\FloatTok}[1]{\textcolor[rgb]{0.25,0.63,0.44}{{#1}}}
    \newcommand{\CharTok}[1]{\textcolor[rgb]{0.25,0.44,0.63}{{#1}}}
    \newcommand{\StringTok}[1]{\textcolor[rgb]{0.25,0.44,0.63}{{#1}}}
    \newcommand{\CommentTok}[1]{\textcolor[rgb]{0.38,0.63,0.69}{\textit{{#1}}}}
    \newcommand{\OtherTok}[1]{\textcolor[rgb]{0.00,0.44,0.13}{{#1}}}
    \newcommand{\AlertTok}[1]{\textcolor[rgb]{1.00,0.00,0.00}{\textbf{{#1}}}}
    \newcommand{\FunctionTok}[1]{\textcolor[rgb]{0.02,0.16,0.49}{{#1}}}
    \newcommand{\RegionMarkerTok}[1]{{#1}}
    \newcommand{\ErrorTok}[1]{\textcolor[rgb]{1.00,0.00,0.00}{\textbf{{#1}}}}
    \newcommand{\NormalTok}[1]{{#1}}

    % Additional commands for more recent versions of Pandoc
    \newcommand{\ConstantTok}[1]{\textcolor[rgb]{0.53,0.00,0.00}{{#1}}}
    \newcommand{\SpecialCharTok}[1]{\textcolor[rgb]{0.25,0.44,0.63}{{#1}}}
    \newcommand{\VerbatimStringTok}[1]{\textcolor[rgb]{0.25,0.44,0.63}{{#1}}}
    \newcommand{\SpecialStringTok}[1]{\textcolor[rgb]{0.73,0.40,0.53}{{#1}}}
    \newcommand{\ImportTok}[1]{{#1}}
    \newcommand{\DocumentationTok}[1]{\textcolor[rgb]{0.73,0.13,0.13}{\textit{{#1}}}}
    \newcommand{\AnnotationTok}[1]{\textcolor[rgb]{0.38,0.63,0.69}{\textbf{\textit{{#1}}}}}
    \newcommand{\CommentVarTok}[1]{\textcolor[rgb]{0.38,0.63,0.69}{\textbf{\textit{{#1}}}}}
    \newcommand{\VariableTok}[1]{\textcolor[rgb]{0.10,0.09,0.49}{{#1}}}
    \newcommand{\ControlFlowTok}[1]{\textcolor[rgb]{0.00,0.44,0.13}{\textbf{{#1}}}}
    \newcommand{\OperatorTok}[1]{\textcolor[rgb]{0.40,0.40,0.40}{{#1}}}
    \newcommand{\BuiltInTok}[1]{{#1}}
    \newcommand{\ExtensionTok}[1]{{#1}}
    \newcommand{\PreprocessorTok}[1]{\textcolor[rgb]{0.74,0.48,0.00}{{#1}}}
    \newcommand{\AttributeTok}[1]{\textcolor[rgb]{0.49,0.56,0.16}{{#1}}}
    \newcommand{\InformationTok}[1]{\textcolor[rgb]{0.38,0.63,0.69}{\textbf{\textit{{#1}}}}}
    \newcommand{\WarningTok}[1]{\textcolor[rgb]{0.38,0.63,0.69}{\textbf{\textit{{#1}}}}}


    % Define a nice break command that doesn't care if a line doesn't already
    % exist.
    \def\br{\hspace*{\fill} \\* }
    % Math Jax compatibility definitions
    \def\gt{>}
    \def\lt{<}
    \let\Oldtex\TeX
    \let\Oldlatex\LaTeX
    \renewcommand{\TeX}{\textrm{\Oldtex}}
    \renewcommand{\LaTeX}{\textrm{\Oldlatex}}
    % Document parameters
    % Document title
    \title{CHM\_Lab1}
    
    
    
    
    
% Pygments definitions
\makeatletter
\def\PY@reset{\let\PY@it=\relax \let\PY@bf=\relax%
    \let\PY@ul=\relax \let\PY@tc=\relax%
    \let\PY@bc=\relax \let\PY@ff=\relax}
\def\PY@tok#1{\csname PY@tok@#1\endcsname}
\def\PY@toks#1+{\ifx\relax#1\empty\else%
    \PY@tok{#1}\expandafter\PY@toks\fi}
\def\PY@do#1{\PY@bc{\PY@tc{\PY@ul{%
    \PY@it{\PY@bf{\PY@ff{#1}}}}}}}
\def\PY#1#2{\PY@reset\PY@toks#1+\relax+\PY@do{#2}}

\@namedef{PY@tok@w}{\def\PY@tc##1{\textcolor[rgb]{0.73,0.73,0.73}{##1}}}
\@namedef{PY@tok@c}{\let\PY@it=\textit\def\PY@tc##1{\textcolor[rgb]{0.24,0.48,0.48}{##1}}}
\@namedef{PY@tok@cp}{\def\PY@tc##1{\textcolor[rgb]{0.61,0.40,0.00}{##1}}}
\@namedef{PY@tok@k}{\let\PY@bf=\textbf\def\PY@tc##1{\textcolor[rgb]{0.00,0.50,0.00}{##1}}}
\@namedef{PY@tok@kp}{\def\PY@tc##1{\textcolor[rgb]{0.00,0.50,0.00}{##1}}}
\@namedef{PY@tok@kt}{\def\PY@tc##1{\textcolor[rgb]{0.69,0.00,0.25}{##1}}}
\@namedef{PY@tok@o}{\def\PY@tc##1{\textcolor[rgb]{0.40,0.40,0.40}{##1}}}
\@namedef{PY@tok@ow}{\let\PY@bf=\textbf\def\PY@tc##1{\textcolor[rgb]{0.67,0.13,1.00}{##1}}}
\@namedef{PY@tok@nb}{\def\PY@tc##1{\textcolor[rgb]{0.00,0.50,0.00}{##1}}}
\@namedef{PY@tok@nf}{\def\PY@tc##1{\textcolor[rgb]{0.00,0.00,1.00}{##1}}}
\@namedef{PY@tok@nc}{\let\PY@bf=\textbf\def\PY@tc##1{\textcolor[rgb]{0.00,0.00,1.00}{##1}}}
\@namedef{PY@tok@nn}{\let\PY@bf=\textbf\def\PY@tc##1{\textcolor[rgb]{0.00,0.00,1.00}{##1}}}
\@namedef{PY@tok@ne}{\let\PY@bf=\textbf\def\PY@tc##1{\textcolor[rgb]{0.80,0.25,0.22}{##1}}}
\@namedef{PY@tok@nv}{\def\PY@tc##1{\textcolor[rgb]{0.10,0.09,0.49}{##1}}}
\@namedef{PY@tok@no}{\def\PY@tc##1{\textcolor[rgb]{0.53,0.00,0.00}{##1}}}
\@namedef{PY@tok@nl}{\def\PY@tc##1{\textcolor[rgb]{0.46,0.46,0.00}{##1}}}
\@namedef{PY@tok@ni}{\let\PY@bf=\textbf\def\PY@tc##1{\textcolor[rgb]{0.44,0.44,0.44}{##1}}}
\@namedef{PY@tok@na}{\def\PY@tc##1{\textcolor[rgb]{0.41,0.47,0.13}{##1}}}
\@namedef{PY@tok@nt}{\let\PY@bf=\textbf\def\PY@tc##1{\textcolor[rgb]{0.00,0.50,0.00}{##1}}}
\@namedef{PY@tok@nd}{\def\PY@tc##1{\textcolor[rgb]{0.67,0.13,1.00}{##1}}}
\@namedef{PY@tok@s}{\def\PY@tc##1{\textcolor[rgb]{0.73,0.13,0.13}{##1}}}
\@namedef{PY@tok@sd}{\let\PY@it=\textit\def\PY@tc##1{\textcolor[rgb]{0.73,0.13,0.13}{##1}}}
\@namedef{PY@tok@si}{\let\PY@bf=\textbf\def\PY@tc##1{\textcolor[rgb]{0.64,0.35,0.47}{##1}}}
\@namedef{PY@tok@se}{\let\PY@bf=\textbf\def\PY@tc##1{\textcolor[rgb]{0.67,0.36,0.12}{##1}}}
\@namedef{PY@tok@sr}{\def\PY@tc##1{\textcolor[rgb]{0.64,0.35,0.47}{##1}}}
\@namedef{PY@tok@ss}{\def\PY@tc##1{\textcolor[rgb]{0.10,0.09,0.49}{##1}}}
\@namedef{PY@tok@sx}{\def\PY@tc##1{\textcolor[rgb]{0.00,0.50,0.00}{##1}}}
\@namedef{PY@tok@m}{\def\PY@tc##1{\textcolor[rgb]{0.40,0.40,0.40}{##1}}}
\@namedef{PY@tok@gh}{\let\PY@bf=\textbf\def\PY@tc##1{\textcolor[rgb]{0.00,0.00,0.50}{##1}}}
\@namedef{PY@tok@gu}{\let\PY@bf=\textbf\def\PY@tc##1{\textcolor[rgb]{0.50,0.00,0.50}{##1}}}
\@namedef{PY@tok@gd}{\def\PY@tc##1{\textcolor[rgb]{0.63,0.00,0.00}{##1}}}
\@namedef{PY@tok@gi}{\def\PY@tc##1{\textcolor[rgb]{0.00,0.52,0.00}{##1}}}
\@namedef{PY@tok@gr}{\def\PY@tc##1{\textcolor[rgb]{0.89,0.00,0.00}{##1}}}
\@namedef{PY@tok@ge}{\let\PY@it=\textit}
\@namedef{PY@tok@gs}{\let\PY@bf=\textbf}
\@namedef{PY@tok@gp}{\let\PY@bf=\textbf\def\PY@tc##1{\textcolor[rgb]{0.00,0.00,0.50}{##1}}}
\@namedef{PY@tok@go}{\def\PY@tc##1{\textcolor[rgb]{0.44,0.44,0.44}{##1}}}
\@namedef{PY@tok@gt}{\def\PY@tc##1{\textcolor[rgb]{0.00,0.27,0.87}{##1}}}
\@namedef{PY@tok@err}{\def\PY@bc##1{{\setlength{\fboxsep}{\string -\fboxrule}\fcolorbox[rgb]{1.00,0.00,0.00}{1,1,1}{\strut ##1}}}}
\@namedef{PY@tok@kc}{\let\PY@bf=\textbf\def\PY@tc##1{\textcolor[rgb]{0.00,0.50,0.00}{##1}}}
\@namedef{PY@tok@kd}{\let\PY@bf=\textbf\def\PY@tc##1{\textcolor[rgb]{0.00,0.50,0.00}{##1}}}
\@namedef{PY@tok@kn}{\let\PY@bf=\textbf\def\PY@tc##1{\textcolor[rgb]{0.00,0.50,0.00}{##1}}}
\@namedef{PY@tok@kr}{\let\PY@bf=\textbf\def\PY@tc##1{\textcolor[rgb]{0.00,0.50,0.00}{##1}}}
\@namedef{PY@tok@bp}{\def\PY@tc##1{\textcolor[rgb]{0.00,0.50,0.00}{##1}}}
\@namedef{PY@tok@fm}{\def\PY@tc##1{\textcolor[rgb]{0.00,0.00,1.00}{##1}}}
\@namedef{PY@tok@vc}{\def\PY@tc##1{\textcolor[rgb]{0.10,0.09,0.49}{##1}}}
\@namedef{PY@tok@vg}{\def\PY@tc##1{\textcolor[rgb]{0.10,0.09,0.49}{##1}}}
\@namedef{PY@tok@vi}{\def\PY@tc##1{\textcolor[rgb]{0.10,0.09,0.49}{##1}}}
\@namedef{PY@tok@vm}{\def\PY@tc##1{\textcolor[rgb]{0.10,0.09,0.49}{##1}}}
\@namedef{PY@tok@sa}{\def\PY@tc##1{\textcolor[rgb]{0.73,0.13,0.13}{##1}}}
\@namedef{PY@tok@sb}{\def\PY@tc##1{\textcolor[rgb]{0.73,0.13,0.13}{##1}}}
\@namedef{PY@tok@sc}{\def\PY@tc##1{\textcolor[rgb]{0.73,0.13,0.13}{##1}}}
\@namedef{PY@tok@dl}{\def\PY@tc##1{\textcolor[rgb]{0.73,0.13,0.13}{##1}}}
\@namedef{PY@tok@s2}{\def\PY@tc##1{\textcolor[rgb]{0.73,0.13,0.13}{##1}}}
\@namedef{PY@tok@sh}{\def\PY@tc##1{\textcolor[rgb]{0.73,0.13,0.13}{##1}}}
\@namedef{PY@tok@s1}{\def\PY@tc##1{\textcolor[rgb]{0.73,0.13,0.13}{##1}}}
\@namedef{PY@tok@mb}{\def\PY@tc##1{\textcolor[rgb]{0.40,0.40,0.40}{##1}}}
\@namedef{PY@tok@mf}{\def\PY@tc##1{\textcolor[rgb]{0.40,0.40,0.40}{##1}}}
\@namedef{PY@tok@mh}{\def\PY@tc##1{\textcolor[rgb]{0.40,0.40,0.40}{##1}}}
\@namedef{PY@tok@mi}{\def\PY@tc##1{\textcolor[rgb]{0.40,0.40,0.40}{##1}}}
\@namedef{PY@tok@il}{\def\PY@tc##1{\textcolor[rgb]{0.40,0.40,0.40}{##1}}}
\@namedef{PY@tok@mo}{\def\PY@tc##1{\textcolor[rgb]{0.40,0.40,0.40}{##1}}}
\@namedef{PY@tok@ch}{\let\PY@it=\textit\def\PY@tc##1{\textcolor[rgb]{0.24,0.48,0.48}{##1}}}
\@namedef{PY@tok@cm}{\let\PY@it=\textit\def\PY@tc##1{\textcolor[rgb]{0.24,0.48,0.48}{##1}}}
\@namedef{PY@tok@cpf}{\let\PY@it=\textit\def\PY@tc##1{\textcolor[rgb]{0.24,0.48,0.48}{##1}}}
\@namedef{PY@tok@c1}{\let\PY@it=\textit\def\PY@tc##1{\textcolor[rgb]{0.24,0.48,0.48}{##1}}}
\@namedef{PY@tok@cs}{\let\PY@it=\textit\def\PY@tc##1{\textcolor[rgb]{0.24,0.48,0.48}{##1}}}

\def\PYZbs{\char`\\}
\def\PYZus{\char`\_}
\def\PYZob{\char`\{}
\def\PYZcb{\char`\}}
\def\PYZca{\char`\^}
\def\PYZam{\char`\&}
\def\PYZlt{\char`\<}
\def\PYZgt{\char`\>}
\def\PYZsh{\char`\#}
\def\PYZpc{\char`\%}
\def\PYZdl{\char`\$}
\def\PYZhy{\char`\-}
\def\PYZsq{\char`\'}
\def\PYZdq{\char`\"}
\def\PYZti{\char`\~}
% for compatibility with earlier versions
\def\PYZat{@}
\def\PYZlb{[}
\def\PYZrb{]}
\makeatother


    % For linebreaks inside Verbatim environment from package fancyvrb.
    \makeatletter
        \newbox\Wrappedcontinuationbox
        \newbox\Wrappedvisiblespacebox
        \newcommand*\Wrappedvisiblespace {\textcolor{red}{\textvisiblespace}}
        \newcommand*\Wrappedcontinuationsymbol {\textcolor{red}{\llap{\tiny$\m@th\hookrightarrow$}}}
        \newcommand*\Wrappedcontinuationindent {3ex }
        \newcommand*\Wrappedafterbreak {\kern\Wrappedcontinuationindent\copy\Wrappedcontinuationbox}
        % Take advantage of the already applied Pygments mark-up to insert
        % potential linebreaks for TeX processing.
        %        {, <, #, %, $, ' and ": go to next line.
        %        _, }, ^, &, >, - and ~: stay at end of broken line.
        % Use of \textquotesingle for straight quote.
        \newcommand*\Wrappedbreaksatspecials {%
            \def\PYGZus{\discretionary{\char`\_}{\Wrappedafterbreak}{\char`\_}}%
            \def\PYGZob{\discretionary{}{\Wrappedafterbreak\char`\{}{\char`\{}}%
            \def\PYGZcb{\discretionary{\char`\}}{\Wrappedafterbreak}{\char`\}}}%
            \def\PYGZca{\discretionary{\char`\^}{\Wrappedafterbreak}{\char`\^}}%
            \def\PYGZam{\discretionary{\char`\&}{\Wrappedafterbreak}{\char`\&}}%
            \def\PYGZlt{\discretionary{}{\Wrappedafterbreak\char`\<}{\char`\<}}%
            \def\PYGZgt{\discretionary{\char`\>}{\Wrappedafterbreak}{\char`\>}}%
            \def\PYGZsh{\discretionary{}{\Wrappedafterbreak\char`\#}{\char`\#}}%
            \def\PYGZpc{\discretionary{}{\Wrappedafterbreak\char`\%}{\char`\%}}%
            \def\PYGZdl{\discretionary{}{\Wrappedafterbreak\char`\$}{\char`\$}}%
            \def\PYGZhy{\discretionary{\char`\-}{\Wrappedafterbreak}{\char`\-}}%
            \def\PYGZsq{\discretionary{}{\Wrappedafterbreak\textquotesingle}{\textquotesingle}}%
            \def\PYGZdq{\discretionary{}{\Wrappedafterbreak\char`\"}{\char`\"}}%
            \def\PYGZti{\discretionary{\char`\~}{\Wrappedafterbreak}{\char`\~}}%
        }
        % Some characters . , ; ? ! / are not pygmentized.
        % This macro makes them "active" and they will insert potential linebreaks
        \newcommand*\Wrappedbreaksatpunct {%
            \lccode`\~`\.\lowercase{\def~}{\discretionary{\hbox{\char`\.}}{\Wrappedafterbreak}{\hbox{\char`\.}}}%
            \lccode`\~`\,\lowercase{\def~}{\discretionary{\hbox{\char`\,}}{\Wrappedafterbreak}{\hbox{\char`\,}}}%
            \lccode`\~`\;\lowercase{\def~}{\discretionary{\hbox{\char`\;}}{\Wrappedafterbreak}{\hbox{\char`\;}}}%
            \lccode`\~`\:\lowercase{\def~}{\discretionary{\hbox{\char`\:}}{\Wrappedafterbreak}{\hbox{\char`\:}}}%
            \lccode`\~`\?\lowercase{\def~}{\discretionary{\hbox{\char`\?}}{\Wrappedafterbreak}{\hbox{\char`\?}}}%
            \lccode`\~`\!\lowercase{\def~}{\discretionary{\hbox{\char`\!}}{\Wrappedafterbreak}{\hbox{\char`\!}}}%
            \lccode`\~`\/\lowercase{\def~}{\discretionary{\hbox{\char`\/}}{\Wrappedafterbreak}{\hbox{\char`\/}}}%
            \catcode`\.\active
            \catcode`\,\active
            \catcode`\;\active
            \catcode`\:\active
            \catcode`\?\active
            \catcode`\!\active
            \catcode`\/\active
            \lccode`\~`\~
        }
    \makeatother

    \let\OriginalVerbatim=\Verbatim
    \makeatletter
    \renewcommand{\Verbatim}[1][1]{%
        %\parskip\z@skip
        \sbox\Wrappedcontinuationbox {\Wrappedcontinuationsymbol}%
        \sbox\Wrappedvisiblespacebox {\FV@SetupFont\Wrappedvisiblespace}%
        \def\FancyVerbFormatLine ##1{\hsize\linewidth
            \vtop{\raggedright\hyphenpenalty\z@\exhyphenpenalty\z@
                \doublehyphendemerits\z@\finalhyphendemerits\z@
                \strut ##1\strut}%
        }%
        % If the linebreak is at a space, the latter will be displayed as visible
        % space at end of first line, and a continuation symbol starts next line.
        % Stretch/shrink are however usually zero for typewriter font.
        \def\FV@Space {%
            \nobreak\hskip\z@ plus\fontdimen3\font minus\fontdimen4\font
            \discretionary{\copy\Wrappedvisiblespacebox}{\Wrappedafterbreak}
            {\kern\fontdimen2\font}%
        }%

        % Allow breaks at special characters using \PYG... macros.
        \Wrappedbreaksatspecials
        % Breaks at punctuation characters . , ; ? ! and / need catcode=\active
        \OriginalVerbatim[#1,codes*=\Wrappedbreaksatpunct]%
    }
    \makeatother

    % Exact colors from NB
    \definecolor{incolor}{HTML}{303F9F}
    \definecolor{outcolor}{HTML}{D84315}
    \definecolor{cellborder}{HTML}{CFCFCF}
    \definecolor{cellbackground}{HTML}{F7F7F7}

    % prompt
    \makeatletter
    \newcommand{\boxspacing}{\kern\kvtcb@left@rule\kern\kvtcb@boxsep}
    \makeatother
    \newcommand{\prompt}[4]{
        {\ttfamily\llap{{\color{#2}[#3]:\hspace{3pt}#4}}\vspace{-\baselineskip}}
    }
    

    
    % Prevent overflowing lines due to hard-to-break entities
    \sloppy
    % Setup hyperref package
    \hypersetup{
      breaklinks=true,  % so long urls are correctly broken across lines
      colorlinks=true,
      urlcolor=urlcolor,
      linkcolor=linkcolor,
      citecolor=citecolor,
      }
    % Slightly bigger margins than the latex defaults
    
    \geometry{verbose,tmargin=1in,bmargin=1in,lmargin=1in,rmargin=1in}
    
\newtheorem{theorem}{Теорема}

\begin{document}
    
    \begin{titlepage}
    \newpage
    
    \begin{center}
    МИНИСТЕРСТВО ОБРАЗОВАНИЯ РЕСПУБЛИКИ БЕЛАРУСЬ БЕЛОРУССКИЙ ГОСУДАРСТВЕННЫЙ УНИВЕРСИТЕТ \\
    Факультет прикладной математики и инворматики \\ Кафедра вычислительной математики
 
    \end{center}
    
    \vspace{8em}
    
    \vspace{2em}
    
    \begin{center}
    \textsc{\textbf{Отчет по лабораторной работе 5 \\ "Методы решения ОДУ" \linebreak Вариант 5}}
    \end{center}
    
    \vspace{6em}
    
    \begin{flushright}
        Выполнил:\\
        Карпович Артём Дмитриевич\\
        студент 3 курса 7 группы
    \end{flushright}
    
    \begin{flushright}
        Преподаватель:\\
        Репников Василий Иванович
    \end{flushright}
    
    \vspace{\fill}
    
    \vspace{\fill}
    
    \begin{center}
    Минск, 2024
    \end{center}
    
    \end{titlepage}
    

    
    \section*{Методы решения
ОДУ}\label{ux43cux435ux442ux43eux434ux44b-ux440ux435ux448ux435ux43dux438ux44f-ux43eux434ux443}

    \subsection*{Постановка
задач}\label{ux43fux43eux441ux442ux430ux43dux43eux432ux43aux430-ux437ux430ux434ux430ux447}

    \begin{enumerate}
\def\labelenumi{\arabic{enumi}.}
\tightlist
\item
  Построить экстраполяционный метод Адамса пятого порядка (дать
  представление в двух формах).
\item
  Определить порядок точности метода \[\begin{cases}
  y_{j+1}=y_j+\frac{\tau}{6}(f_j+4f_{j+\frac{1}{2}}+f_{j+1}),\\
  y_j=y_{j+1}-\tau f_{j+\frac{1}{2}}, \\
  y_{j+\frac{1}{2}}=y_{j+1}-\frac{\tau}{2}f_{j+1}.
  \end{cases}\]
\item
  Найти интервал устойчивости метода из п. 1;
\item
  С заданной точность \(\varepsilon=10^{-4}\) найти решение задачи Коши
  с помощью: а) метода последовательного повышения порядка точности
  четвертого порядка; б) явного метода Рунге-Кутта третьего порядка; в)
  интерполяционого метода Адамса третьего порядка \[\begin{cases}
  u'=\frac{1}{5}\begin{pmatrix}-104 & -198 \\
  -198 & -401 \end{pmatrix} u, t \in [0; 5], \\
  u(0)=\begin{pmatrix}5 \\ 10 \end{pmatrix}.
  \end{cases}\]
\end{enumerate}

    \subsection*{Задача 1}\label{ux437ux430ux434ux430ux447ux430-1}

    Экстраполяционным методом Адамса называется метод, получаемый по формуле
\[y_{j+1} = y_j + h\sum_{i=0}^{k}A_i f(x_{j-i}, y_{j-i}),\] где
коэффициенты вычисляются через интегралы
\[A_i = \dfrac{(-1)^i}{i! (k-i)!}\int\limits_0^1 \dfrac{\alpha(\alpha+1)\ldots (\alpha+k)}{\alpha+i}d\alpha.\]
В случае метода пятого порядка, нам необходимо посчитать интегралы
\(A_i\) для всех \(i=\overline{0,k}.\) Поскольку нас интересует метод
пятого порядка, то выберем \(k=4\)
\[i=0: A_0 = \frac{1}{24}\int_0^1(\alpha+1)(\alpha+2)(\alpha+3)(\alpha+4)d\alpha=\frac{1901}{720} \approx 2.64;\]
\[i=1: A_1 = -\frac{1}{6}\int_0^1\alpha(\alpha+2)(\alpha+3)(\alpha+4)d\alpha=-\frac{1387}{360} \approx -3.85;\]
\[i=2: A_2 = \frac{1}{4}\int_0^1\alpha(\alpha+1)(\alpha+3)(\alpha+4)d\alpha=\frac{109}{30} \approx 3.63;\]
\[i=3: A_3 = -\frac{1}{6}\int_0^1\alpha(\alpha+1)(\alpha+2)(\alpha+4)d\alpha=-\frac{637}{360}\approx -1.77;\]
\[i=4: A_4 = \frac{1}{24}\int_0^1\alpha(\alpha+1)(\alpha+2)(\alpha+3)d\alpha=\frac{251}{720}\approx 0.35;\]
Проверим вычисленные значения с помощью инструментов Python

    \begin{tcolorbox}[breakable, size=fbox, boxrule=1pt, pad at break*=1mm,colback=cellbackground, colframe=cellborder]
\prompt{In}{incolor}{6}{\boxspacing}
\begin{Verbatim}[commandchars=\\\{\}]
\PY{k+kn}{from} \PY{n+nn}{scipy} \PY{k+kn}{import} \PY{n}{integrate}
\PY{k+kn}{import} \PY{n+nn}{math}

\PY{n}{k} \PY{o}{=} \PY{l+m+mi}{4}

\PY{k}{def} \PY{n+nf}{f}\PY{p}{(}\PY{n}{x}\PY{p}{,} \PY{n}{k}\PY{p}{,} \PY{n}{n}\PY{o}{=}\PY{l+m+mi}{4}\PY{p}{)}\PY{p}{:}
    \PY{n}{res} \PY{o}{=} \PY{l+m+mi}{1}
    
    \PY{k}{for} \PY{n}{i} \PY{o+ow}{in} \PY{n+nb}{range}\PY{p}{(}\PY{n}{n} \PY{o}{+} \PY{l+m+mi}{1}\PY{p}{)}\PY{p}{:}
       \PY{n}{res} \PY{o}{*}\PY{o}{=} \PY{p}{(}\PY{n}{x} \PY{o}{+} \PY{n}{i}\PY{p}{)}
    
    \PY{k}{return} \PY{n}{res} \PY{o}{/} \PY{p}{(}\PY{n}{x} \PY{o}{+} \PY{n}{k}\PY{p}{)}

\PY{k}{for} \PY{n}{i} \PY{o+ow}{in} \PY{n+nb}{range}\PY{p}{(}\PY{n}{k} \PY{o}{+} \PY{l+m+mi}{1}\PY{p}{)}\PY{p}{:}
    \PY{n+nb}{print}\PY{p}{(}\PY{l+s+sa}{f}\PY{l+s+s1}{\PYZsq{}}\PY{l+s+s1}{Интеграл }\PY{l+s+si}{\PYZob{}}\PY{n}{i}\PY{l+s+si}{\PYZcb{}}\PY{l+s+s1}{ равен: }\PY{l+s+si}{\PYZob{}}\PY{p}{(}\PY{p}{(}\PY{o}{\PYZhy{}}\PY{l+m+mi}{1}\PY{p}{)}\PY{o}{*}\PY{o}{*}\PY{n}{i}\PY{p}{)}\PY{+w}{ }\PY{o}{/}\PY{+w}{ }\PY{p}{(}\PY{n}{math}\PY{o}{.}\PY{n}{factorial}\PY{p}{(}\PY{n}{i}\PY{p}{)}\PY{+w}{ }\PY{o}{*}\PY{+w}{ }\PY{n}{math}\PY{o}{.}\PY{n}{factorial}\PY{p}{(}\PY{n}{k}\PY{+w}{ }\PY{o}{\PYZhy{}}\PY{+w}{ }\PY{n}{i}\PY{p}{)}\PY{p}{)}\PY{+w}{ }\PY{o}{*}\PY{+w}{ }\PY{n}{integrate}\PY{o}{.}\PY{n}{quad}\PY{p}{(}\PY{n}{f}\PY{p}{,}\PY{+w}{ }\PY{l+m+mi}{0}\PY{p}{,}\PY{+w}{ }\PY{l+m+mi}{1}\PY{p}{,}\PY{+w}{ }\PY{n}{args}\PY{o}{=}\PY{p}{(}\PY{n}{i}\PY{p}{,}\PY{p}{)}\PY{p}{)}\PY{p}{[}\PY{l+m+mi}{0}\PY{p}{]}\PY{l+s+si}{\PYZcb{}}\PY{l+s+s1}{\PYZsq{}}\PY{p}{)}
\end{Verbatim}
\end{tcolorbox}

    \begin{Verbatim}[commandchars=\\\{\}]
Интеграл 0 равен: 2.6402777777777775
Интеграл 1 равен: -3.852777777777778
Интеграл 2 равен: 3.6333333333333337
Интеграл 3 равен: -1.7694444444444442
Интеграл 4 равен: 0.3486111111111112
    \end{Verbatim}

    Таким образом, получаем
\(A_0 \approx 2.64, A_1 \approx -3.85, A_2 \approx 3.63, A_3 \approx -1.77, A_4 \approx 0.35\).
Тогда метод Адамса пятого порядка будет иметь вид

\[y_{j+1} = y_j + h(\frac{1901}{720}f_j -\frac{1387}{360} f_{j-1} + \frac{109}{30} f_{j-2}-\frac{637}{360} f_{j-3} +\frac{251}{720} f_{j-4}).\]

Получим представление метода Адамса пятого порядка через конечные
разности, формула для этого метода имеет вид
\[y_{j+1} = y_j + h\sum_{i=0}^{k}C_i \Delta ^i f_{j-i},\]
\[C_i = \dfrac{1}{i!}\int\limits_0^1 \alpha(\alpha+1)\ldots(\alpha+i-1)d\alpha.\]
Аналогичным образом, найдем все \(C_i\) для \(i=\overline{0,k},\) \(k\)
выбираем то же \[i=0: C_0 = 0;\]
\[i=1: C_1 = \int_0^1\alpha d\alpha = 1;\]
\[i=2: C_2 = \frac{1}{2!}\int_0^1\alpha(\alpha+1)d\alpha = \frac{1}{2}\int_0^1(\alpha^2+\alpha) d\alpha = \frac{5}{12};\]
\[i=3: C_3 = \frac{1}{3!}\int_0^1\alpha(\alpha+1)(\alpha+2)d\alpha=\frac{1}{6}\int_0^1 (\alpha^3+3\alpha^2+2\alpha) d\alpha = \frac{3}{8};\]
\[i=4: C_4 = \frac{1}{4!}\int_0^1\alpha(\alpha+1)(\alpha+2)(\alpha+3) d\alpha = \frac{1}{24}\int_0^1(\alpha^4+6\alpha^3+11\alpha^2+6\alpha) d\alpha = \frac{1}{24}\Bigl(\frac{1}{5}+\frac{3}{2}+\frac{11}{3}+3\Bigr) = \frac{251}{720};\]
Таким образом, можем составить экстраполяционый метод Адамса пятого
порядка через конечные разности
\[y_{j+1}=y_j+h\Bigl( \Delta^1 f_{j-1} + \frac{5}{12} \Delta^2 f_{j-2} + \frac{3}{8} \Delta^3 f_{j-3} + \frac{251}{720} \Delta^4 f_{j-4}\Bigr).\]

    \subsection*{Задача 2}\label{ux437ux430ux434ux430ux447ux430-2}

    Для определения порядка точности нашего метода первым делом оценим
локальные погрешности, определяемые следующим образом
\[r(t_j, \tau)=u(t_{j+1})-F\big(u(t_{j-q}),\ldots, u(t_j), u(t_{j+1}),\ldots,u(t_{j+s})\big).\eqno(2.1)\]
Для удобства переопределим \(u(t_j)=u_j.\) Тогда, переходя к нашему
методу, будем поочередно рассматривать уравнения от меньшего порядка к
большему. Приступим к рассмотрению третьего уравнения
\[y_{j+\frac{1}{2}}=y_{j+1}-\frac{\tau}{2}f_{j+1}\] Тогда, применяя
формулу (2.1), получим
\[r(t_j,\tau)=u_{j+\frac{1}{2}}-u_{j+1}+\frac{\tau}{2}f_{j+1}\] Для
оценки погрешности разложим функции по степеням \(\tau\)
\[u_{j+\frac{1}{2}}\approx u(t_j+\frac{\tau}{2}) = u_j + \frac{\tau}{2\cdot 1!}u'_j+\frac{\tau^2}{4 \cdot 2!}u''_j+O(\tau^3);\]
\[u_{j+1} \approx u(t_j+\tau)=u_j+\frac{\tau}{1!}u'_j+\frac{\tau^2}{2!}u''_j+O(\tau^3);\]
Для того, чтобы разложить \(f_{j+1}\) по степеням \(\tau\) воспользуемся
тем, что мы решаем задачу Коши вида
\[f(x,u(x)) = u'(x),\  u|_{x=x_0}=u_0.\] Тогда
\[f_{j+1} \approx f(t_j+\tau, u(t_j)) = u'(t_j+\tau)=u'_j+\frac{\tau}{1!}u''_j+\frac{\tau^2}{2!}u'''_j+O(\tau^3).\]
Подставим полученные разложения в рассматриваемое уравнение

    \[r(t_j,\tau)\approx u_j + \frac{\tau}{2}u'_j+\frac{\tau^2}{8}u''_j -u_j-\tau u'_j-\frac{\tau^2}{2}u''_j+\frac{\tau}{2}\Bigl( u'_j+\tau u''_j+\frac{\tau^2}{2}u'''_j \Bigr) + O(\tau^3)=\frac{\tau}{2}u'_j-\tau u'_j + \frac{\tau}{2}u'_j+\frac{\tau^2}{8}u''_j-\frac{\tau^2}{2}u''_j+\frac{\tau^2}{2}u''_j+\frac{\tau^3}{4}u'''_j+O(\tau^3)=\frac{\tau^2}{8}u''_j+\frac{\tau^3}{4}u'''_j+O(\tau^3)=O(\tau^2).\]
Таким образом, локальная погрешность равна \(r(t_j,\tau)=O(\tau^2).\)
Перейдем к следующему уравнению \[y_j=y_{j+1}-\tau f_{j+\frac{1}{2}}.\]
Проделаем ту же процедуру
\[r(t_j,\tau)=u_{j}-u_{j+1}+\tau f_{j+\frac{1}{2}}\] Приступим к
разложению по степеням \(\tau\). Поскольку мы уже раскладывали
\(u_{j+1},\) а разложение \(u_j\) приведет нас к \(u_j\), то сразу
перейдем к разложению третьего слагаемого
\[f_{j+\frac{1}{2}}\approx u'(t_j+\frac{\tau}{2}) = u'_j+\frac{\tau}{2}u''_j+\frac{\tau^2}{8}u'''_j+O(\tau^3).\]
Подставим

    \[r(t_j, \tau) \approx u_j - u_j - \tau u'_j - \frac{\tau^2}{2}u''_j+\tau \Bigl( u'_j+\frac{\tau}{2}u''_j+\frac{\tau^2}{8}u'''_j \Bigr) + O(\tau^3)=\frac{\tau^3}{8}u'''_j + O(\tau^3) = O(\tau^3).\]
Таким образом, локальная погрешность в этом случае равна
\(r(t_j, \tau) = O(\tau^3)\). Перейдем к последнему(первому) уравнению
\[r(t_j, \tau) = u_{j+1}-u_j-\frac{\tau}{6}(f_j+4f_{j+\frac{1}{2}}+f_{j+1}).\]
Аналогично предыдущему уравнению, сразу перейдем к подстановке, учитвая,
что \(f_j \approx u'_j\)
\[r(t_j, \tau) \approx u_j+\tau u'_j+\frac{\tau^2}{2}u''_j - u_j - \frac{\tau}{6}\Bigl(u'_j + 4 \Bigl(u'_j+\frac{\tau}{2}u''_j+\frac{\tau^2}{8}u'''_j\Bigr) + u'_j+\tau u''_j+\frac{\tau^2}{2}u'''_j\Bigr) + O(\tau^3)=O(\tau^3).\]
Таким образом, мы получили, что у нас сократились все слагаемые, поэтому
нам необходимо увеличить количество членов разложения, скажем, до
третьей степени, сразу выполним подстановку с учетом полученных
результатов
\[r(t_j, \tau) \approx \frac{\tau^3}{6}u'''_j-\frac{\tau}{6}\Bigl( \frac{4\tau^3}{12}u^{(4)}_j+\frac{\tau^3}{6}u^{(4)}_j\Bigr) + O(\tau^4) = O(\tau^3).\]
Таким образом, локальная погрешность равна \(r(t_j, \tau)=O(\tau^3),\)
найдем глобальную погрешность нашего метода
\[\psi(t_j, \tau) = \dfrac{O(\tau^3)}{\tau} = O(\tau^2).\] Отсюда
следует, что рассматриваемый метод является методом второго порядка
точности.

    \subsection*{Задача 3}\label{ux437ux430ux434ux430ux447ux430-3}

    Для нахождения интервала устойчивости для метода Адамса четвертого
порядка
\[y_{j+1} = y_j + h(\frac{1901}{720}f_j -\frac{1387}{360} f_{j-1} + \frac{109}{30} f_{j-2}-\frac{637}{360} f_{j-3} +\frac{251}{720} f_{j-4})\]
применим модельное уравнение вида
\[u'(x) = \lambda u(x),\quad \lambda \in \mathbb C,\ \operatorname{Re} \lambda < 0,\quad (1)\]
для которого известно, что задача Коши является устойчивой. Учтем
обозначения \[u(x_j)=y_j, \ u'(x_j)=f(x_j, u(x_j))=f_j.\] Тогда
модельное уравнения (1) можно переписать в виде \[f_j=\lambda y_j.\]
Тогда наш метод можно переписать в виде
\[y_{j+1} = y_j + h(\frac{1901}{720}\lambda y_j -\frac{1387}{360}\lambda y_{j-1} + \frac{109}{30}\lambda y_{j-2}-\frac{637}{360}\lambda y_{j-3} +\frac{251}{720}\lambda y_{j-4}).\]
Перенесём все слагаемые в левую сторону
\[y_{j+1} - y_j - h(\frac{1901}{720}\lambda y_j -\frac{1387}{360}\lambda y_{j-1} + \frac{109}{30}\lambda y_{j-2}-\frac{637}{360}\lambda y_{j-3} +\frac{251}{720}\lambda y_{j-4})=0.\]
Воспользуемся заменой \(z=\lambda h,\) тогда получим
\[y_{j+1} - y_j - \frac{1901}{720}z y_j +\frac{1387}{360}z y_{j-1} - \frac{109}{30}z y_{j-2}+\frac{637}{360}z y_{j-3} -\frac{251}{720}z y_{j-4}=0.\]
Вынесем общие множители за скобки
\[y_{j+1} - y_j(1 + \frac{1901}{720}z) +\frac{1387}{360}z y_{j-1} - \frac{109}{30}z y_{j-2}+\frac{637}{360}z y_{j-3} -\frac{251}{720}z y_{j-4}=0.\]
Запишем для этого уравнения характеристического уравнение
\[q^5 - q^4(1 + \frac{1901}{720}z) +\frac{1387}{360}z q^3 - \frac{109}{30}z q^2+\frac{637}{360}z q -\frac{251}{720}z =0.\]
Попробуем выразить отсюда \(z\).
\[q^5 - q^4 + z(- \frac{1901}{720}q^4 +\frac{1387}{360} q^3 - \frac{109}{30} q^2+\frac{637}{360} q -\frac{251}{720}) =0;\]

\[z(- \frac{1901}{720}q^4 +\frac{1387}{360} q^3 - \frac{109}{30} q^2+\frac{637}{360} q -\frac{251}{720}) =q^4-q^5;\]

\[z=\frac{q^4-q^5}{(- \frac{1901}{720}q^4 +\frac{1387}{360} q^3 - \frac{109}{30} q^2+\frac{637}{360} q -\frac{251}{720})}.\]

Введем обозначение \(q=e^{i\varphi}\) и тогда получим
\[z=z(\varphi)=\frac{e^{4i\varphi}-e^{5i\varphi}}{(- \frac{1901}{720}e^{4i\varphi} +\frac{1387}{360} e^{3i\varphi} - \frac{109}{30} e^{2i\varphi}+\frac{637}{360} e^{i\varphi} -\frac{251}{720})}, \varphi \in [0, 2\pi].\]
Тогда
\[z(0)=0, \ z(\pi)=\frac{\cos{4\pi}-\cos5\pi}{(- \frac{1901}{720}\cos{4\pi} +\frac{1387}{360} \cos{3\pi} - \frac{109}{30} \cos{2\pi}+\frac{637}{360} \cos{\pi} -\frac{251}{720})}=-\frac{91}{551}.\]
Таким образом, мы получили следующий интервал устойчивости
\[\Bigl[-\frac{91}{551}; 0\Bigr].\]

    \subsection*{Задача 4}\label{ux437ux430ux434ux430ux447ux430-4}

    Перепишем нашу задачу Коши \[\begin{cases}
u'=\frac{1}{5}\begin{pmatrix}-104 & -198 \\
-198 & -401 \end{pmatrix} u, t \in [0; 5], \\
u(0)=\begin{pmatrix}5 \\ 10 \end{pmatrix}.
\end{cases}\] Зададим данную дифференциальную модель программно

    \begin{tcolorbox}[breakable, size=fbox, boxrule=1pt, pad at break*=1mm,colback=cellbackground, colframe=cellborder]
\prompt{In}{incolor}{16}{\boxspacing}
\begin{Verbatim}[commandchars=\\\{\}]
\PY{k+kn}{import} \PY{n+nn}{numpy} \PY{k}{as} \PY{n+nn}{np}
\PY{k+kn}{import} \PY{n+nn}{matplotlib}\PY{n+nn}{.}\PY{n+nn}{pyplot} \PY{k}{as} \PY{n+nn}{plt}

\PY{k}{def} \PY{n+nf}{model}\PY{p}{(}\PY{n}{u}\PY{p}{,} \PY{n}{t}\PY{p}{)}\PY{p}{:}
    \PY{n}{u\PYZus{}deriv} \PY{o}{=} \PY{l+m+mi}{1} \PY{o}{/} \PY{l+m+mi}{5} \PY{o}{*} \PY{n}{np}\PY{o}{.}\PY{n}{dot}\PY{p}{(}\PY{n}{np}\PY{o}{.}\PY{n}{array}\PY{p}{(}\PY{p}{[}\PY{p}{[}\PY{o}{\PYZhy{}}\PY{l+m+mi}{104}\PY{p}{,} \PY{o}{\PYZhy{}}\PY{l+m+mi}{198}\PY{p}{]}\PY{p}{,} \PY{p}{[}\PY{o}{\PYZhy{}}\PY{l+m+mi}{198}\PY{p}{,} \PY{o}{\PYZhy{}}\PY{l+m+mi}{401}\PY{p}{]}\PY{p}{]}\PY{p}{)}\PY{p}{,} \PY{n}{u}\PY{p}{)}
    
    \PY{k}{return} \PY{n}{u\PYZus{}deriv}

\PY{n}{u\PYZus{}0} \PY{o}{=} \PY{p}{[}\PY{l+m+mi}{5}\PY{p}{,} \PY{l+m+mi}{10}\PY{p}{]}

\PY{n}{t\PYZus{}start} \PY{o}{=} \PY{l+m+mi}{0}
\PY{n}{t\PYZus{}end} \PY{o}{=} \PY{l+m+mi}{5}
\PY{n}{N} \PY{o}{=} \PY{l+m+mi}{1000}

\PY{n}{t} \PY{o}{=} \PY{n}{np}\PY{o}{.}\PY{n}{linspace}\PY{p}{(}\PY{n}{t\PYZus{}start}\PY{p}{,} \PY{n}{t\PYZus{}end}\PY{p}{,} \PY{n}{N}\PY{p}{)}
\end{Verbatim}
\end{tcolorbox}

    Аналитическое решение найдем с использованием инструментов
WolframMathematica \[u_1(t) = 5e^{-100 t},\ u_2(t) = 10e^{-100 t}.\]
Зададим и изобразим ее программно, чтобы в дальнейшем сравнивать
полученные решения.

    \begin{tcolorbox}[breakable, size=fbox, boxrule=1pt, pad at break*=1mm,colback=cellbackground, colframe=cellborder]
\prompt{In}{incolor}{18}{\boxspacing}
\begin{Verbatim}[commandchars=\\\{\}]
\PY{k}{def} \PY{n+nf}{solution}\PY{p}{(}\PY{n}{nodes}\PY{p}{)}\PY{p}{:}
    \PY{n}{t} \PY{o}{=} \PY{n}{np}\PY{o}{.}\PY{n}{array}\PY{p}{(}\PY{n}{nodes}\PY{p}{)}
    \PY{k}{return} \PY{n}{np}\PY{o}{.}\PY{n}{array}\PY{p}{(}\PY{p}{[}\PY{l+m+mi}{5} \PY{o}{*} \PY{n}{np}\PY{o}{.}\PY{n}{e}\PY{o}{*}\PY{o}{*}\PY{p}{(}\PY{o}{\PYZhy{}}\PY{l+m+mi}{100} \PY{o}{*} \PY{n}{t}\PY{p}{)}\PY{p}{,} \PY{l+m+mi}{10} \PY{o}{*} \PY{n}{np}\PY{o}{.}\PY{n}{e}\PY{o}{*}\PY{o}{*}\PY{p}{(}\PY{o}{\PYZhy{}}\PY{l+m+mi}{100} \PY{o}{*} \PY{n}{t}\PY{p}{)}\PY{p}{]}\PY{p}{)}\PY{o}{.}\PY{n}{T}

\PY{n}{plt}\PY{o}{.}\PY{n}{plot}\PY{p}{(}\PY{n}{t}\PY{p}{,} \PY{n}{solution}\PY{p}{(}\PY{n}{t}\PY{p}{)}\PY{p}{[}\PY{p}{:}\PY{p}{,} \PY{l+m+mi}{0}\PY{p}{]}\PY{p}{,} \PY{n}{label}\PY{o}{=}\PY{l+s+s1}{\PYZsq{}}\PY{l+s+s1}{u\PYZus{}1(t)}\PY{l+s+s1}{\PYZsq{}}\PY{p}{)}
\PY{n}{plt}\PY{o}{.}\PY{n}{plot}\PY{p}{(}\PY{n}{t}\PY{p}{,} \PY{n}{solution}\PY{p}{(}\PY{n}{t}\PY{p}{)}\PY{p}{[}\PY{p}{:}\PY{p}{,} \PY{l+m+mi}{1}\PY{p}{]}\PY{p}{,} \PY{n}{label}\PY{o}{=}\PY{l+s+s1}{\PYZsq{}}\PY{l+s+s1}{u\PYZus{}2(t)}\PY{l+s+s1}{\PYZsq{}}\PY{p}{)}
\PY{n}{plt}\PY{o}{.}\PY{n}{grid}\PY{p}{(}\PY{k+kc}{True}\PY{p}{)}
\PY{n}{plt}\PY{o}{.}\PY{n}{xlabel}\PY{p}{(}\PY{l+s+s1}{\PYZsq{}}\PY{l+s+s1}{t}\PY{l+s+s1}{\PYZsq{}}\PY{p}{)}
\PY{n}{plt}\PY{o}{.}\PY{n}{ylabel}\PY{p}{(}\PY{l+s+s1}{\PYZsq{}}\PY{l+s+s1}{u(t)}\PY{l+s+s1}{\PYZsq{}}\PY{p}{)}
\PY{n}{plt}\PY{o}{.}\PY{n}{legend}\PY{p}{(}\PY{p}{)}
\PY{n}{plt}\PY{o}{.}\PY{n}{show}\PY{p}{(}\PY{p}{)}
\end{Verbatim}
\end{tcolorbox}

    \begin{center}
    \adjustimage{max size={0.9\linewidth}{0.9\paperheight}}{output_17_0.png}
    \end{center}
    { \hspace*{\fill} \\}
    
    \subsubsection*{Метод последовательного повышения порядка точности
четвертого
порядка}\label{ux43cux435ux442ux43eux434-ux43fux43eux441ux43bux435ux434ux43eux432ux430ux442ux435ux43bux44cux43dux43eux433ux43e-ux43fux43eux432ux44bux448ux435ux43dux438ux44f-ux43fux43eux440ux44fux434ux43aux430-ux442ux43eux447ux43dux43eux441ux442ux438-ux447ux435ux442ux432ux435ux440ux442ux43eux433ux43e-ux43fux43eux440ux44fux434ux43aux430}

    \paragraph{Построение
метода}\label{ux43fux43eux441ux442ux440ux43eux435ux43dux438ux435-ux43cux435ux442ux43eux434ux430}

Первая проблема, с которой мы сталкиваемся, необходимость построения
такого метода. Поскольку нам нужен метод четвертого порядка, то выбираем
\(k=4\) и получаем систему уравнения для коэффициентов
\[\sum_{i=0}^q A_i = 1,\quad \sum_{i=0}^q A_i \alpha_i = \dfrac12,\quad \sum_{i=0}^q A_i\alpha_i^2 = \dfrac13,\quad \sum_{i=0}^q A_i\alpha_i^3=\frac{1}{4}.\]
Выберем \(q=2,\) чтобы получить систему из четырех уравнений с четырьмя
неизвестными \[\begin{cases}
A_0 + A_1 = 1,\\
A_0\alpha_0 + A_1\alpha_1 = \dfrac12,\\
A_0 \alpha_0^2 + A_1\alpha_1^2 = \dfrac13, \\
A_0 \alpha_0^3 + A_1\alpha_1^3 = \dfrac14.
\end{cases}\] Для решения данной системы домножим поочередно первое,
второе, третье уравнения на \(-\alpha_0\) и соответственно сложим
полученное со вторым, третьим и четвертым \[\begin{cases}
A_1\alpha_1 - A_1\alpha_0 = \dfrac12-\alpha_0,\\
A_1 \alpha_1^2 - A_1\alpha_0 \alpha_1 = \dfrac13-\frac{1}{2}\alpha_0, \\
A_1 \alpha_1^3 - A_1\alpha_0 \alpha_1^2 = \dfrac14-\frac{1}{3}\alpha_0.
\end{cases}\] Получили систему из трех уравнений, с которой проделаем
аналогичную операцию, только домножать будем на \(-\alpha_1\), тогда
получим следующую систему \[\begin{cases}
\alpha_0 \alpha_1 -\frac{1}{2}\alpha_0-\frac{1}{2}\alpha_1+\frac{1}{3}=0,\\
\frac{1}{2}\alpha_0 \alpha_1 -\frac{1}{3}\alpha_0-\frac{1}{3}\alpha_1+\frac{1}{4}=0.
\end{cases}\] Получили систему из двух уравнений с двумя неизвестными,
для решения которой домножим первое уравнение на \(-\frac{1}{2}\) и
прибавим его ко второму
\[(\alpha_0+\alpha_1)(\frac{1}{4}-\frac{1}{3})-\frac{1}{6}+\frac{1}{4}=0 \Rightarrow -(\alpha_0+\alpha_1)+\frac{1}{12}=0 \Rightarrow \alpha_0+\alpha_1=\frac{1}{12}.\]
Выразим отсюда \(\alpha_0=\frac{1}{12}-\alpha_1\) и подставим в первое
уравнение \[\alpha_1^2+\frac{\alpha_1}{12}-\frac{7}{24}=0.\] Получили
квадратное уравнение, корнями которого являются
\[\alpha_1 = \frac{1}{2},\quad \alpha_1=-\frac{7}{12}.\] Отсюда найдем
пару \(\alpha_0\)
\[\alpha_0 = -\frac{5}{12},\quad \alpha_0 = \frac{2}{3}.\] Выберем любую
пару \(\alpha_0, \alpha_1\)
\[\alpha_0 = -\frac{5}{12},\quad \alpha_1 = \frac{1}{2}.\] Вернемся к
нашей исходной системе и подставим эти значения в первые два уравнения
\[\begin{cases}
A_0+A_1 = 1,\\
-A_0\frac{5}{12}+\frac{1}{2}A_1=\frac{1}{2}.
\end{cases} \Rightarrow \begin{cases}
A_0=1-A_1,\\
-(1-A_1)\frac{5}{12}+\frac{1}{2}A_1=\frac{1}{2}.
\end{cases}\Rightarrow A_1=1,\quad A_0=0.\] Таким образом, имеем
\[A_0=0,\quad A_1=1,\quad \alpha_0 = -\frac{5}{12},\quad \alpha_1 = \frac{1}{2}.\]
Подставим данные коэффициенты в формулу метода
\[y_{j+1} = y_j + \tau \sum_{i=0}^{q}A_if(t_j + \alpha_i \tau, y(t_j + \alpha_i \tau)),\]
получим неявный метод четвертого порядка\\
\[\overset{[5]}y_{j+1}=\overset{[5]}y_j+h\overset{[5]}f_{j+\frac{1}{2}}.\]
Чтобы сделать метод явным, необходимо дополнить его методами более
низкого порядка, например методом ``3/4'' \[\begin{cases}
            \overset{[4]}{y}_{j+1} = \overset{[4]}{y}_j + \dfrac{h}{4}\left(\overset{[4]}{f}_j +3 \overset{[3]}{f}_{j+\frac23}\right),\\
        \overset{[3]}{y}_{j+\frac23} = \overset{[4]}{y}_j + \dfrac23h\overset{[2]}{f}_{j+\frac13},\\
        \overset{[2]}{y}_{j+\frac13} = \overset{[4]}{y}_j + \dfrac13h\overset{[4]}{f}_{j}.
        \end{cases}\] Для понижения порядка
\(\overset{[5]}f_{j+\frac{1}{2}}\) заменим в методе ``3/4''
\(h=\frac{1}{2}h\)

    \[\begin{cases}
\overset{[4]}{y}_{j+\frac{1}{2}} = \overset{[5]}{y}_j + \dfrac{h}{8}\left(\overset{[5]}{f}_j +3 \overset{[3]}{f}_{j+\frac13}\right),\\
\overset{[3]}{y}_{j+\frac13} = \overset{[5]}{y}_j + \dfrac13h\overset{[2]}{f}_{j+\frac16},\\
\overset{[2]}{y}_{j+\frac16} = \overset{[5]}{y}_j + \dfrac16h\overset{[4]}{f}_{j}.
\end{cases}\] Соберем построенные методы и получим явный метод
четвертого порядка \[\begin{cases}
\overset{[5]}y_{j+1}=\overset{[5]}y_j+h\overset{[4]}f_{j+\frac{1}{2}}, \\
\overset{[4]}{y}_{j+\frac{1}{2}} = \overset{[5]}{y}_j + \dfrac{h}{8}\left(\overset{[5]}{f}_j +3 \overset{[3]}{f}_{j+\frac13}\right),\\
\overset{[3]}{y}_{j+\frac13} = \overset{[5]}{y}_j + \dfrac13h\overset{[2]}{f}_{j+\frac16},\\
\overset{[2]}{y}_{j+\frac16} = \overset{[5]}{y}_j + \dfrac16h\overset{[5]}{f}_{j}.
\end{cases}\] Программную реализацию я не осилил.

    \subsubsection*{Явный метод Рунге-Кутта третьего
порядка}\label{ux44fux432ux43dux44bux439-ux43cux435ux442ux43eux434-ux440ux443ux43dux433ux435-ux43aux443ux442ux442ux430-ux442ux440ux435ux442ux44cux435ux433ux43e-ux43fux43eux440ux44fux434ux43aux430}

    Запишем более экономичный с точки зрения количства операций метод
Рунге-Кутта третьего порядка \[\begin{cases}
                y_{j+1} = y_j + \dfrac14 (\varphi_0 + 3\varphi_2),\\
                \varphi_0 = hf(x_j, y_j),\\
                \varphi_1 = hf\left(x_j + \dfrac13h, y_j + \dfrac13 \varphi_0\right),\\
                \varphi_2 = hf\left(x_j + \dfrac23h, y_j + \dfrac23\varphi_1\right).
            \end{cases}\]

    \begin{tcolorbox}[breakable, size=fbox, boxrule=1pt, pad at break*=1mm,colback=cellbackground, colframe=cellborder]
\prompt{In}{incolor}{24}{\boxspacing}
\begin{Verbatim}[commandchars=\\\{\}]
\PY{k}{def} \PY{n+nf}{runge\PYZus{}kutta}\PY{p}{(}\PY{n}{f}\PY{p}{,} \PY{n}{y0}\PY{p}{,} \PY{n}{t\PYZus{}start}\PY{p}{,} \PY{n}{t\PYZus{}end}\PY{p}{,} \PY{n}{epsilon}\PY{p}{)}\PY{p}{:}
    \PY{n}{y} \PY{o}{=} \PY{p}{[}\PY{n}{np}\PY{o}{.}\PY{n}{array}\PY{p}{(}\PY{n}{y0}\PY{p}{)}\PY{p}{]}
    \PY{n}{t} \PY{o}{=} \PY{n}{t\PYZus{}start}
    \PY{n}{nodes} \PY{o}{=} \PY{p}{[}\PY{n}{t\PYZus{}start}\PY{p}{]}
    \PY{n}{j} \PY{o}{=} \PY{l+m+mi}{0}
    \PY{n}{h1} \PY{o}{=} \PY{n}{epsilon}\PY{o}{*}\PY{o}{*}\PY{p}{(}\PY{l+m+mi}{1}\PY{o}{/}\PY{l+m+mi}{3}\PY{p}{)}
    \PY{n}{k} \PY{o}{=} \PY{l+m+mf}{1.005}
    
    \PY{k}{while} \PY{k+kc}{True}\PY{p}{:}
        
        \PY{k}{while} \PY{k+kc}{True}\PY{p}{:}
            \PY{n}{phi\PYZus{}0\PYZus{}h1} \PY{o}{=} \PY{n}{h1} \PY{o}{*} \PY{n}{f}\PY{p}{(}\PY{n}{y}\PY{p}{[}\PY{n}{j}\PY{p}{]}\PY{p}{,} \PY{n}{t}\PY{p}{)}
            \PY{n}{phi\PYZus{}1\PYZus{}h1} \PY{o}{=} \PY{n}{h1} \PY{o}{*} \PY{n}{f}\PY{p}{(}\PY{n}{y}\PY{p}{[}\PY{n}{j}\PY{p}{]} \PY{o}{+} \PY{l+m+mi}{1}\PY{o}{/}\PY{l+m+mi}{3} \PY{o}{*} \PY{n}{phi\PYZus{}0\PYZus{}h1}\PY{p}{,} \PY{n}{t} \PY{o}{+} \PY{l+m+mi}{1}\PY{o}{/}\PY{l+m+mi}{3} \PY{o}{*} \PY{n}{h1}\PY{p}{)}
            \PY{n}{phi\PYZus{}2\PYZus{}h1} \PY{o}{=} \PY{n}{h1} \PY{o}{*} \PY{n}{f}\PY{p}{(}\PY{n}{y}\PY{p}{[}\PY{n}{j}\PY{p}{]} \PY{o}{+} \PY{l+m+mi}{2}\PY{o}{/}\PY{l+m+mi}{3} \PY{o}{*} \PY{n}{phi\PYZus{}1\PYZus{}h1}\PY{p}{,} \PY{n}{t} \PY{o}{+} \PY{l+m+mi}{2}\PY{o}{/}\PY{l+m+mi}{3} \PY{o}{*} \PY{n}{h1}\PY{p}{)}
            
            \PY{n}{y\PYZus{}h1} \PY{o}{=} \PY{n}{y}\PY{p}{[}\PY{n}{j}\PY{p}{]} \PY{o}{+} \PY{l+m+mi}{1}\PY{o}{/}\PY{l+m+mi}{4} \PY{o}{*} \PY{p}{(}\PY{n}{phi\PYZus{}0\PYZus{}h1} \PY{o}{+} \PY{l+m+mi}{3} \PY{o}{*} \PY{n}{phi\PYZus{}2\PYZus{}h1}\PY{p}{)}
            
            \PY{n}{h2} \PY{o}{=} \PY{n}{h1} \PY{o}{/} \PY{l+m+mi}{2}
            \PY{n}{phi\PYZus{}0\PYZus{}h2} \PY{o}{=} \PY{n}{h2} \PY{o}{*} \PY{n}{f}\PY{p}{(}\PY{n}{y}\PY{p}{[}\PY{n}{j}\PY{p}{]}\PY{p}{,} \PY{n}{t}\PY{p}{)}
            \PY{n}{phi\PYZus{}1\PYZus{}h2} \PY{o}{=} \PY{n}{h2} \PY{o}{*} \PY{n}{f}\PY{p}{(}\PY{n}{y}\PY{p}{[}\PY{n}{j}\PY{p}{]} \PY{o}{+} \PY{l+m+mi}{1}\PY{o}{/}\PY{l+m+mi}{3} \PY{o}{*} \PY{n}{phi\PYZus{}0\PYZus{}h2}\PY{p}{,} \PY{n}{t} \PY{o}{+} \PY{l+m+mi}{1}\PY{o}{/}\PY{l+m+mi}{3} \PY{o}{*} \PY{n}{h2}\PY{p}{)}
            \PY{n}{phi\PYZus{}2\PYZus{}h2} \PY{o}{=} \PY{n}{h2} \PY{o}{*} \PY{n}{f}\PY{p}{(}\PY{n}{y}\PY{p}{[}\PY{n}{j}\PY{p}{]} \PY{o}{+} \PY{l+m+mi}{2}\PY{o}{/}\PY{l+m+mi}{3} \PY{o}{*} \PY{n}{phi\PYZus{}1\PYZus{}h2}\PY{p}{,} \PY{n}{t} \PY{o}{+} \PY{l+m+mi}{2}\PY{o}{/}\PY{l+m+mi}{3} \PY{o}{*} \PY{n}{h2}\PY{p}{)}
            
            \PY{n}{y\PYZus{}h2} \PY{o}{=} \PY{n}{y}\PY{p}{[}\PY{n}{j}\PY{p}{]} \PY{o}{+} \PY{l+m+mi}{1}\PY{o}{/}\PY{l+m+mi}{4} \PY{o}{*} \PY{p}{(}\PY{n}{phi\PYZus{}0\PYZus{}h2} \PY{o}{+} \PY{l+m+mi}{3} \PY{o}{*} \PY{n}{phi\PYZus{}2\PYZus{}h2}\PY{p}{)}
            
            \PY{k}{if} \PY{n}{np}\PY{o}{.}\PY{n}{max}\PY{p}{(}\PY{n}{y\PYZus{}h2} \PY{o}{\PYZhy{}} \PY{n}{y\PYZus{}h1}\PY{p}{)} \PY{o}{/} \PY{p}{(}\PY{l+m+mi}{1} \PY{o}{\PYZhy{}} \PY{p}{(}\PY{n}{h2} \PY{o}{/} \PY{n}{h1}\PY{p}{)}\PY{o}{*}\PY{o}{*}\PY{l+m+mi}{3}\PY{p}{)} \PY{o}{\PYZlt{}} \PY{n}{epsilon}\PY{p}{:}
                \PY{n}{y}\PY{o}{.}\PY{n}{append}\PY{p}{(}\PY{n}{y\PYZus{}h1}\PY{p}{)}
                \PY{k}{break} 
            \PY{k}{else}\PY{p}{:}
                \PY{n}{h1} \PY{o}{/}\PY{o}{=} \PY{l+m+mi}{2}
                
        \PY{n}{j} \PY{o}{+}\PY{o}{=} \PY{l+m+mi}{1}
        \PY{k}{if} \PY{n}{t} \PY{o}{+} \PY{n}{h1} \PY{o}{\PYZgt{}} \PY{n}{t\PYZus{}end}\PY{p}{:} 
            \PY{n}{nodes}\PY{o}{.}\PY{n}{append}\PY{p}{(}\PY{n}{t\PYZus{}end}\PY{p}{)} 
            \PY{k}{break}
        \PY{k}{else}\PY{p}{:}
            \PY{n}{t} \PY{o}{+}\PY{o}{=} \PY{n}{h1} 
            \PY{n}{nodes}\PY{o}{.}\PY{n}{append}\PY{p}{(}\PY{n}{t}\PY{p}{)}
            \PY{n}{h1} \PY{o}{*}\PY{o}{=} \PY{n}{k}
    
    \PY{k}{return} \PY{n}{np}\PY{o}{.}\PY{n}{stack}\PY{p}{(}\PY{n}{y}\PY{p}{,} \PY{n}{axis}\PY{o}{=}\PY{l+m+mi}{0}\PY{p}{)}\PY{p}{,} \PY{n}{nodes}
\end{Verbatim}
\end{tcolorbox}

    \begin{tcolorbox}[breakable, size=fbox, boxrule=1pt, pad at break*=1mm,colback=cellbackground, colframe=cellborder]
\prompt{In}{incolor}{25}{\boxspacing}
\begin{Verbatim}[commandchars=\\\{\}]
\PY{n}{sol}\PY{p}{,} \PY{n}{t\PYZus{}0} \PY{o}{=} \PY{n}{runge\PYZus{}kutta}\PY{p}{(}\PY{n}{model}\PY{p}{,} \PY{n}{u\PYZus{}0}\PY{p}{,} \PY{n}{t\PYZus{}start}\PY{p}{,} \PY{n}{t\PYZus{}end}\PY{p}{,} \PY{l+m+mf}{1e\PYZhy{}4}\PY{p}{)}
\end{Verbatim}
\end{tcolorbox}

    Построим полученное решение

    \begin{tcolorbox}[breakable, size=fbox, boxrule=1pt, pad at break*=1mm,colback=cellbackground, colframe=cellborder]
\prompt{In}{incolor}{27}{\boxspacing}
\begin{Verbatim}[commandchars=\\\{\}]
\PY{n}{plt}\PY{o}{.}\PY{n}{figure}\PY{p}{(}\PY{n}{figsize}\PY{o}{=}\PY{p}{(}\PY{l+m+mi}{10}\PY{p}{,} \PY{l+m+mi}{5}\PY{p}{)}\PY{p}{)}
\PY{n}{plt}\PY{o}{.}\PY{n}{plot}\PY{p}{(}\PY{n}{t\PYZus{}0}\PY{p}{,} \PY{n}{sol}\PY{p}{[}\PY{p}{:}\PY{p}{,} \PY{l+m+mi}{0}\PY{p}{]}\PY{p}{,} \PY{n}{label}\PY{o}{=}\PY{l+s+s1}{\PYZsq{}}\PY{l+s+s1}{u\PYZus{}1(t)}\PY{l+s+s1}{\PYZsq{}}\PY{p}{)}
\PY{n}{plt}\PY{o}{.}\PY{n}{plot}\PY{p}{(}\PY{n}{t\PYZus{}0}\PY{p}{,} \PY{n}{sol}\PY{p}{[}\PY{p}{:}\PY{p}{,} \PY{l+m+mi}{1}\PY{p}{]}\PY{p}{,} \PY{n}{label}\PY{o}{=}\PY{l+s+s1}{\PYZsq{}}\PY{l+s+s1}{u\PYZus{}2(t)}\PY{l+s+s1}{\PYZsq{}}\PY{p}{)}
\PY{n}{plt}\PY{o}{.}\PY{n}{grid}\PY{p}{(}\PY{k+kc}{True}\PY{p}{)}
\PY{n}{plt}\PY{o}{.}\PY{n}{xlabel}\PY{p}{(}\PY{l+s+s1}{\PYZsq{}}\PY{l+s+s1}{t}\PY{l+s+s1}{\PYZsq{}}\PY{p}{)}
\PY{n}{plt}\PY{o}{.}\PY{n}{ylabel}\PY{p}{(}\PY{l+s+s1}{\PYZsq{}}\PY{l+s+s1}{u(t)}\PY{l+s+s1}{\PYZsq{}}\PY{p}{)}
\PY{n}{plt}\PY{o}{.}\PY{n}{legend}\PY{p}{(}\PY{p}{)}
\PY{n}{plt}\PY{o}{.}\PY{n}{show}\PY{p}{(}\PY{p}{)}
\end{Verbatim}
\end{tcolorbox}

    \begin{center}
    \adjustimage{max size={0.9\linewidth}{0.9\paperheight}}{output_26_0.png}
    \end{center}
    { \hspace*{\fill} \\}
    
    Как видим, полученное решение визуально совпадает с аналитическим
решением.

    \subsubsection*{Интерполяционный метод Адамса третьего
порядка}\label{ux438ux43dux442ux435ux440ux43fux43eux43bux44fux446ux438ux43eux43dux43dux44bux439-ux43cux435ux442ux43eux434-ux430ux434ux430ux43cux441ux430-ux442ux440ux435ux442ux44cux435ux433ux43e-ux43fux43eux440ux44fux434ux43aux430}

    Интерполяционный метод Адамса третьего порядка имеет следующий вид
\[y_{j+1}=y_j+\frac{h}{12}(5f_{j+1}+8f_j-f_{j-1}).\] Поскольку для
реализации данного метода нам потребуются значения \(y_0, y_1\), то
обратимся к предыдущему пункту, методу Рунге-Кутта 3-его порядка.

    \begin{tcolorbox}[breakable, size=fbox, boxrule=1pt, pad at break*=1mm,colback=cellbackground, colframe=cellborder]
\prompt{In}{incolor}{30}{\boxspacing}
\begin{Verbatim}[commandchars=\\\{\}]
\PY{k}{def} \PY{n+nf}{adams}\PY{p}{(}\PY{n}{f}\PY{p}{,} \PY{n}{y0}\PY{p}{,} \PY{n}{t\PYZus{}0}\PY{p}{,} \PY{n}{t}\PY{p}{,} \PY{n}{epsilon}\PY{p}{)}\PY{p}{:}
    \PY{n}{y} \PY{o}{=} \PY{n}{y0}\PY{o}{.}\PY{n}{tolist}\PY{p}{(}\PY{p}{)}
    \PY{n}{t\PYZus{}new} \PY{o}{=} \PY{n}{t\PYZus{}0}
    \PY{n}{j} \PY{o}{=} \PY{l+m+mi}{2}
    \PY{n}{h\PYZus{}current} \PY{o}{=} \PY{n}{epsilon}\PY{o}{*}\PY{o}{*}\PY{p}{(}\PY{l+m+mi}{1}\PY{o}{/}\PY{l+m+mi}{2}\PY{p}{)}
    \PY{n}{k} \PY{o}{=} \PY{l+m+mf}{1.005}

    \PY{k}{while} \PY{k+kc}{True}\PY{p}{:}
        \PY{k}{while} \PY{k+kc}{True}\PY{p}{:}
            \PY{n}{y\PYZus{}high\PYZus{}k} \PY{o}{=} \PY{n}{y}\PY{p}{[}\PY{n}{j}\PY{p}{]}
            \PY{n}{y\PYZus{}high} \PY{o}{=} \PY{n}{y}\PY{p}{[}\PY{n}{j}\PY{p}{]}

            \PY{k}{while} \PY{k+kc}{True}\PY{p}{:}
                \PY{n}{y\PYZus{}high} \PY{o}{=} \PY{n}{y}\PY{p}{[}\PY{n}{j}\PY{p}{]} \PY{o}{+} \PY{n}{h\PYZus{}current} \PY{o}{/} \PY{l+m+mi}{12} \PY{o}{*} \PY{p}{(}\PY{l+m+mi}{5} \PY{o}{*} \PY{n}{f}\PY{p}{(}\PY{n}{y\PYZus{}high\PYZus{}k}\PY{p}{,} \PY{n}{t\PYZus{}new}\PY{p}{[}\PY{n}{j}\PY{p}{]}\PY{o}{+}\PY{n}{h\PYZus{}current}\PY{p}{)} \PY{o}{+} \PY{l+m+mi}{8} \PY{o}{*} \PY{n}{f}\PY{p}{(}\PY{n}{y}\PY{p}{[}\PY{n}{j}\PY{p}{]}\PY{p}{,} \PY{n}{t\PYZus{}new}\PY{p}{[}\PY{n}{j}\PY{p}{]}\PY{p}{)} \PY{o}{\PYZhy{}} \PY{n}{f}\PY{p}{(}\PY{n}{y}\PY{p}{[}\PY{n}{j}\PY{o}{\PYZhy{}}\PY{l+m+mi}{1}\PY{p}{]}\PY{p}{,} \PY{n}{t\PYZus{}new}\PY{p}{[}\PY{n}{j}\PY{o}{\PYZhy{}}\PY{l+m+mi}{1}\PY{p}{]}\PY{p}{)}\PY{p}{)}

                \PY{k}{if} \PY{n}{np}\PY{o}{.}\PY{n}{max}\PY{p}{(}\PY{n}{np}\PY{o}{.}\PY{n}{absolute}\PY{p}{(}\PY{n}{y\PYZus{}high} \PY{o}{\PYZhy{}} \PY{n}{y\PYZus{}high\PYZus{}k}\PY{p}{)}\PY{p}{)} \PY{o}{\PYZlt{}} \PY{n}{epsilon} \PY{o}{*} \PY{l+m+mf}{1e\PYZhy{}3}\PY{p}{:}
                    \PY{k}{break}

                \PY{n}{y\PYZus{}high\PYZus{}k} \PY{o}{=} \PY{n}{y\PYZus{}high}

            \PY{n}{h\PYZus{}half} \PY{o}{=} \PY{l+m+mf}{0.5} \PY{o}{*} \PY{n}{h\PYZus{}current}

            \PY{n}{y\PYZus{}low\PYZus{}k} \PY{o}{=} \PY{n}{y}\PY{p}{[}\PY{n}{j}\PY{p}{]}
            \PY{n}{y\PYZus{}low} \PY{o}{=} \PY{n}{y}\PY{p}{[}\PY{n}{j}\PY{p}{]}

            \PY{k}{while} \PY{k+kc}{True}\PY{p}{:}
                \PY{n}{y\PYZus{}low} \PY{o}{=} \PY{n}{y}\PY{p}{[}\PY{n}{j}\PY{p}{]} \PY{o}{+} \PY{n}{h\PYZus{}half} \PY{o}{/} \PY{l+m+mi}{12} \PY{o}{*} \PY{p}{(}\PY{l+m+mi}{5} \PY{o}{*} \PY{n}{f}\PY{p}{(}\PY{n}{y\PYZus{}low\PYZus{}k}\PY{p}{,} \PY{n}{t\PYZus{}new}\PY{p}{[}\PY{n}{j}\PY{p}{]}\PY{o}{+}\PY{n}{h\PYZus{}half}\PY{p}{)} \PY{o}{+} \PY{l+m+mi}{8} \PY{o}{*} \PY{n}{f}\PY{p}{(}\PY{n}{y}\PY{p}{[}\PY{n}{j}\PY{p}{]}\PY{p}{,} \PY{n}{t\PYZus{}new}\PY{p}{[}\PY{n}{j}\PY{p}{]}\PY{p}{)} \PY{o}{\PYZhy{}} \PY{n}{f}\PY{p}{(}\PY{n}{y}\PY{p}{[}\PY{n}{j}\PY{o}{\PYZhy{}}\PY{l+m+mi}{1}\PY{p}{]}\PY{p}{,} \PY{n}{t\PYZus{}new}\PY{p}{[}\PY{n}{j}\PY{o}{\PYZhy{}}\PY{l+m+mi}{1}\PY{p}{]}\PY{p}{)}\PY{p}{)}

                \PY{k}{if} \PY{n}{np}\PY{o}{.}\PY{n}{max}\PY{p}{(}\PY{n}{np}\PY{o}{.}\PY{n}{absolute}\PY{p}{(}\PY{n}{y\PYZus{}low} \PY{o}{\PYZhy{}} \PY{n}{y\PYZus{}low\PYZus{}k}\PY{p}{)}\PY{p}{)} \PY{o}{\PYZlt{}} \PY{n}{epsilon} \PY{o}{*} \PY{l+m+mf}{1e\PYZhy{}3}\PY{p}{:}
                    \PY{k}{break}

                \PY{n}{y\PYZus{}low\PYZus{}k} \PY{o}{=} \PY{n}{y\PYZus{}low}

            \PY{n}{error} \PY{o}{=} \PY{n}{np}\PY{o}{.}\PY{n}{max}\PY{p}{(}\PY{n}{y\PYZus{}low} \PY{o}{\PYZhy{}} \PY{n}{y\PYZus{}high}\PY{p}{)} \PY{o}{/} \PY{p}{(}\PY{l+m+mi}{1} \PY{o}{\PYZhy{}} \PY{p}{(}\PY{n}{h\PYZus{}half} \PY{o}{/} \PY{n}{h\PYZus{}current}\PY{p}{)}\PY{o}{*}\PY{o}{*}\PY{l+m+mi}{3}\PY{p}{)}

            \PY{k}{if} \PY{n}{error} \PY{o}{\PYZlt{}}\PY{o}{=} \PY{n}{epsilon}\PY{p}{:}
                \PY{n}{y}\PY{o}{.}\PY{n}{append}\PY{p}{(}\PY{n}{y\PYZus{}high}\PY{p}{)}
                \PY{k}{break}
            \PY{k}{else}\PY{p}{:}
                \PY{n}{h\PYZus{}current} \PY{o}{*}\PY{o}{=} \PY{l+m+mf}{0.5}

        \PY{n}{j} \PY{o}{+}\PY{o}{=} \PY{l+m+mi}{1}
        
        \PY{k}{if} \PY{n}{t\PYZus{}new}\PY{p}{[}\PY{n}{j}\PY{o}{\PYZhy{}}\PY{l+m+mi}{1}\PY{p}{]} \PY{o}{+} \PY{n}{h\PYZus{}current} \PY{o}{\PYZgt{}} \PY{n}{t}\PY{p}{[}\PY{o}{\PYZhy{}}\PY{l+m+mi}{1}\PY{p}{]}\PY{p}{:}
            \PY{n}{t\PYZus{}new}\PY{o}{.}\PY{n}{append}\PY{p}{(}\PY{n}{t}\PY{p}{[}\PY{o}{\PYZhy{}}\PY{l+m+mi}{1}\PY{p}{]}\PY{p}{)}
            \PY{k}{break}

        \PY{k}{else}\PY{p}{:}
            \PY{n}{t\PYZus{}current} \PY{o}{=} \PY{n}{t\PYZus{}new}\PY{p}{[}\PY{n}{j}\PY{o}{\PYZhy{}}\PY{l+m+mi}{1}\PY{p}{]} \PY{o}{+} \PY{n}{h\PYZus{}current}
            \PY{n}{t\PYZus{}new}\PY{o}{.}\PY{n}{append}\PY{p}{(}\PY{n}{t\PYZus{}current}\PY{p}{)}
            \PY{n}{h\PYZus{}current} \PY{o}{*}\PY{o}{=} \PY{n}{k}

    \PY{k}{return} \PY{n}{np}\PY{o}{.}\PY{n}{stack}\PY{p}{(}\PY{n}{y}\PY{p}{,} \PY{n}{axis}\PY{o}{=}\PY{l+m+mi}{0}\PY{p}{)}\PY{p}{,} \PY{n}{t\PYZus{}new}
\end{Verbatim}
\end{tcolorbox}

    \begin{tcolorbox}[breakable, size=fbox, boxrule=1pt, pad at break*=1mm,colback=cellbackground, colframe=cellborder]
\prompt{In}{incolor}{31}{\boxspacing}
\begin{Verbatim}[commandchars=\\\{\}]
\PY{n}{solution}\PY{p}{,} \PY{n}{nodes} \PY{o}{=} \PY{n}{adams}\PY{p}{(}\PY{n}{model}\PY{p}{,} \PY{n}{sol}\PY{p}{,} \PY{n}{t\PYZus{}0}\PY{p}{,} \PY{n}{t}\PY{p}{,} \PY{l+m+mf}{1e\PYZhy{}4}\PY{p}{)}
\end{Verbatim}
\end{tcolorbox}

    Построим полученное решение

    \begin{tcolorbox}[breakable, size=fbox, boxrule=1pt, pad at break*=1mm,colback=cellbackground, colframe=cellborder]
\prompt{In}{incolor}{67}{\boxspacing}
\begin{Verbatim}[commandchars=\\\{\}]
\PY{n}{plt}\PY{o}{.}\PY{n}{figure}\PY{p}{(}\PY{n}{figsize}\PY{o}{=}\PY{p}{(}\PY{l+m+mi}{10}\PY{p}{,} \PY{l+m+mi}{5}\PY{p}{)}\PY{p}{)}
\PY{n}{plt}\PY{o}{.}\PY{n}{plot}\PY{p}{(}\PY{n}{nodes}\PY{p}{,} \PY{n}{solution}\PY{p}{[}\PY{p}{:}\PY{p}{,} \PY{l+m+mi}{0}\PY{p}{]}\PY{p}{,} \PY{n}{label}\PY{o}{=}\PY{l+s+s1}{\PYZsq{}}\PY{l+s+s1}{u\PYZus{}1(t)}\PY{l+s+s1}{\PYZsq{}}\PY{p}{)}
\PY{n}{plt}\PY{o}{.}\PY{n}{plot}\PY{p}{(}\PY{n}{nodes}\PY{p}{,} \PY{n}{solution}\PY{p}{[}\PY{p}{:}\PY{p}{,} \PY{l+m+mi}{1}\PY{p}{]}\PY{p}{,} \PY{n}{label}\PY{o}{=}\PY{l+s+s1}{\PYZsq{}}\PY{l+s+s1}{u\PYZus{}2(t)}\PY{l+s+s1}{\PYZsq{}}\PY{p}{)}
\PY{n}{plt}\PY{o}{.}\PY{n}{grid}\PY{p}{(}\PY{k+kc}{True}\PY{p}{)}
\PY{n}{plt}\PY{o}{.}\PY{n}{xlabel}\PY{p}{(}\PY{l+s+s1}{\PYZsq{}}\PY{l+s+s1}{t}\PY{l+s+s1}{\PYZsq{}}\PY{p}{)}
\PY{n}{plt}\PY{o}{.}\PY{n}{ylabel}\PY{p}{(}\PY{l+s+s1}{\PYZsq{}}\PY{l+s+s1}{u(t)}\PY{l+s+s1}{\PYZsq{}}\PY{p}{)}
\PY{n}{plt}\PY{o}{.}\PY{n}{legend}\PY{p}{(}\PY{p}{)}
\PY{n}{plt}\PY{o}{.}\PY{n}{show}\PY{p}{(}\PY{p}{)}
\end{Verbatim}
\end{tcolorbox}

    \begin{center}
    \adjustimage{max size={0.9\linewidth}{0.9\paperheight}}{output_26_0.png}
    \end{center}
    
    Опять-таки, визуально график решения, полученного методом Адамса,
совпадает с графиком аналитического решения.
\end{document}
