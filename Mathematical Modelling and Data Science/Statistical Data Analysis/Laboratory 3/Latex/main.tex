\DeclareUnicodeCharacter{041B}{\CYRL}


\documentclass[11pt]{article}

    \usepackage[breakable]{tcolorbox}
    \usepackage{parskip} % Stop auto-indenting (to mimic markdown behaviour)
    \usepackage[english,russian]{babel}
    

    % Basic figure setup, for now with no caption control since it's done
    % automatically by Pandoc (which extracts ![](path) syntax from Markdown).
    \usepackage{graphicx}
    % Maintain compatibility with old templates. Remove in nbconvert 6.0
    \let\Oldincludegraphics\includegraphics
    % Ensure that by default, figures have no caption (until we provide a
    % proper Figure object with a Caption API and a way to capture that
    % in the conversion process - todo).
    \usepackage{caption}
    \DeclareCaptionFormat{nocaption}{}
    \captionsetup{format=nocaption,aboveskip=0pt,belowskip=0pt}

    \usepackage{float}
    \floatplacement{figure}{H} % forces figures to be placed at the correct location
    \usepackage{xcolor} % Allow colors to be defined
    \usepackage{enumerate} % Needed for markdown enumerations to work
    \usepackage{geometry} % Used to adjust the document margins
    \usepackage{amsmath} % Equations
    \usepackage{amssymb} % Equations
    \usepackage{textcomp} % defines textquotesingle
    % Hack from http://tex.stackexchange.com/a/47451/13684:
    \AtBeginDocument{%
        \def\PYZsq{\textquotesingle}% Upright quotes in Pygmentized code
    }
    \usepackage{upquote} % Upright quotes for verbatim code
    \usepackage{eurosym} % defines \euro

    \usepackage{iftex}
    \ifPDFTeX
        \usepackage[T1]{fontenc}
        \IfFileExists{alphabeta.sty}{
              \usepackage{alphabeta}
          }{
              \usepackage[mathletters]{ucs}
              \usepackage[utf8x]{inputenc}
          }
    \else
        \usepackage{fontspec}
        \usepackage{unicode-math}
    \fi

    \usepackage{fancyvrb} % verbatim replacement that allows latex
    \usepackage{grffile} % extends the file name processing of package graphics
                         % to support a larger range
    \makeatletter % fix for old versions of grffile with XeLaTeX
    \@ifpackagelater{grffile}{2019/11/01}
    {
      % Do nothing on new versions
    }
    {
      \def\Gread@@xetex#1{%
        \IfFileExists{"\Gin@base".bb}%
        {\Gread@eps{\Gin@base.bb}}%
        {\Gread@@xetex@aux#1}%
      }
    }
    \makeatother
    \usepackage[Export]{adjustbox} % Used to constrain images to a maximum size
    \adjustboxset{max size={0.9\linewidth}{0.9\paperheight}}

    % The hyperref package gives us a pdf with properly built
    % internal navigation ('pdf bookmarks' for the table of contents,
    % internal cross-reference links, web links for URLs, etc.)
    \usepackage{hyperref}
    % The default LaTeX title has an obnoxious amount of whitespace. By default,
    % titling removes some of it. It also provides customization options.
    \usepackage{titling}
    \usepackage{longtable} % longtable support required by pandoc >1.10
    \usepackage{booktabs}  % table support for pandoc > 1.12.2
    \usepackage{array}     % table support for pandoc >= 2.11.3
    \usepackage{calc}      % table minipage width calculation for pandoc >= 2.11.1
    \usepackage[inline]{enumitem} % IRkernel/repr support (it uses the enumerate* environment)
    \usepackage[normalem]{ulem} % ulem is needed to support strikethroughs (\sout)
                                % normalem makes italics be italics, not underlines
    \usepackage{mathrsfs}
    

    
    % Colors for the hyperref package
    \definecolor{urlcolor}{rgb}{0,.145,.698}
    \definecolor{linkcolor}{rgb}{.71,0.21,0.01}
    \definecolor{citecolor}{rgb}{.12,.54,.11}

    % ANSI colors
    \definecolor{ansi-black}{HTML}{3E424D}
    \definecolor{ansi-black-intense}{HTML}{282C36}
    \definecolor{ansi-red}{HTML}{E75C58}
    \definecolor{ansi-red-intense}{HTML}{B22B31}
    \definecolor{ansi-green}{HTML}{00A250}
    \definecolor{ansi-green-intense}{HTML}{007427}
    \definecolor{ansi-yellow}{HTML}{DDB62B}
    \definecolor{ansi-yellow-intense}{HTML}{B27D12}
    \definecolor{ansi-blue}{HTML}{208FFB}
    \definecolor{ansi-blue-intense}{HTML}{0065CA}
    \definecolor{ansi-magenta}{HTML}{D160C4}
    \definecolor{ansi-magenta-intense}{HTML}{A03196}
    \definecolor{ansi-cyan}{HTML}{60C6C8}
    \definecolor{ansi-cyan-intense}{HTML}{258F8F}
    \definecolor{ansi-white}{HTML}{C5C1B4}
    \definecolor{ansi-white-intense}{HTML}{A1A6B2}
    \definecolor{ansi-default-inverse-fg}{HTML}{FFFFFF}
    \definecolor{ansi-default-inverse-bg}{HTML}{000000}

    % common color for the border for error outputs.
    \definecolor{outerrorbackground}{HTML}{FFDFDF}

    % commands and environments needed by pandoc snippets
    % extracted from the output of `pandoc -s`
    \providecommand{\tightlist}{%
      \setlength{\itemsep}{0pt}\setlength{\parskip}{0pt}}
    \DefineVerbatimEnvironment{Highlighting}{Verbatim}{commandchars=\\\{\}}
    % Add ',fontsize=\small' for more characters per line
    \newenvironment{Shaded}{}{}
    \newcommand{\KeywordTok}[1]{\textcolor[rgb]{0.00,0.44,0.13}{\textbf{{#1}}}}
    \newcommand{\DataTypeTok}[1]{\textcolor[rgb]{0.56,0.13,0.00}{{#1}}}
    \newcommand{\DecValTok}[1]{\textcolor[rgb]{0.25,0.63,0.44}{{#1}}}
    \newcommand{\BaseNTok}[1]{\textcolor[rgb]{0.25,0.63,0.44}{{#1}}}
    \newcommand{\FloatTok}[1]{\textcolor[rgb]{0.25,0.63,0.44}{{#1}}}
    \newcommand{\CharTok}[1]{\textcolor[rgb]{0.25,0.44,0.63}{{#1}}}
    \newcommand{\StringTok}[1]{\textcolor[rgb]{0.25,0.44,0.63}{{#1}}}
    \newcommand{\CommentTok}[1]{\textcolor[rgb]{0.38,0.63,0.69}{\textit{{#1}}}}
    \newcommand{\OtherTok}[1]{\textcolor[rgb]{0.00,0.44,0.13}{{#1}}}
    \newcommand{\AlertTok}[1]{\textcolor[rgb]{1.00,0.00,0.00}{\textbf{{#1}}}}
    \newcommand{\FunctionTok}[1]{\textcolor[rgb]{0.02,0.16,0.49}{{#1}}}
    \newcommand{\RegionMarkerTok}[1]{{#1}}
    \newcommand{\ErrorTok}[1]{\textcolor[rgb]{1.00,0.00,0.00}{\textbf{{#1}}}}
    \newcommand{\NormalTok}[1]{{#1}}

    % Additional commands for more recent versions of Pandoc
    \newcommand{\ConstantTok}[1]{\textcolor[rgb]{0.53,0.00,0.00}{{#1}}}
    \newcommand{\SpecialCharTok}[1]{\textcolor[rgb]{0.25,0.44,0.63}{{#1}}}
    \newcommand{\VerbatimStringTok}[1]{\textcolor[rgb]{0.25,0.44,0.63}{{#1}}}
    \newcommand{\SpecialStringTok}[1]{\textcolor[rgb]{0.73,0.40,0.53}{{#1}}}
    \newcommand{\ImportTok}[1]{{#1}}
    \newcommand{\DocumentationTok}[1]{\textcolor[rgb]{0.73,0.13,0.13}{\textit{{#1}}}}
    \newcommand{\AnnotationTok}[1]{\textcolor[rgb]{0.38,0.63,0.69}{\textbf{\textit{{#1}}}}}
    \newcommand{\CommentVarTok}[1]{\textcolor[rgb]{0.38,0.63,0.69}{\textbf{\textit{{#1}}}}}
    \newcommand{\VariableTok}[1]{\textcolor[rgb]{0.10,0.09,0.49}{{#1}}}
    \newcommand{\ControlFlowTok}[1]{\textcolor[rgb]{0.00,0.44,0.13}{\textbf{{#1}}}}
    \newcommand{\OperatorTok}[1]{\textcolor[rgb]{0.40,0.40,0.40}{{#1}}}
    \newcommand{\BuiltInTok}[1]{{#1}}
    \newcommand{\ExtensionTok}[1]{{#1}}
    \newcommand{\PreprocessorTok}[1]{\textcolor[rgb]{0.74,0.48,0.00}{{#1}}}
    \newcommand{\AttributeTok}[1]{\textcolor[rgb]{0.49,0.56,0.16}{{#1}}}
    \newcommand{\InformationTok}[1]{\textcolor[rgb]{0.38,0.63,0.69}{\textbf{\textit{{#1}}}}}
    \newcommand{\WarningTok}[1]{\textcolor[rgb]{0.38,0.63,0.69}{\textbf{\textit{{#1}}}}}


    % Define a nice break command that doesn't care if a line doesn't already
    % exist.
    \def\br{\hspace*{\fill} \\* }
    % Math Jax compatibility definitions
    \def\gt{>}
    \def\lt{<}
    \let\Oldtex\TeX
    \let\Oldlatex\LaTeX
    \renewcommand{\TeX}{\textrm{\Oldtex}}
    \renewcommand{\LaTeX}{\textrm{\Oldlatex}}
    % Document parameters
    % Document title
    \title{Лабораторная работа 3}
    \author{Предварительный, корреляционный и регрессионный анализ неоднородных данных}
    
    
    
    
    
% Pygments definitions
\makeatletter
\def\PY@reset{\let\PY@it=\relax \let\PY@bf=\relax%
    \let\PY@ul=\relax \let\PY@tc=\relax%
    \let\PY@bc=\relax \let\PY@ff=\relax}
\def\PY@tok#1{\csname PY@tok@#1\endcsname}
\def\PY@toks#1+{\ifx\relax#1\empty\else%
    \PY@tok{#1}\expandafter\PY@toks\fi}
\def\PY@do#1{\PY@bc{\PY@tc{\PY@ul{%
    \PY@it{\PY@bf{\PY@ff{#1}}}}}}}
\def\PY#1#2{\PY@reset\PY@toks#1+\relax+\PY@do{#2}}

\@namedef{PY@tok@w}{\def\PY@tc##1{\textcolor[rgb]{0.73,0.73,0.73}{##1}}}
\@namedef{PY@tok@c}{\let\PY@it=\textit\def\PY@tc##1{\textcolor[rgb]{0.24,0.48,0.48}{##1}}}
\@namedef{PY@tok@cp}{\def\PY@tc##1{\textcolor[rgb]{0.61,0.40,0.00}{##1}}}
\@namedef{PY@tok@k}{\let\PY@bf=\textbf\def\PY@tc##1{\textcolor[rgb]{0.00,0.50,0.00}{##1}}}
\@namedef{PY@tok@kp}{\def\PY@tc##1{\textcolor[rgb]{0.00,0.50,0.00}{##1}}}
\@namedef{PY@tok@kt}{\def\PY@tc##1{\textcolor[rgb]{0.69,0.00,0.25}{##1}}}
\@namedef{PY@tok@o}{\def\PY@tc##1{\textcolor[rgb]{0.40,0.40,0.40}{##1}}}
\@namedef{PY@tok@ow}{\let\PY@bf=\textbf\def\PY@tc##1{\textcolor[rgb]{0.67,0.13,1.00}{##1}}}
\@namedef{PY@tok@nb}{\def\PY@tc##1{\textcolor[rgb]{0.00,0.50,0.00}{##1}}}
\@namedef{PY@tok@nf}{\def\PY@tc##1{\textcolor[rgb]{0.00,0.00,1.00}{##1}}}
\@namedef{PY@tok@nc}{\let\PY@bf=\textbf\def\PY@tc##1{\textcolor[rgb]{0.00,0.00,1.00}{##1}}}
\@namedef{PY@tok@nn}{\let\PY@bf=\textbf\def\PY@tc##1{\textcolor[rgb]{0.00,0.00,1.00}{##1}}}
\@namedef{PY@tok@ne}{\let\PY@bf=\textbf\def\PY@tc##1{\textcolor[rgb]{0.80,0.25,0.22}{##1}}}
\@namedef{PY@tok@nv}{\def\PY@tc##1{\textcolor[rgb]{0.10,0.09,0.49}{##1}}}
\@namedef{PY@tok@no}{\def\PY@tc##1{\textcolor[rgb]{0.53,0.00,0.00}{##1}}}
\@namedef{PY@tok@nl}{\def\PY@tc##1{\textcolor[rgb]{0.46,0.46,0.00}{##1}}}
\@namedef{PY@tok@ni}{\let\PY@bf=\textbf\def\PY@tc##1{\textcolor[rgb]{0.44,0.44,0.44}{##1}}}
\@namedef{PY@tok@na}{\def\PY@tc##1{\textcolor[rgb]{0.41,0.47,0.13}{##1}}}
\@namedef{PY@tok@nt}{\let\PY@bf=\textbf\def\PY@tc##1{\textcolor[rgb]{0.00,0.50,0.00}{##1}}}
\@namedef{PY@tok@nd}{\def\PY@tc##1{\textcolor[rgb]{0.67,0.13,1.00}{##1}}}
\@namedef{PY@tok@s}{\def\PY@tc##1{\textcolor[rgb]{0.73,0.13,0.13}{##1}}}
\@namedef{PY@tok@sd}{\let\PY@it=\textit\def\PY@tc##1{\textcolor[rgb]{0.73,0.13,0.13}{##1}}}
\@namedef{PY@tok@si}{\let\PY@bf=\textbf\def\PY@tc##1{\textcolor[rgb]{0.64,0.35,0.47}{##1}}}
\@namedef{PY@tok@se}{\let\PY@bf=\textbf\def\PY@tc##1{\textcolor[rgb]{0.67,0.36,0.12}{##1}}}
\@namedef{PY@tok@sr}{\def\PY@tc##1{\textcolor[rgb]{0.64,0.35,0.47}{##1}}}
\@namedef{PY@tok@ss}{\def\PY@tc##1{\textcolor[rgb]{0.10,0.09,0.49}{##1}}}
\@namedef{PY@tok@sx}{\def\PY@tc##1{\textcolor[rgb]{0.00,0.50,0.00}{##1}}}
\@namedef{PY@tok@m}{\def\PY@tc##1{\textcolor[rgb]{0.40,0.40,0.40}{##1}}}
\@namedef{PY@tok@gh}{\let\PY@bf=\textbf\def\PY@tc##1{\textcolor[rgb]{0.00,0.00,0.50}{##1}}}
\@namedef{PY@tok@gu}{\let\PY@bf=\textbf\def\PY@tc##1{\textcolor[rgb]{0.50,0.00,0.50}{##1}}}
\@namedef{PY@tok@gd}{\def\PY@tc##1{\textcolor[rgb]{0.63,0.00,0.00}{##1}}}
\@namedef{PY@tok@gi}{\def\PY@tc##1{\textcolor[rgb]{0.00,0.52,0.00}{##1}}}
\@namedef{PY@tok@gr}{\def\PY@tc##1{\textcolor[rgb]{0.89,0.00,0.00}{##1}}}
\@namedef{PY@tok@ge}{\let\PY@it=\textit}
\@namedef{PY@tok@gs}{\let\PY@bf=\textbf}
\@namedef{PY@tok@gp}{\let\PY@bf=\textbf\def\PY@tc##1{\textcolor[rgb]{0.00,0.00,0.50}{##1}}}
\@namedef{PY@tok@go}{\def\PY@tc##1{\textcolor[rgb]{0.44,0.44,0.44}{##1}}}
\@namedef{PY@tok@gt}{\def\PY@tc##1{\textcolor[rgb]{0.00,0.27,0.87}{##1}}}
\@namedef{PY@tok@err}{\def\PY@bc##1{{\setlength{\fboxsep}{\string -\fboxrule}\fcolorbox[rgb]{1.00,0.00,0.00}{1,1,1}{\strut ##1}}}}
\@namedef{PY@tok@kc}{\let\PY@bf=\textbf\def\PY@tc##1{\textcolor[rgb]{0.00,0.50,0.00}{##1}}}
\@namedef{PY@tok@kd}{\let\PY@bf=\textbf\def\PY@tc##1{\textcolor[rgb]{0.00,0.50,0.00}{##1}}}
\@namedef{PY@tok@kn}{\let\PY@bf=\textbf\def\PY@tc##1{\textcolor[rgb]{0.00,0.50,0.00}{##1}}}
\@namedef{PY@tok@kr}{\let\PY@bf=\textbf\def\PY@tc##1{\textcolor[rgb]{0.00,0.50,0.00}{##1}}}
\@namedef{PY@tok@bp}{\def\PY@tc##1{\textcolor[rgb]{0.00,0.50,0.00}{##1}}}
\@namedef{PY@tok@fm}{\def\PY@tc##1{\textcolor[rgb]{0.00,0.00,1.00}{##1}}}
\@namedef{PY@tok@vc}{\def\PY@tc##1{\textcolor[rgb]{0.10,0.09,0.49}{##1}}}
\@namedef{PY@tok@vg}{\def\PY@tc##1{\textcolor[rgb]{0.10,0.09,0.49}{##1}}}
\@namedef{PY@tok@vi}{\def\PY@tc##1{\textcolor[rgb]{0.10,0.09,0.49}{##1}}}
\@namedef{PY@tok@vm}{\def\PY@tc##1{\textcolor[rgb]{0.10,0.09,0.49}{##1}}}
\@namedef{PY@tok@sa}{\def\PY@tc##1{\textcolor[rgb]{0.73,0.13,0.13}{##1}}}
\@namedef{PY@tok@sb}{\def\PY@tc##1{\textcolor[rgb]{0.73,0.13,0.13}{##1}}}
\@namedef{PY@tok@sc}{\def\PY@tc##1{\textcolor[rgb]{0.73,0.13,0.13}{##1}}}
\@namedef{PY@tok@dl}{\def\PY@tc##1{\textcolor[rgb]{0.73,0.13,0.13}{##1}}}
\@namedef{PY@tok@s2}{\def\PY@tc##1{\textcolor[rgb]{0.73,0.13,0.13}{##1}}}
\@namedef{PY@tok@sh}{\def\PY@tc##1{\textcolor[rgb]{0.73,0.13,0.13}{##1}}}
\@namedef{PY@tok@s1}{\def\PY@tc##1{\textcolor[rgb]{0.73,0.13,0.13}{##1}}}
\@namedef{PY@tok@mb}{\def\PY@tc##1{\textcolor[rgb]{0.40,0.40,0.40}{##1}}}
\@namedef{PY@tok@mf}{\def\PY@tc##1{\textcolor[rgb]{0.40,0.40,0.40}{##1}}}
\@namedef{PY@tok@mh}{\def\PY@tc##1{\textcolor[rgb]{0.40,0.40,0.40}{##1}}}
\@namedef{PY@tok@mi}{\def\PY@tc##1{\textcolor[rgb]{0.40,0.40,0.40}{##1}}}
\@namedef{PY@tok@il}{\def\PY@tc##1{\textcolor[rgb]{0.40,0.40,0.40}{##1}}}
\@namedef{PY@tok@mo}{\def\PY@tc##1{\textcolor[rgb]{0.40,0.40,0.40}{##1}}}
\@namedef{PY@tok@ch}{\let\PY@it=\textit\def\PY@tc##1{\textcolor[rgb]{0.24,0.48,0.48}{##1}}}
\@namedef{PY@tok@cm}{\let\PY@it=\textit\def\PY@tc##1{\textcolor[rgb]{0.24,0.48,0.48}{##1}}}
\@namedef{PY@tok@cpf}{\let\PY@it=\textit\def\PY@tc##1{\textcolor[rgb]{0.24,0.48,0.48}{##1}}}
\@namedef{PY@tok@c1}{\let\PY@it=\textit\def\PY@tc##1{\textcolor[rgb]{0.24,0.48,0.48}{##1}}}
\@namedef{PY@tok@cs}{\let\PY@it=\textit\def\PY@tc##1{\textcolor[rgb]{0.24,0.48,0.48}{##1}}}

\def\PYZbs{\char`\\}
\def\PYZus{\char`\_}
\def\PYZob{\char`\{}
\def\PYZcb{\char`\}}
\def\PYZca{\char`\^}
\def\PYZam{\char`\&}
\def\PYZlt{\char`\<}
\def\PYZgt{\char`\>}
\def\PYZsh{\char`\#}
\def\PYZpc{\char`\%}
\def\PYZdl{\char`\$}
\def\PYZhy{\char`\-}
\def\PYZsq{\char`\'}
\def\PYZdq{\char`\"}
\def\PYZti{\char`\~}
% for compatibility with earlier versions
\def\PYZat{@}
\def\PYZlb{[}
\def\PYZrb{]}
\makeatother


    % For linebreaks inside Verbatim environment from package fancyvrb.
    \makeatletter
        \newbox\Wrappedcontinuationbox
        \newbox\Wrappedvisiblespacebox
        \newcommand*\Wrappedvisiblespace {\textcolor{red}{\textvisiblespace}}
        \newcommand*\Wrappedcontinuationsymbol {\textcolor{red}{\llap{\tiny$\m@th\hookrightarrow$}}}
        \newcommand*\Wrappedcontinuationindent {3ex }
        \newcommand*\Wrappedafterbreak {\kern\Wrappedcontinuationindent\copy\Wrappedcontinuationbox}
        % Take advantage of the already applied Pygments mark-up to insert
        % potential linebreaks for TeX processing.
        %        {, <, #, %, $, ' and ": go to next line.
        %        _, }, ^, &, >, - and ~: stay at end of broken line.
        % Use of \textquotesingle for straight quote.
        \newcommand*\Wrappedbreaksatspecials {%
            \def\PYGZus{\discretionary{\char`\_}{\Wrappedafterbreak}{\char`\_}}%
            \def\PYGZob{\discretionary{}{\Wrappedafterbreak\char`\{}{\char`\{}}%
            \def\PYGZcb{\discretionary{\char`\}}{\Wrappedafterbreak}{\char`\}}}%
            \def\PYGZca{\discretionary{\char`\^}{\Wrappedafterbreak}{\char`\^}}%
            \def\PYGZam{\discretionary{\char`\&}{\Wrappedafterbreak}{\char`\&}}%
            \def\PYGZlt{\discretionary{}{\Wrappedafterbreak\char`\<}{\char`\<}}%
            \def\PYGZgt{\discretionary{\char`\>}{\Wrappedafterbreak}{\char`\>}}%
            \def\PYGZsh{\discretionary{}{\Wrappedafterbreak\char`\#}{\char`\#}}%
            \def\PYGZpc{\discretionary{}{\Wrappedafterbreak\char`\%}{\char`\%}}%
            \def\PYGZdl{\discretionary{}{\Wrappedafterbreak\char`\$}{\char`\$}}%
            \def\PYGZhy{\discretionary{\char`\-}{\Wrappedafterbreak}{\char`\-}}%
            \def\PYGZsq{\discretionary{}{\Wrappedafterbreak\textquotesingle}{\textquotesingle}}%
            \def\PYGZdq{\discretionary{}{\Wrappedafterbreak\char`\"}{\char`\"}}%
            \def\PYGZti{\discretionary{\char`\~}{\Wrappedafterbreak}{\char`\~}}%
        }
        % Some characters . , ; ? ! / are not pygmentized.
        % This macro makes them "active" and they will insert potential linebreaks
        \newcommand*\Wrappedbreaksatpunct {%
            \lccode`\~`\.\lowercase{\def~}{\discretionary{\hbox{\char`\.}}{\Wrappedafterbreak}{\hbox{\char`\.}}}%
            \lccode`\~`\,\lowercase{\def~}{\discretionary{\hbox{\char`\,}}{\Wrappedafterbreak}{\hbox{\char`\,}}}%
            \lccode`\~`\;\lowercase{\def~}{\discretionary{\hbox{\char`\;}}{\Wrappedafterbreak}{\hbox{\char`\;}}}%
            \lccode`\~`\:\lowercase{\def~}{\discretionary{\hbox{\char`\:}}{\Wrappedafterbreak}{\hbox{\char`\:}}}%
            \lccode`\~`\?\lowercase{\def~}{\discretionary{\hbox{\char`\?}}{\Wrappedafterbreak}{\hbox{\char`\?}}}%
            \lccode`\~`\!\lowercase{\def~}{\discretionary{\hbox{\char`\!}}{\Wrappedafterbreak}{\hbox{\char`\!}}}%
            \lccode`\~`\/\lowercase{\def~}{\discretionary{\hbox{\char`\/}}{\Wrappedafterbreak}{\hbox{\char`\/}}}%
            \catcode`\.\active
            \catcode`\,\active
            \catcode`\;\active
            \catcode`\:\active
            \catcode`\?\active
            \catcode`\!\active
            \catcode`\/\active
            \lccode`\~`\~
        }
    \makeatother

    \let\OriginalVerbatim=\Verbatim
    \makeatletter
    \renewcommand{\Verbatim}[1][1]{%
        %\parskip\z@skip
        \sbox\Wrappedcontinuationbox {\Wrappedcontinuationsymbol}%
        \sbox\Wrappedvisiblespacebox {\FV@SetupFont\Wrappedvisiblespace}%
        \def\FancyVerbFormatLine ##1{\hsize\linewidth
            \vtop{\raggedright\hyphenpenalty\z@\exhyphenpenalty\z@
                \doublehyphendemerits\z@\finalhyphendemerits\z@
                \strut ##1\strut}%
        }%
        % If the linebreak is at a space, the latter will be displayed as visible
        % space at end of first line, and a continuation symbol starts next line.
        % Stretch/shrink are however usually zero for typewriter font.
        \def\FV@Space {%
            \nobreak\hskip\z@ plus\fontdimen3\font minus\fontdimen4\font
            \discretionary{\copy\Wrappedvisiblespacebox}{\Wrappedafterbreak}
            {\kern\fontdimen2\font}%
        }%

        % Allow breaks at special characters using \PYG... macros.
        \Wrappedbreaksatspecials
        % Breaks at punctuation characters . , ; ? ! and / need catcode=\active
        \OriginalVerbatim[#1,codes*=\Wrappedbreaksatpunct]%
    }
    \makeatother

    % Exact colors from NB
    \definecolor{incolor}{HTML}{303F9F}
    \definecolor{outcolor}{HTML}{D84315}
    \definecolor{cellborder}{HTML}{CFCFCF}
    \definecolor{cellbackground}{HTML}{F7F7F7}

    % prompt
    \makeatletter
    \newcommand{\boxspacing}{\kern\kvtcb@left@rule\kern\kvtcb@boxsep}
    \makeatother
    \newcommand{\prompt}[4]{
        {\ttfamily\llap{{\color{#2}[#3]:\hspace{3pt}#4}}\vspace{-\baselineskip}}
    }
    

    
    % Prevent overflowing lines due to hard-to-break entities
    \sloppy
    % Setup hyperref package
    \hypersetup{
      breaklinks=true,  % so long urls are correctly broken across lines
      colorlinks=true,
      urlcolor=urlcolor,
      linkcolor=linkcolor,
      citecolor=citecolor,
      }
    % Slightly bigger margins than the latex defaults
    
    \geometry{verbose,tmargin=1in,bmargin=1in,lmargin=1in,rmargin=1in}
    
    

\begin{document}
    
    \maketitle
    \begin{tcolorbox}[breakable, size=fbox, boxrule=1pt, pad at break*=1mm,colback=cellbackground, colframe=cellborder]
\prompt{In}{incolor}{1}{\boxspacing}
\begin{Verbatim}[commandchars=\\\{\}]
\PY{k+kn}{import} \PY{n+nn}{pandas} \PY{k}{as} \PY{n+nn}{pd}
\PY{k+kn}{import} \PY{n+nn}{scipy}\PY{n+nn}{.}\PY{n+nn}{stats}
\PY{k+kn}{import} \PY{n+nn}{statsmodels}\PY{n+nn}{.}\PY{n+nn}{api} \PY{k}{as} \PY{n+nn}{sm}
\PY{k+kn}{import} \PY{n+nn}{matplotlib}\PY{n+nn}{.}\PY{n+nn}{pyplot} \PY{k}{as} \PY{n+nn}{plt}
\PY{k+kn}{import} \PY{n+nn}{seaborn} \PY{k}{as} \PY{n+nn}{sns}
\PY{k+kn}{import} \PY{n+nn}{numpy} \PY{k}{as} \PY{n+nn}{np}
\PY{k+kn}{import} \PY{n+nn}{sklearn}

\PY{k+kn}{from} \PY{n+nn}{sklearn} \PY{k+kn}{import} \PY{n}{datasets}
\PY{k+kn}{from} \PY{n+nn}{sklearn}\PY{n+nn}{.}\PY{n+nn}{cluster} \PY{k+kn}{import} \PY{n}{KMeans}
\PY{k+kn}{from} \PY{n+nn}{statistics} \PY{k+kn}{import} \PY{n}{correlation}
\PY{k+kn}{from} \PY{n+nn}{scipy}\PY{n+nn}{.}\PY{n+nn}{stats} \PY{k+kn}{import} \PY{n}{norm}\PY{p}{,} \PY{n}{skew}\PY{p}{,} \PY{n}{kurtosis}\PY{p}{,} \PY{n}{shapiro}\PY{p}{,} \PY{n}{chisquare}\PY{p}{,} \PY{n}{gaussian\PYZus{}kde}\PY{p}{,} \PY{n}{kstest}\PY{p}{,} \PY{n}{ttest\PYZus{}ind}\PY{p}{,} \PY{n}{spearmanr}
\PY{k+kn}{import} \PY{n+nn}{statsmodels}\PY{n+nn}{.}\PY{n+nn}{api} \PY{k}{as} \PY{n+nn}{sm}
\end{Verbatim}
\end{tcolorbox}

    Подгрузили необходимый dataset, создали dataframe с необходимыми нам
видами ириса ``setosa'' и ``virginica''.

    \begin{tcolorbox}[breakable, size=fbox, boxrule=1pt, pad at break*=1mm,colback=cellbackground, colframe=cellborder]
\prompt{In}{incolor}{2}{\boxspacing}
\begin{Verbatim}[commandchars=\\\{\}]
\PY{n}{ds} \PY{o}{=} \PY{n}{datasets}\PY{o}{.}\PY{n}{load\PYZus{}iris}\PY{p}{(}\PY{p}{)}

\PY{n}{ext\PYZus{}target} \PY{o}{=} \PY{n}{ds}\PY{o}{.}\PY{n}{target}\PY{p}{[}\PY{p}{:}\PY{p}{,} \PY{k+kc}{None}\PY{p}{]}
\PY{n}{df} \PY{o}{=} \PY{n}{pd}\PY{o}{.}\PY{n}{DataFrame}\PY{p}{(}
    \PY{n}{np}\PY{o}{.}\PY{n}{concatenate}\PY{p}{(}\PY{p}{(}\PY{n}{ds}\PY{o}{.}\PY{n}{data}\PY{p}{,} \PY{n}{ds}\PY{o}{.}\PY{n}{target\PYZus{}names}\PY{p}{[}\PY{n}{ext\PYZus{}target}\PY{p}{]}\PY{p}{)}\PY{p}{,} \PY{n}{axis}\PY{o}{=}\PY{l+m+mi}{1}\PY{p}{)}\PY{p}{,}
    \PY{n}{columns}\PY{o}{=}\PY{n}{ds}\PY{o}{.}\PY{n}{feature\PYZus{}names} \PY{o}{+} \PY{p}{[}\PY{l+s+s1}{\PYZsq{}}\PY{l+s+s1}{target\PYZus{}name}\PY{l+s+s1}{\PYZsq{}}\PY{p}{]}\PY{p}{)}

\PY{n}{df} \PY{o}{=} \PY{n}{df}\PY{o}{.}\PY{n}{loc}\PY{p}{[}\PY{n}{df}\PY{p}{[}\PY{l+s+s1}{\PYZsq{}}\PY{l+s+s1}{target\PYZus{}name}\PY{l+s+s1}{\PYZsq{}}\PY{p}{]} \PY{o}{!=} \PY{l+s+s1}{\PYZsq{}}\PY{l+s+s1}{versicolor}\PY{l+s+s1}{\PYZsq{}}\PY{p}{]}\PY{o}{.}\PY{n}{reset\PYZus{}index}\PY{p}{(}\PY{n}{drop}\PY{o}{=}\PY{k+kc}{True}\PY{p}{)}

\PY{n}{df}
\end{Verbatim}
\end{tcolorbox}

            \begin{tcolorbox}[breakable, size=fbox, boxrule=.5pt, pad at break*=1mm, opacityfill=0]
\prompt{Out}{outcolor}{2}{\boxspacing}
\begin{Verbatim}[commandchars=\\\{\}]
   sepal length (cm) sepal width (cm) petal length (cm) petal width (cm)  \textbackslash{}
0                5.1              3.5               1.4              0.2
1                4.9              3.0               1.4              0.2
2                4.7              3.2               1.3              0.2
3                4.6              3.1               1.5              0.2
4                5.0              3.6               1.4              0.2
..               {\ldots}              {\ldots}               {\ldots}              {\ldots}
95               6.7              3.0               5.2              2.3
96               6.3              2.5               5.0              1.9
97               6.5              3.0               5.2              2.0
98               6.2              3.4               5.4              2.3
99               5.9              3.0               5.1              1.8

   target\_name
0       setosa
1       setosa
2       setosa
3       setosa
4       setosa
..         {\ldots}
95   virginica
96   virginica
97   virginica
98   virginica
99   virginica

[100 rows x 5 columns]
\end{Verbatim}
\end{tcolorbox}
        
\section*{Предварительный анализ}

Проведем предварительный анализа для переменной SEPALWID

    \begin{tcolorbox}[breakable, size=fbox, boxrule=1pt, pad at break*=1mm,colback=cellbackground, colframe=cellborder]
\prompt{In}{incolor}{3}{\boxspacing}
\begin{Verbatim}[commandchars=\\\{\}]
\PY{n}{sepalwid\PYZus{}df} \PY{o}{=} \PY{n}{df}\PY{p}{[}\PY{p}{[}\PY{l+s+s1}{\PYZsq{}}\PY{l+s+s1}{sepal width (cm)}\PY{l+s+s1}{\PYZsq{}}\PY{p}{,} \PY{l+s+s1}{\PYZsq{}}\PY{l+s+s1}{target\PYZus{}name}\PY{l+s+s1}{\PYZsq{}}\PY{p}{]}\PY{p}{]}
\PY{n}{sepalwid\PYZus{}df}
\end{Verbatim}
\end{tcolorbox}

            \begin{tcolorbox}[breakable, size=fbox, boxrule=.5pt, pad at break*=1mm, opacityfill=0]
\prompt{Out}{outcolor}{3}{\boxspacing}
\begin{Verbatim}[commandchars=\\\{\}]
   sepal width (cm) target\_name
0               3.5      setosa
1               3.0      setosa
2               3.2      setosa
3               3.1      setosa
4               3.6      setosa
..              {\ldots}         {\ldots}
95              3.0   virginica
96              2.5   virginica
97              3.0   virginica
98              3.4   virginica
99              3.0   virginica

[100 rows x 2 columns]
\end{Verbatim}
\end{tcolorbox}
        
    \begin{tcolorbox}[breakable, size=fbox, boxrule=1pt, pad at break*=1mm,colback=cellbackground, colframe=cellborder]
\prompt{In}{incolor}{4}{\boxspacing}
\begin{Verbatim}[commandchars=\\\{\}]
\PY{n}{sepalwid} \PY{o}{=} \PY{n}{sepalwid\PYZus{}df}\PY{p}{[}\PY{l+s+s1}{\PYZsq{}}\PY{l+s+s1}{sepal width (cm)}\PY{l+s+s1}{\PYZsq{}}\PY{p}{]}\PY{o}{.}\PY{n}{astype}\PY{p}{(}\PY{n+nb}{float}\PY{p}{)}

\PY{n}{sepalwid}
\end{Verbatim}
\end{tcolorbox}

            \begin{tcolorbox}[breakable, size=fbox, boxrule=.5pt, pad at break*=1mm, opacityfill=0]
\prompt{Out}{outcolor}{4}{\boxspacing}
\begin{Verbatim}[commandchars=\\\{\}]
0     3.5
1     3.0
2     3.2
3     3.1
4     3.6
     {\ldots}
95    3.0
96    2.5
97    3.0
98    3.4
99    3.0
Name: sepal width (cm), Length: 100, dtype: float64
\end{Verbatim}
\end{tcolorbox}
        
    Выдвинем гипотезу о нормальности распределения нашей переменной. Для
проверки этой гипотезы первым делом построим гистограмму распределения.

    \begin{tcolorbox}[breakable, size=fbox, boxrule=1pt, pad at break*=1mm,colback=cellbackground, colframe=cellborder]
\prompt{In}{incolor}{5}{\boxspacing}
\begin{Verbatim}[commandchars=\\\{\}]
\PY{n}{sns}\PY{o}{.}\PY{n}{displot}\PY{p}{(}\PY{n}{sepalwid}\PY{p}{,} \PY{n}{kde}\PY{o}{=}\PY{k+kc}{True}\PY{p}{)}

\PY{n}{plt}\PY{o}{.}\PY{n}{show}\PY{p}{(}\PY{p}{)}
\end{Verbatim}
\end{tcolorbox}

    \begin{center}
    \adjustimage{max size={0.9\linewidth}{0.9\paperheight}}{output_8_0.png}
    \end{center}
    { \hspace*{\fill} \\}
    
    На гистограмме видны горбы, однако по ядерной оценке видно, что наша
гистограмма весомо не отличается от гистограммы нормального
распределения, что позволяет нам не отклонять гипотезу о нормальности
распределения.

Следующим этапом станет проверка критериев 
\begin{itemize}
    \item Колмогорова-Смирнова
    \item Шапиро-Уилка
    \item $x^2$-Пирсона
\end{itemize}

    \begin{tcolorbox}[breakable, size=fbox, boxrule=1pt, pad at break*=1mm,colback=cellbackground, colframe=cellborder]
\prompt{In}{incolor}{6}{\boxspacing}
\begin{Verbatim}[commandchars=\\\{\}]
\PY{n+nb}{print}\PY{p}{(}\PY{n}{kstest}\PY{p}{(}\PY{n}{sepalwid}\PY{p}{,} \PY{l+s+s1}{\PYZsq{}}\PY{l+s+s1}{norm}\PY{l+s+s1}{\PYZsq{}}\PY{p}{)}\PY{p}{)}
\PY{n+nb}{print}\PY{p}{(}\PY{n}{scipy}\PY{o}{.}\PY{n}{stats}\PY{o}{.}\PY{n}{shapiro}\PY{p}{(}\PY{n}{sepalwid}\PY{p}{)}\PY{p}{)}
\PY{n+nb}{print}\PY{p}{(}\PY{n}{scipy}\PY{o}{.}\PY{n}{stats}\PY{o}{.}\PY{n}{chisquare}\PY{p}{(}\PY{n}{sepalwid}\PY{p}{)}\PY{p}{)}
\end{Verbatim}
\end{tcolorbox}

    \begin{Verbatim}[commandchars=\\\{\}]
KstestResult(statistic=0.9860965524865014, pvalue=4.104688155544011e-186,
statistic\_location=2.2, statistic\_sign=-1)
ShapiroResult(statistic=0.9852025508880615, pvalue=0.3284207880496979)
Power\_divergenceResult(statistic=5.401405810684162, pvalue=1.0)
    \end{Verbatim}

    Из критериев видно, что единственным критерием, который отклоняет
гипотезу о нормальности распределения, является критерий
Колмогорова-Смирнова(p-value = 4.104688155544011e-186 \textless{} 0.05).
Однако оставшиеся критерии позволяют не отклонять гипотезу о
нормальности.

Проведем анализ дискриптивных статистик.

    \begin{tcolorbox}[breakable, size=fbox, boxrule=1pt, pad at break*=1mm,colback=cellbackground, colframe=cellborder]
\prompt{In}{incolor}{7}{\boxspacing}
\begin{Verbatim}[commandchars=\\\{\}]
\PY{n}{describe} \PY{o}{=} \PY{n}{pd}\PY{o}{.}\PY{n}{DataFrame}\PY{p}{(}\PY{n}{pd}\PY{o}{.}\PY{n}{concat}\PY{p}{(}\PY{p}{[}\PY{n}{sepalwid}\PY{o}{.}\PY{n}{describe}\PY{p}{(}\PY{p}{)}\PY{p}{,}
                        \PY{n}{pd}\PY{o}{.}\PY{n}{Series}\PY{p}{(}\PY{n}{sepalwid}\PY{o}{.}\PY{n}{median}\PY{p}{(}\PY{p}{)}\PY{p}{,} \PY{n}{index}\PY{o}{=}\PY{p}{[}\PY{l+s+s1}{\PYZsq{}}\PY{l+s+s1}{median}\PY{l+s+s1}{\PYZsq{}}\PY{p}{]}\PY{p}{)}\PY{p}{,}
                        \PY{n}{pd}\PY{o}{.}\PY{n}{Series}\PY{p}{(}\PY{n}{sepalwid}\PY{o}{.}\PY{n}{mode}\PY{p}{(}\PY{p}{)}\PY{o}{.}\PY{n}{to\PYZus{}string}\PY{p}{(}\PY{n}{index}\PY{o}{=}\PY{k+kc}{False}\PY{p}{)}\PY{p}{,} \PY{n}{index} \PY{o}{=} \PY{p}{[}\PY{l+s+s1}{\PYZsq{}}\PY{l+s+s1}{mode}\PY{l+s+s1}{\PYZsq{}}\PY{p}{]}\PY{p}{)}\PY{p}{,}
                        \PY{n}{pd}\PY{o}{.}\PY{n}{Series}\PY{p}{(}\PY{n}{sepalwid}\PY{o}{.}\PY{n}{skew}\PY{p}{(}\PY{p}{)}\PY{p}{,} \PY{n}{index}\PY{o}{=}\PY{p}{[}\PY{l+s+s1}{\PYZsq{}}\PY{l+s+s1}{skewness}\PY{l+s+s1}{\PYZsq{}}\PY{p}{]}\PY{p}{)}\PY{p}{,}
                        \PY{n}{pd}\PY{o}{.}\PY{n}{Series}\PY{p}{(}\PY{n}{sepalwid}\PY{o}{.}\PY{n}{kurt}\PY{p}{(}\PY{p}{)}\PY{p}{,} \PY{n}{index}\PY{o}{=}\PY{p}{[}\PY{l+s+s1}{\PYZsq{}}\PY{l+s+s1}{kurtosis}\PY{l+s+s1}{\PYZsq{}}\PY{p}{]}\PY{p}{)}\PY{p}{]}\PY{p}{)}\PY{p}{)}\PY{o}{.}\PY{n}{T}

\PY{n}{describe}
\end{Verbatim}
\end{tcolorbox}

            \begin{tcolorbox}[breakable, size=fbox, boxrule=.5pt, pad at break*=1mm, opacityfill=0]
\prompt{Out}{outcolor}{7}{\boxspacing}
\begin{Verbatim}[commandchars=\\\{\}]
   count   mean       std  min  25\%  50\%    75\%  max median mode  skewness  \textbackslash{}
0  100.0  3.201  0.417906  2.2  3.0  3.2  3.425  4.4    3.2  3.0  0.283451

  kurtosis
0   0.1194
\end{Verbatim}
\end{tcolorbox}
        
    Отсюда видно, что мода, медиана и среднее имеют близкие значения, а
медиана и среднее практически равны. Значения выборки лежат в интервале
2.2 - 3.2. Значение коэффициента ассиметрии равно 0.283451, что в целом
небольшое значение, свидетельствующее об отклонении выборки влево. Также
из таблицы можно увидеть, что выборка обладает небольшим по значению
положительным коэффициентом эксцесса, что говорит о том, что выборка
имеет более острую вершину.

Построим график ``ящик с усами''.

    \begin{tcolorbox}[breakable, size=fbox, boxrule=1pt, pad at break*=1mm,colback=cellbackground, colframe=cellborder]
\prompt{In}{incolor}{8}{\boxspacing}
\begin{Verbatim}[commandchars=\\\{\}]
\PY{n}{sns}\PY{o}{.}\PY{n}{boxplot}\PY{p}{(}\PY{n}{pd}\PY{o}{.}\PY{n}{DataFrame}\PY{p}{(}\PY{n}{sepalwid}\PY{p}{)}\PY{p}{)}

\PY{n}{plt}\PY{o}{.}\PY{n}{show}\PY{p}{(}\PY{p}{)}
\end{Verbatim}
\end{tcolorbox}

    \begin{center}
    \adjustimage{max size={0.9\linewidth}{0.9\paperheight}}{output_14_0.png}
    \end{center}
    { \hspace*{\fill} \\}
    
    Отсюда видно, что он достаточно симметричен для того, чтобы не отклонять
гипотезу о нормальности распределения. Однако заметны некоторые
аномальные значения, находящиеся за пределами ``усов''. Это не является
весомой причиной для отклонения гипотезы о нормальности.

Построим график ``квантиль-квантиль''.

    \begin{tcolorbox}[breakable, size=fbox, boxrule=1pt, pad at break*=1mm,colback=cellbackground, colframe=cellborder]
\prompt{In}{incolor}{9}{\boxspacing}
\begin{Verbatim}[commandchars=\\\{\}]
\PY{n}{fig} \PY{o}{=} \PY{n}{sm}\PY{o}{.}\PY{n}{qqplot}\PY{p}{(}\PY{n}{sepalwid}\PY{p}{,} \PY{n}{line}\PY{o}{=}\PY{l+s+s1}{\PYZsq{}}\PY{l+s+s1}{r}\PY{l+s+s1}{\PYZsq{}}\PY{p}{)}

\PY{n}{plt}\PY{o}{.}\PY{n}{show}\PY{p}{(}\PY{p}{)}
\end{Verbatim}
\end{tcolorbox}

    \begin{center}
    \adjustimage{max size={0.9\linewidth}{0.9\paperheight}}{output_16_0.png}
    \end{center}
    { \hspace*{\fill} \\}
    
    На этом графике видно, что значения выборки близки к линии нормального
распределения, но на концах заметны некоторые отклонения, которые также
были замечены и на графике ``ящик с усами''.

    \subsubsection{Корреляционный
анализ}\label{ux43aux43eux440ux440ux435ux43bux44fux446ux438ux43eux43dux43dux44bux439-ux430ux43dux430ux43bux438ux437}

Исследуем связь значения случайной величины SEPALWID с остальными
случайными величинами. Для того чтобы увидеть общую картину построим
матрицу диаграмм рассеяния.

    \begin{tcolorbox}[breakable, size=fbox, boxrule=1pt, pad at break*=1mm,colback=cellbackground, colframe=cellborder]
\prompt{In}{incolor}{10}{\boxspacing}
\begin{Verbatim}[commandchars=\\\{\}]
\PY{n}{sns}\PY{o}{.}\PY{n}{pairplot}\PY{p}{(}\PY{n}{df}\PY{p}{[}\PY{p}{[}\PY{l+s+s1}{\PYZsq{}}\PY{l+s+s1}{sepal length (cm)}\PY{l+s+s1}{\PYZsq{}}\PY{p}{,} \PY{l+s+s1}{\PYZsq{}}\PY{l+s+s1}{sepal width (cm)}\PY{l+s+s1}{\PYZsq{}}\PY{p}{,} \PY{l+s+s1}{\PYZsq{}}\PY{l+s+s1}{petal length (cm)}\PY{l+s+s1}{\PYZsq{}}\PY{p}{,} \PY{l+s+s1}{\PYZsq{}}\PY{l+s+s1}{petal width (cm)}\PY{l+s+s1}{\PYZsq{}}\PY{p}{]}\PY{p}{]}\PY{o}{.}\PY{n}{astype}\PY{p}{(}\PY{n+nb}{float}\PY{p}{)}\PY{p}{,} \PY{n}{kind}\PY{o}{=}\PY{l+s+s1}{\PYZsq{}}\PY{l+s+s1}{reg}\PY{l+s+s1}{\PYZsq{}}\PY{p}{)}

\PY{n}{plt}\PY{o}{.}\PY{n}{show}\PY{p}{(}\PY{p}{)}
\end{Verbatim}
\end{tcolorbox}

    \begin{center}
    \adjustimage{max size={0.9\linewidth}{0.9\paperheight}}{output_19_0.png}
    \end{center}
    { \hspace*{\fill} \\}
    
    \paragraph{Корреляция между SEPALWID и
SEPALLEN}\label{ux43aux43eux440ux440ux435ux43bux44fux446ux438ux44f-ux43cux435ux436ux434ux443-sepalwid-ux438-sepallen}

Рассмотрим зависимость между переменными SEPALWID и SEPALLEN.

Для этого сначала построим диаграмму рассеяния для этих двух переменных.

    \begin{tcolorbox}[breakable, size=fbox, boxrule=1pt, pad at break*=1mm,colback=cellbackground, colframe=cellborder]
\prompt{In}{incolor}{11}{\boxspacing}
\begin{Verbatim}[commandchars=\\\{\}]
\PY{n}{sepallen} \PY{o}{=} \PY{n}{df}\PY{p}{[}\PY{l+s+s1}{\PYZsq{}}\PY{l+s+s1}{sepal length (cm)}\PY{l+s+s1}{\PYZsq{}}\PY{p}{]}\PY{o}{.}\PY{n}{astype}\PY{p}{(}\PY{n+nb}{float}\PY{p}{)}

\PY{n}{sns}\PY{o}{.}\PY{n}{regplot}\PY{p}{(}\PY{n}{x} \PY{o}{=} \PY{n}{sepallen}\PY{p}{,} \PY{n}{y} \PY{o}{=} \PY{n}{sepalwid}\PY{p}{)}

\PY{n}{plt}\PY{o}{.}\PY{n}{show}\PY{p}{(}\PY{p}{)}
\end{Verbatim}
\end{tcolorbox}

    \begin{center}
    \adjustimage{max size={0.9\linewidth}{0.9\paperheight}}{output_21_0.png}
    \end{center}
    { \hspace*{\fill} \\}
    
    Сильной зависимости на графике не наблюдается, однако необходимо всё же
выяснить, какова корреляция между ними, для этого можно использовать
следующее: + коэффициент корреляции Пирсона - в случае нормальности
распределения обеих переменных; + коэффициент корреляции Спирмена - в
случае нарушения нормальности распределения хотя бы одной из переменных.

В предварительном анализе переменной SEPALWID мы уже подтвердили
гипотезу о нормальности её распределения. Поэтому сразу перейдем к
проверке на нормальность переменной SEPALLEN. Для этого построим
гистограмму.

    \begin{tcolorbox}[breakable, size=fbox, boxrule=1pt, pad at break*=1mm,colback=cellbackground, colframe=cellborder]
\prompt{In}{incolor}{12}{\boxspacing}
\begin{Verbatim}[commandchars=\\\{\}]
\PY{n}{sns}\PY{o}{.}\PY{n}{histplot}\PY{p}{(}\PY{n}{sepallen}\PY{p}{,} \PY{n}{kde}\PY{o}{=}\PY{k+kc}{True}\PY{p}{)}

\PY{n}{plt}\PY{o}{.}\PY{n}{show}\PY{p}{(}\PY{p}{)}
\end{Verbatim}
\end{tcolorbox}

    \begin{center}
    \adjustimage{max size={0.9\linewidth}{0.9\paperheight}}{output_23_0.png}
    \end{center}
    { \hspace*{\fill} \\}
    
    На графике видна ассиметрия и видны два ``горба'', которые указывают на
то, что у распределния переменной SEPALLEN нарушается свойство
унимодулярности, что уже позволяет опровергнуть нашу гипотезу. Чтобы
сделать более точный вывод, построим график ``квантиль-квантиль''.

    \begin{tcolorbox}[breakable, size=fbox, boxrule=1pt, pad at break*=1mm,colback=cellbackground, colframe=cellborder]
\prompt{In}{incolor}{13}{\boxspacing}
\begin{Verbatim}[commandchars=\\\{\}]
\PY{n}{fig} \PY{o}{=} \PY{n}{sm}\PY{o}{.}\PY{n}{qqplot}\PY{p}{(}\PY{n}{sepallen}\PY{p}{,} \PY{n}{line}\PY{o}{=}\PY{l+s+s1}{\PYZsq{}}\PY{l+s+s1}{r}\PY{l+s+s1}{\PYZsq{}}\PY{p}{)}

\PY{n}{plt}\PY{o}{.}\PY{n}{show}\PY{p}{(}\PY{p}{)}
\end{Verbatim}
\end{tcolorbox}

    \begin{center}
    \adjustimage{max size={0.9\linewidth}{0.9\paperheight}}{output_25_0.png}
    \end{center}
    { \hspace*{\fill} \\}
    
    Видно, что вдоль всей линии нормального распределения так или иначе
присуствуют отклонения значений SEPALLEN от нее.

Для окончательного подтверждения гипотезу о том, что переменная не имеет
нормального распределения, рассмотрим критерии + Колмогорова-Смирнова, +
Шапиро-Уилка + х\^{}2-Пирсона

    \begin{tcolorbox}[breakable, size=fbox, boxrule=1pt, pad at break*=1mm,colback=cellbackground, colframe=cellborder]
\prompt{In}{incolor}{14}{\boxspacing}
\begin{Verbatim}[commandchars=\\\{\}]
\PY{n+nb}{print}\PY{p}{(}\PY{n}{kstest}\PY{p}{(}\PY{n}{sepallen}\PY{p}{,} \PY{l+s+s1}{\PYZsq{}}\PY{l+s+s1}{norm}\PY{l+s+s1}{\PYZsq{}}\PY{p}{)}\PY{p}{)}
\PY{n+nb}{print}\PY{p}{(}\PY{n}{scipy}\PY{o}{.}\PY{n}{stats}\PY{o}{.}\PY{n}{shapiro}\PY{p}{(}\PY{n}{sepallen}\PY{p}{)}\PY{p}{)}
\PY{n+nb}{print}\PY{p}{(}\PY{n}{scipy}\PY{o}{.}\PY{n}{stats}\PY{o}{.}\PY{n}{chisquare}\PY{p}{(}\PY{n}{sepallen}\PY{p}{)}\PY{p}{)}
\end{Verbatim}
\end{tcolorbox}

    \begin{Verbatim}[commandchars=\\\{\}]
KstestResult(statistic=0.999991460094529, pvalue=0.0, statistic\_location=4.3,
statistic\_sign=-1)
ShapiroResult(statistic=0.9455552101135254, pvalue=0.00042768396087922156)
Power\_divergenceResult(statistic=15.261186820769368, pvalue=1.0)
    \end{Verbatim}

    Здесь же у нас уже два критерия из трех имеют p-значение меньше 0.05,
что говорит о том, что переменная SEPALLEN не имеет нормального
распределения, что в свою очередь не дает нам использовать коэффициент
корреляции Пирсона, поэтому применим коэффициент корреляции Спирмена:

    \begin{tcolorbox}[breakable, size=fbox, boxrule=1pt, pad at break*=1mm,colback=cellbackground, colframe=cellborder]
\prompt{In}{incolor}{15}{\boxspacing}
\begin{Verbatim}[commandchars=\\\{\}]
\PY{n}{rho}\PY{p}{,} \PY{n}{p} \PY{o}{=} \PY{n}{spearmanr}\PY{p}{(}\PY{n}{sepallen}\PY{p}{,} \PY{n}{sepalwid}\PY{p}{)}

\PY{n+nb}{print}\PY{p}{(}\PY{l+s+s1}{\PYZsq{}}\PY{l+s+s1}{Spearman correlation coefficient:}\PY{l+s+s1}{\PYZsq{}}\PY{p}{,} \PY{n}{rho}\PY{p}{)}
\PY{n+nb}{print}\PY{p}{(}\PY{l+s+s1}{\PYZsq{}}\PY{l+s+s1}{Spearman P\PYZhy{}value:}\PY{l+s+s1}{\PYZsq{}}\PY{p}{,} \PY{n}{p}\PY{p}{)}
\end{Verbatim}
\end{tcolorbox}

    \begin{Verbatim}[commandchars=\\\{\}]
Spearman correlation coefficient: -0.23314058309509358
Spearman P-value: 0.019576865854219888
    \end{Verbatim}

    Дествительно, получили, что зависимость обратная и не сильно большая,
однако p-значения \textless{} 0.05, что говорит о том, что истинный
коэффициент корреляции не равен нулю.

    \paragraph{Корреляция между SEPALWID и
PETALLEN}\label{ux43aux43eux440ux440ux435ux43bux44fux446ux438ux44f-ux43cux435ux436ux434ux443-sepalwid-ux438-petallen}

Рассмотрим зависимость между переменными SEPALWID и PETALLEN.

Для этого сначала построим диаграмму рассеяния для этих двух переменных.

    \begin{tcolorbox}[breakable, size=fbox, boxrule=1pt, pad at break*=1mm,colback=cellbackground, colframe=cellborder]
\prompt{In}{incolor}{16}{\boxspacing}
\begin{Verbatim}[commandchars=\\\{\}]
\PY{n}{petallen} \PY{o}{=} \PY{n}{df}\PY{p}{[}\PY{l+s+s1}{\PYZsq{}}\PY{l+s+s1}{petal length (cm)}\PY{l+s+s1}{\PYZsq{}}\PY{p}{]}\PY{o}{.}\PY{n}{astype}\PY{p}{(}\PY{n+nb}{float}\PY{p}{)}

\PY{n}{sns}\PY{o}{.}\PY{n}{regplot}\PY{p}{(}\PY{n}{x}\PY{o}{=}\PY{n}{petallen}\PY{p}{,} \PY{n}{y}\PY{o}{=}\PY{n}{sepalwid}\PY{p}{)}

\PY{n}{plt}\PY{o}{.}\PY{n}{show}\PY{p}{(}\PY{p}{)}
\end{Verbatim}
\end{tcolorbox}

    \begin{center}
    \adjustimage{max size={0.9\linewidth}{0.9\paperheight}}{output_32_0.png}
    \end{center}
    { \hspace*{\fill} \\}
    
    Получили ситуацию, аналогичную прошлому подпункту, у нас нет
ярко-выраженной корреляции. Будем действовать по тому же алгоритму.

Выдвинем гипотезу о нормальности распределния PETALLEN, для ее
подтверждения или опровержения построим гистограмму.

    \begin{tcolorbox}[breakable, size=fbox, boxrule=1pt, pad at break*=1mm,colback=cellbackground, colframe=cellborder]
\prompt{In}{incolor}{17}{\boxspacing}
\begin{Verbatim}[commandchars=\\\{\}]
\PY{n}{sns}\PY{o}{.}\PY{n}{histplot}\PY{p}{(}\PY{n}{petallen}\PY{p}{,} \PY{n}{kde}\PY{o}{=}\PY{l+s+s1}{\PYZsq{}}\PY{l+s+s1}{r}\PY{l+s+s1}{\PYZsq{}}\PY{p}{)}

\PY{n}{plt}\PY{o}{.}\PY{n}{show}\PY{p}{(}\PY{p}{)}
\end{Verbatim}
\end{tcolorbox}

    \begin{center}
    \adjustimage{max size={0.9\linewidth}{0.9\paperheight}}{output_34_0.png}
    \end{center}
    { \hspace*{\fill} \\}
    
    Видно, что ничего общего с нормальным распределением данная гистограмма
не имеет, и уже сейчас можно отклонить гипотезу о нормальности
распределния PETALLEN. Для подтверждения этого факта построим график
``квантиль-квантиль''.

    \begin{tcolorbox}[breakable, size=fbox, boxrule=1pt, pad at break*=1mm,colback=cellbackground, colframe=cellborder]
\prompt{In}{incolor}{18}{\boxspacing}
\begin{Verbatim}[commandchars=\\\{\}]
\PY{n}{fig} \PY{o}{=} \PY{n}{sm}\PY{o}{.}\PY{n}{qqplot}\PY{p}{(}\PY{n}{petallen}\PY{p}{,} \PY{n}{line}\PY{o}{=}\PY{l+s+s1}{\PYZsq{}}\PY{l+s+s1}{r}\PY{l+s+s1}{\PYZsq{}}\PY{p}{)}

\PY{n}{plt}\PY{o}{.}\PY{n}{show}\PY{p}{(}\PY{p}{)}
\end{Verbatim}
\end{tcolorbox}

    \begin{center}
    \adjustimage{max size={0.9\linewidth}{0.9\paperheight}}{output_36_0.png}
    \end{center}
    { \hspace*{\fill} \\}
    
    Видно, что значения мало того, что не лежат на линии нормального
распределния, так еще имеют некоторую ``пропасть'', свидетельствующую о
разделении на кластеры.

Проверим критерии

    \begin{tcolorbox}[breakable, size=fbox, boxrule=1pt, pad at break*=1mm,colback=cellbackground, colframe=cellborder]
\prompt{In}{incolor}{19}{\boxspacing}
\begin{Verbatim}[commandchars=\\\{\}]
\PY{n+nb}{print}\PY{p}{(}\PY{n}{kstest}\PY{p}{(}\PY{n}{petallen}\PY{p}{,} \PY{l+s+s1}{\PYZsq{}}\PY{l+s+s1}{norm}\PY{l+s+s1}{\PYZsq{}}\PY{p}{)}\PY{p}{)}
\PY{n+nb}{print}\PY{p}{(}\PY{n}{scipy}\PY{o}{.}\PY{n}{stats}\PY{o}{.}\PY{n}{shapiro}\PY{p}{(}\PY{n}{petallen}\PY{p}{)}\PY{p}{)}
\PY{n+nb}{print}\PY{p}{(}\PY{n}{scipy}\PY{o}{.}\PY{n}{stats}\PY{o}{.}\PY{n}{chisquare}\PY{p}{(}\PY{n}{petallen}\PY{p}{)}\PY{p}{)}
\end{Verbatim}
\end{tcolorbox}

    \begin{Verbatim}[commandchars=\\\{\}]
KstestResult(statistic=0.8649303297782918, pvalue=3.066167865765482e-87,
statistic\_location=1.2, statistic\_sign=-1)
ShapiroResult(statistic=0.7805752754211426, pvalue=6.53977913489534e-11)
Power\_divergenceResult(statistic=123.92503564299972, pvalue=0.04572602442277413)
    \end{Verbatim}

    Все три критерия имеют p-значения \textless{} 0.05, что точно позволяет
нам отклонить гипотезу о нормальности распределния переменной PETALLEN.
Поэтому применим коэффициент корреляции Спирмена.

    \begin{tcolorbox}[breakable, size=fbox, boxrule=1pt, pad at break*=1mm,colback=cellbackground, colframe=cellborder]
\prompt{In}{incolor}{20}{\boxspacing}
\begin{Verbatim}[commandchars=\\\{\}]
\PY{n}{rho}\PY{p}{,} \PY{n}{p} \PY{o}{=} \PY{n}{spearmanr}\PY{p}{(}\PY{n}{petallen}\PY{p}{,} \PY{n}{sepalwid}\PY{p}{)}

\PY{n+nb}{print}\PY{p}{(}\PY{l+s+s1}{\PYZsq{}}\PY{l+s+s1}{Spearman correlation coefficient:}\PY{l+s+s1}{\PYZsq{}}\PY{p}{,} \PY{n}{rho}\PY{p}{)}
\PY{n+nb}{print}\PY{p}{(}\PY{l+s+s1}{\PYZsq{}}\PY{l+s+s1}{Spearman P\PYZhy{}value:}\PY{l+s+s1}{\PYZsq{}}\PY{p}{,} \PY{n}{p}\PY{p}{)}
\end{Verbatim}
\end{tcolorbox}

    \begin{Verbatim}[commandchars=\\\{\}]
Spearman correlation coefficient: -0.3761537388906749
Spearman P-value: 0.00011483558688011132
    \end{Verbatim}

    Значения коэффициента уже более весомо и свидетельствует об обратной
зависимости. P-значение t-критерия \textless{} 0.05, поэтому истинный
коэффициент корреляции не равен нулю.

    \paragraph{Корреляция между SEPALWID и
PETALLEN}\label{ux43aux43eux440ux440ux435ux43bux44fux446ux438ux44f-ux43cux435ux436ux434ux443-sepalwid-ux438-petallen}

Рассмотрим зависимость между переменными SEPALWID и PETALWID.

Для этого сначала построим диаграмму рассеяния для этих двух переменных.

    \begin{tcolorbox}[breakable, size=fbox, boxrule=1pt, pad at break*=1mm,colback=cellbackground, colframe=cellborder]
\prompt{In}{incolor}{21}{\boxspacing}
\begin{Verbatim}[commandchars=\\\{\}]
\PY{n}{petalwid} \PY{o}{=} \PY{n}{df}\PY{p}{[}\PY{l+s+s1}{\PYZsq{}}\PY{l+s+s1}{petal width (cm)}\PY{l+s+s1}{\PYZsq{}}\PY{p}{]}\PY{o}{.}\PY{n}{astype}\PY{p}{(}\PY{n+nb}{float}\PY{p}{)}

\PY{n}{sns}\PY{o}{.}\PY{n}{regplot}\PY{p}{(}\PY{n}{x}\PY{o}{=}\PY{n}{petalwid}\PY{p}{,} \PY{n}{y}\PY{o}{=}\PY{n}{sepalwid}\PY{p}{)}

\PY{n}{plt}\PY{o}{.}\PY{n}{show}\PY{p}{(}\PY{p}{)}
\end{Verbatim}
\end{tcolorbox}

    \begin{center}
    \adjustimage{max size={0.9\linewidth}{0.9\paperheight}}{output_43_0.png}
    \end{center}
    { \hspace*{\fill} \\}
    
    Получили ситуацию, очень похожую на прошлый подпункт. Будем действовать
по тому же алгоритму.

Выдвинем гипотезу о нормальности распределния PETALWID, для ее
подтверждения или опровержения построим гистограмму.

    \begin{tcolorbox}[breakable, size=fbox, boxrule=1pt, pad at break*=1mm,colback=cellbackground, colframe=cellborder]
\prompt{In}{incolor}{22}{\boxspacing}
\begin{Verbatim}[commandchars=\\\{\}]
\PY{n}{sns}\PY{o}{.}\PY{n}{histplot}\PY{p}{(}\PY{n}{petalwid}\PY{p}{,} \PY{n}{kde}\PY{o}{=}\PY{l+s+s1}{\PYZsq{}}\PY{l+s+s1}{r}\PY{l+s+s1}{\PYZsq{}}\PY{p}{)}

\PY{n}{plt}\PY{o}{.}\PY{n}{show}\PY{p}{(}\PY{p}{)}
\end{Verbatim}
\end{tcolorbox}

    \begin{center}
    \adjustimage{max size={0.9\linewidth}{0.9\paperheight}}{output_45_0.png}
    \end{center}
    { \hspace*{\fill} \\}
    
    Видно, что ничего общего с нормальным распределением данная гистограмма
не имеет, и уже сейчас можно отклонить гипотезу о нормальности
распределния PETALWID. Для подтверждения этого факта построим график
``квантиль-квантиль''.

    \begin{tcolorbox}[breakable, size=fbox, boxrule=1pt, pad at break*=1mm,colback=cellbackground, colframe=cellborder]
\prompt{In}{incolor}{23}{\boxspacing}
\begin{Verbatim}[commandchars=\\\{\}]
\PY{n}{fig} \PY{o}{=} \PY{n}{sm}\PY{o}{.}\PY{n}{qqplot}\PY{p}{(}\PY{n}{petalwid}\PY{p}{,} \PY{n}{line}\PY{o}{=}\PY{l+s+s1}{\PYZsq{}}\PY{l+s+s1}{r}\PY{l+s+s1}{\PYZsq{}}\PY{p}{)}

\PY{n}{plt}\PY{o}{.}\PY{n}{show}\PY{p}{(}\PY{p}{)}
\end{Verbatim}
\end{tcolorbox}

    \begin{center}
    \adjustimage{max size={0.9\linewidth}{0.9\paperheight}}{output_47_0.png}
    \end{center}
    { \hspace*{\fill} \\}
    
    Видно, что значения мало того, что не лежат на линии нормального
распределния, так еще имеют некоторую ``пропасть'', свидетельствующую о
разделении на кластеры.

Проверим критерии

    \begin{tcolorbox}[breakable, size=fbox, boxrule=1pt, pad at break*=1mm,colback=cellbackground, colframe=cellborder]
\prompt{In}{incolor}{24}{\boxspacing}
\begin{Verbatim}[commandchars=\\\{\}]
\PY{n+nb}{print}\PY{p}{(}\PY{n}{kstest}\PY{p}{(}\PY{n}{petalwid}\PY{p}{,} \PY{l+s+s1}{\PYZsq{}}\PY{l+s+s1}{norm}\PY{l+s+s1}{\PYZsq{}}\PY{p}{)}\PY{p}{)}
\PY{n+nb}{print}\PY{p}{(}\PY{n}{scipy}\PY{o}{.}\PY{n}{stats}\PY{o}{.}\PY{n}{shapiro}\PY{p}{(}\PY{n}{petalwid}\PY{p}{)}\PY{p}{)}
\PY{n+nb}{print}\PY{p}{(}\PY{n}{scipy}\PY{o}{.}\PY{n}{stats}\PY{o}{.}\PY{n}{chisquare}\PY{p}{(}\PY{n}{petalwid}\PY{p}{)}\PY{p}{)}
\end{Verbatim}
\end{tcolorbox}

    \begin{Verbatim}[commandchars=\\\{\}]
KstestResult(statistic=0.539827837277029, pvalue=8.390376214701721e-28,
statistic\_location=0.1, statistic\_sign=-1)
ShapiroResult(statistic=0.7883787155151367, pvalue=1.1055410753524342e-10)
Power\_divergenceResult(statistic=73.45985915492959, pvalue=0.9744939821023966)
    \end{Verbatim}

    Два критерия имеют p-значения \textless{} 0.05, что точно позволяет нам
отклонить гипотезу о нормальности распределния переменной PETALWID.
Поэтому применим коэффициент корреляции Спирмена.

    \begin{tcolorbox}[breakable, size=fbox, boxrule=1pt, pad at break*=1mm,colback=cellbackground, colframe=cellborder]
\prompt{In}{incolor}{25}{\boxspacing}
\begin{Verbatim}[commandchars=\\\{\}]
\PY{n}{rho}\PY{p}{,} \PY{n}{p} \PY{o}{=} \PY{n}{spearmanr}\PY{p}{(}\PY{n}{petalwid}\PY{p}{,} \PY{n}{sepalwid}\PY{p}{)}

\PY{n+nb}{print}\PY{p}{(}\PY{l+s+s1}{\PYZsq{}}\PY{l+s+s1}{Spearman correlation coefficient:}\PY{l+s+s1}{\PYZsq{}}\PY{p}{,} \PY{n}{rho}\PY{p}{)}
\PY{n+nb}{print}\PY{p}{(}\PY{l+s+s1}{\PYZsq{}}\PY{l+s+s1}{Spearman P\PYZhy{}value:}\PY{l+s+s1}{\PYZsq{}}\PY{p}{,} \PY{n}{p}\PY{p}{)}
\end{Verbatim}
\end{tcolorbox}

    \begin{Verbatim}[commandchars=\\\{\}]
Spearman correlation coefficient: -0.35378703960296415
Spearman P-value: 0.0003051288575492674
    \end{Verbatim}

    Значения коэффициента свидетельствует об обратной зависимости.
P-значение t-критерия \textless{} 0.05, поэтому истинный коэффициент
корреляции не равен нулю.

    \subsubsection{Регрессионный
анализ}\label{ux440ux435ux433ux440ux435ux441ux441ux438ux43eux43dux43dux44bux439-ux430ux43dux430ux43bux438ux437}

Проведем регрессионный анализ влияния факторов SEPALWID, SEPALLEN,
PETALLEN и PETALWID для построения регрессионной модели для предсказания
поведения зависимой переменной SEPALLEN.

Построим корреляционную таблицу с коэффициентами Спирмена.

    \begin{tcolorbox}[breakable, size=fbox, boxrule=1pt, pad at break*=1mm,colback=cellbackground, colframe=cellborder]
\prompt{In}{incolor}{26}{\boxspacing}
\begin{Verbatim}[commandchars=\\\{\}]
\PY{n}{corr} \PY{o}{=} \PY{n}{df}\PY{p}{[}\PY{p}{[}\PY{l+s+s1}{\PYZsq{}}\PY{l+s+s1}{sepal length (cm)}\PY{l+s+s1}{\PYZsq{}}\PY{p}{,} \PY{l+s+s1}{\PYZsq{}}\PY{l+s+s1}{sepal width (cm)}\PY{l+s+s1}{\PYZsq{}}\PY{p}{,} \PY{l+s+s1}{\PYZsq{}}\PY{l+s+s1}{petal length (cm)}\PY{l+s+s1}{\PYZsq{}}\PY{p}{,} \PY{l+s+s1}{\PYZsq{}}\PY{l+s+s1}{petal width (cm)}\PY{l+s+s1}{\PYZsq{}}\PY{p}{]}\PY{p}{]}\PY{o}{.}\PY{n}{astype}\PY{p}{(}\PY{n+nb}{float}\PY{p}{)}\PY{o}{.}\PY{n}{corr}\PY{p}{(}\PY{n}{method}\PY{o}{=}\PY{l+s+s1}{\PYZsq{}}\PY{l+s+s1}{spearman}\PY{l+s+s1}{\PYZsq{}}\PY{p}{)}
\PY{n}{corr}\PY{o}{.}\PY{n}{style}\PY{o}{.}\PY{n}{background\PYZus{}gradient}\PY{p}{(}\PY{n}{cmap}\PY{o}{=}\PY{l+s+s1}{\PYZsq{}}\PY{l+s+s1}{coolwarm}\PY{l+s+s1}{\PYZsq{}}\PY{p}{)}
\end{Verbatim}
\end{tcolorbox}

            \begin{tcolorbox}[breakable, size=fbox, boxrule=.5pt, pad at break*=1mm, opacityfill=0]
\prompt{Out}{outcolor}{26}{\boxspacing}
\begin{Verbatim}[commandchars=\\\{\}]
<pandas.io.formats.style.Styler at 0x201aff0e390>
\end{Verbatim}
\end{tcolorbox}
        
    Из этой таблицы видно, что сильной связью обладают независимые
переменные, что необходимо будет учесть при построении регрессионной
модели.

Начнем построение нашей модели с таблицы уровней значимости.

    \begin{tcolorbox}[breakable, size=fbox, boxrule=1pt, pad at break*=1mm,colback=cellbackground, colframe=cellborder]
\prompt{In}{incolor}{27}{\boxspacing}
\begin{Verbatim}[commandchars=\\\{\}]
\PY{k+kn}{import} \PY{n+nn}{statsmodels}\PY{n+nn}{.}\PY{n+nn}{api} \PY{k}{as} \PY{n+nn}{sm}

\PY{n}{x}\PY{p}{,} \PY{n}{y} \PY{o}{=} \PY{n}{df}\PY{p}{[}\PY{p}{[}\PY{l+s+s1}{\PYZsq{}}\PY{l+s+s1}{sepal length (cm)}\PY{l+s+s1}{\PYZsq{}}\PY{p}{,} \PY{l+s+s1}{\PYZsq{}}\PY{l+s+s1}{petal length (cm)}\PY{l+s+s1}{\PYZsq{}}\PY{p}{,} \PY{l+s+s1}{\PYZsq{}}\PY{l+s+s1}{petal width (cm)}\PY{l+s+s1}{\PYZsq{}}\PY{p}{]}\PY{p}{]}\PY{o}{.}\PY{n}{astype}\PY{p}{(}\PY{n+nb}{float}\PY{p}{)}\PY{p}{,} \PY{n}{df}\PY{p}{[}\PY{l+s+s1}{\PYZsq{}}\PY{l+s+s1}{sepal width (cm)}\PY{l+s+s1}{\PYZsq{}}\PY{p}{]}\PY{o}{.}\PY{n}{astype}\PY{p}{(}\PY{n+nb}{float}\PY{p}{)}
\PY{n}{x} \PY{o}{=} \PY{n}{sm}\PY{o}{.}\PY{n}{add\PYZus{}constant} \PY{p}{(}\PY{n}{x}\PY{p}{)}

\PY{n}{model} \PY{o}{=} \PY{n}{sm}\PY{o}{.} \PY{n}{OLS} \PY{p}{(}\PY{n}{y}\PY{p}{,} \PY{n}{x}\PY{p}{)}\PY{o}{.} \PY{n}{fit} \PY{p}{(}\PY{p}{)}
\PY{n+nb}{print}\PY{p}{(}\PY{n}{model}\PY{o}{.} \PY{n}{summary} \PY{p}{(}\PY{p}{)}\PY{p}{)}
\end{Verbatim}
\end{tcolorbox}

    \begin{Verbatim}[commandchars=\\\{\}]
                            OLS Regression Results
==============================================================================
Dep. Variable:       sepal width (cm)   R-squared:                       0.554
Model:                            OLS   Adj. R-squared:                  0.540
Method:                 Least Squares   F-statistic:                     39.77
Date:                Sun, 24 Dec 2023   Prob (F-statistic):           8.57e-17
Time:                        23:52:46   Log-Likelihood:                -13.753
No. Observations:                 100   AIC:                             35.51
Df Residuals:                      96   BIC:                             45.93
Df Model:                           3
Covariance Type:            nonrobust
================================================================================
=====
                        coef    std err          t      P>|t|      [0.025
0.975]
--------------------------------------------------------------------------------
-----
const                 1.0716      0.315      3.401      0.001       0.446
1.697
sepal length (cm)     0.5960      0.073      8.189      0.000       0.452
0.740
petal length (cm)    -0.4975      0.070     -7.082      0.000      -0.637
-0.358
petal width (cm)      0.3687      0.131      2.818      0.006       0.109
0.629
==============================================================================
Omnibus:                        1.260   Durbin-Watson:                   2.098
Prob(Omnibus):                  0.533   Jarque-Bera (JB):                0.750
Skew:                          -0.040   Prob(JB):                        0.687
Kurtosis:                       3.417   Cond. No.                         82.8
==============================================================================

Notes:
[1] Standard Errors assume that the covariance matrix of the errors is correctly
specified.
    \end{Verbatim}

    Здесь видно, что все факторы имеют влияение на переменную SEPALWID, но
следует помнить о сильной зависимости между независимыми переменными
SEPALLEN, PETALLEN и PETALWID. Выбросим из нашей модели ту переменную,
которая оказывает наименьшее влияние на переменную, то есть PETALWID.

    \begin{tcolorbox}[breakable, size=fbox, boxrule=1pt, pad at break*=1mm,colback=cellbackground, colframe=cellborder]
\prompt{In}{incolor}{28}{\boxspacing}
\begin{Verbatim}[commandchars=\\\{\}]
\PY{n}{x}\PY{p}{,} \PY{n}{y} \PY{o}{=} \PY{n}{df}\PY{p}{[}\PY{p}{[}\PY{l+s+s1}{\PYZsq{}}\PY{l+s+s1}{sepal length (cm)}\PY{l+s+s1}{\PYZsq{}}\PY{p}{,} \PY{l+s+s1}{\PYZsq{}}\PY{l+s+s1}{petal length (cm)}\PY{l+s+s1}{\PYZsq{}}\PY{p}{]}\PY{p}{]}\PY{o}{.}\PY{n}{astype}\PY{p}{(}\PY{n+nb}{float}\PY{p}{)}\PY{p}{,} \PY{n}{df}\PY{p}{[}\PY{l+s+s1}{\PYZsq{}}\PY{l+s+s1}{sepal width (cm)}\PY{l+s+s1}{\PYZsq{}}\PY{p}{]}\PY{o}{.}\PY{n}{astype}\PY{p}{(}\PY{n+nb}{float}\PY{p}{)}
\PY{n}{x} \PY{o}{=} \PY{n}{sm}\PY{o}{.}\PY{n}{add\PYZus{}constant} \PY{p}{(}\PY{n}{x}\PY{p}{)}

\PY{n}{model} \PY{o}{=} \PY{n}{sm}\PY{o}{.} \PY{n}{OLS} \PY{p}{(}\PY{n}{y}\PY{p}{,} \PY{n}{x}\PY{p}{)}\PY{o}{.} \PY{n}{fit} \PY{p}{(}\PY{p}{)}
\PY{n+nb}{print}\PY{p}{(}\PY{n}{model}\PY{o}{.} \PY{n}{summary} \PY{p}{(}\PY{p}{)}\PY{p}{)}
\end{Verbatim}
\end{tcolorbox}

    \begin{Verbatim}[commandchars=\\\{\}]
                            OLS Regression Results
==============================================================================
Dep. Variable:       sepal width (cm)   R-squared:                       0.517
Model:                            OLS   Adj. R-squared:                  0.507
Method:                 Least Squares   F-statistic:                     51.97
Date:                Sun, 24 Dec 2023   Prob (F-statistic):           4.56e-16
Time:                        23:52:46   Log-Likelihood:                -17.726
No. Observations:                 100   AIC:                             41.45
Df Residuals:                      97   BIC:                             49.27
Df Model:                           2
Covariance Type:            nonrobust
================================================================================
=====
                        coef    std err          t      P>|t|      [0.025
0.975]
--------------------------------------------------------------------------------
-----
const                 1.1505      0.325      3.542      0.001       0.506
1.795
sepal length (cm)     0.5480      0.073      7.482      0.000       0.403
0.693
petal length (cm)    -0.3212      0.033     -9.719      0.000      -0.387
-0.256
==============================================================================
Omnibus:                        0.663   Durbin-Watson:                   2.009
Prob(Omnibus):                  0.718   Jarque-Bera (JB):                0.254
Skew:                          -0.025   Prob(JB):                        0.881
Kurtosis:                       3.242   Cond. No.                         80.8
==============================================================================

Notes:
[1] Standard Errors assume that the covariance matrix of the errors is correctly
specified.
    \end{Verbatim}

    Помним про зависимость между SEPALLEN и PETALLEN и поступаем тем же
образом, удаляя переменную PETALLEN.

    \begin{tcolorbox}[breakable, size=fbox, boxrule=1pt, pad at break*=1mm,colback=cellbackground, colframe=cellborder]
\prompt{In}{incolor}{29}{\boxspacing}
\begin{Verbatim}[commandchars=\\\{\}]
\PY{n}{x}\PY{p}{,} \PY{n}{y} \PY{o}{=} \PY{n}{df}\PY{p}{[}\PY{p}{[}\PY{l+s+s1}{\PYZsq{}}\PY{l+s+s1}{sepal length (cm)}\PY{l+s+s1}{\PYZsq{}}\PY{p}{]}\PY{p}{]}\PY{o}{.}\PY{n}{astype}\PY{p}{(}\PY{n+nb}{float}\PY{p}{)}\PY{p}{,} \PY{n}{df}\PY{p}{[}\PY{l+s+s1}{\PYZsq{}}\PY{l+s+s1}{sepal width (cm)}\PY{l+s+s1}{\PYZsq{}}\PY{p}{]}\PY{o}{.}\PY{n}{astype}\PY{p}{(}\PY{n+nb}{float}\PY{p}{)}
\PY{n}{x} \PY{o}{=} \PY{n}{sm}\PY{o}{.}\PY{n}{add\PYZus{}constant} \PY{p}{(}\PY{n}{x}\PY{p}{)}

\PY{n}{model} \PY{o}{=} \PY{n}{sm}\PY{o}{.} \PY{n}{OLS} \PY{p}{(}\PY{n}{y}\PY{p}{,} \PY{n}{x}\PY{p}{)}\PY{o}{.} \PY{n}{fit} \PY{p}{(}\PY{p}{)}
\PY{n+nb}{print}\PY{p}{(}\PY{n}{model}\PY{o}{.} \PY{n}{summary} \PY{p}{(}\PY{p}{)}\PY{p}{)}
\end{Verbatim}
\end{tcolorbox}

    \begin{Verbatim}[commandchars=\\\{\}]
                            OLS Regression Results
==============================================================================
Dep. Variable:       sepal width (cm)   R-squared:                       0.047
Model:                            OLS   Adj. R-squared:                  0.038
Method:                 Least Squares   F-statistic:                     4.858
Date:                Sun, 24 Dec 2023   Prob (F-statistic):             0.0299
Time:                        23:52:46   Log-Likelihood:                -51.722
No. Observations:                 100   AIC:                             107.4
Df Residuals:                      98   BIC:                             112.7
Df Model:                           1
Covariance Type:            nonrobust
================================================================================
=====
                        coef    std err          t      P>|t|      [0.025
0.975]
--------------------------------------------------------------------------------
-----
const                 3.7579      0.256     14.680      0.000       3.250
4.266
sepal length (cm)    -0.0961      0.044     -2.204      0.030      -0.183
-0.010
==============================================================================
Omnibus:                        2.471   Durbin-Watson:                   1.609
Prob(Omnibus):                  0.291   Jarque-Bera (JB):                1.849
Skew:                           0.275   Prob(JB):                        0.397
Kurtosis:                       3.376   Cond. No.                         37.7
==============================================================================

Notes:
[1] Standard Errors assume that the covariance matrix of the errors is correctly
specified.
    \end{Verbatim}

    Получили модель вида sepalwid = 3.7579 - 0.0961*sepallen

Для оценки качества модели проведем анализ остатков на нормальность.
Построим гистограмму распределения для остатков.

    \begin{tcolorbox}[breakable, size=fbox, boxrule=1pt, pad at break*=1mm,colback=cellbackground, colframe=cellborder]
\prompt{In}{incolor}{30}{\boxspacing}
\begin{Verbatim}[commandchars=\\\{\}]
\PY{n}{raw\PYZus{}residuals} \PY{o}{=} \PY{n}{model}\PY{o}{.}\PY{n}{resid}

\PY{n}{sns}\PY{o}{.}\PY{n}{displot}\PY{p}{(}\PY{n}{pd}\PY{o}{.}\PY{n}{DataFrame}\PY{p}{(}\PY{n}{raw\PYZus{}residuals}\PY{p}{)}\PY{p}{,} \PY{n}{kde}\PY{o}{=}\PY{k+kc}{True}\PY{p}{)}
\PY{n}{plt}\PY{o}{.}\PY{n}{show}\PY{p}{(}\PY{p}{)}
\end{Verbatim}
\end{tcolorbox}

    \begin{center}
    \adjustimage{max size={0.9\linewidth}{0.9\paperheight}}{output_62_0.png}
    \end{center}
    { \hspace*{\fill} \\}
    
    По графику видно, что мы имеем практически идеальную гистограмму для
нормального распределения. Построим график ``квантиль-квантиль''.

    \begin{tcolorbox}[breakable, size=fbox, boxrule=1pt, pad at break*=1mm,colback=cellbackground, colframe=cellborder]
\prompt{In}{incolor}{31}{\boxspacing}
\begin{Verbatim}[commandchars=\\\{\}]
\PY{n}{fig} \PY{o}{=} \PY{n}{sm}\PY{o}{.}\PY{n}{qqplot}\PY{p}{(}\PY{n}{raw\PYZus{}residuals}\PY{p}{,} \PY{n}{line}\PY{o}{=}\PY{l+s+s1}{\PYZsq{}}\PY{l+s+s1}{r}\PY{l+s+s1}{\PYZsq{}}\PY{p}{)}
\PY{n}{plt}\PY{o}{.}\PY{n}{show}\PY{p}{(}\PY{p}{)}
\end{Verbatim}
\end{tcolorbox}

    \begin{center}
    \adjustimage{max size={0.9\linewidth}{0.9\paperheight}}{output_64_0.png}
    \end{center}
    { \hspace*{\fill} \\}
    
    Из графика видно, что большинство значений лежит на линии нормального
распределния с некоторыми отклонениями на концах. Что дает нам
подтвердить гипотезу о нормальности распределния остатков.

Построим на диаграмму рассеяния остатков и предсказанных моделью
значений.

    \begin{tcolorbox}[breakable, size=fbox, boxrule=1pt, pad at break*=1mm,colback=cellbackground, colframe=cellborder]
\prompt{In}{incolor}{32}{\boxspacing}
\begin{Verbatim}[commandchars=\\\{\}]
\PY{n}{sns}\PY{o}{.}\PY{n}{regplot}\PY{p}{(}\PY{n}{x} \PY{o}{=} \PY{n}{raw\PYZus{}residuals}\PY{p}{,} \PY{n}{y} \PY{o}{=} \PY{n}{model}\PY{o}{.}\PY{n}{predict}\PY{p}{(}\PY{p}{)}\PY{p}{)}

\PY{n}{plt}\PY{o}{.}\PY{n}{show}\PY{p}{(}\PY{p}{)}
\end{Verbatim}
\end{tcolorbox}

    \begin{center}
    \adjustimage{max size={0.9\linewidth}{0.9\paperheight}}{output_66_0.png}
    \end{center}
    { \hspace*{\fill} \\}
    
    Заметно отсутствие какой-либо зависимости между остатками и
предсказанными значениями.

Рассмотрим приемлимость модели в целом, для этого построим таблицу
ANOVA.

    \begin{tcolorbox}[breakable, size=fbox, boxrule=1pt, pad at break*=1mm,colback=cellbackground, colframe=cellborder]
\prompt{In}{incolor}{33}{\boxspacing}
\begin{Verbatim}[commandchars=\\\{\}]
\PY{n}{df\PYZus{}ren} \PY{o}{=} \PY{n}{df}\PY{o}{.}\PY{n}{rename}\PY{p}{(}\PY{n}{columns}\PY{o}{=}\PY{p}{\PYZob{}}\PY{l+s+s1}{\PYZsq{}}\PY{l+s+s1}{sepal width (cm)}\PY{l+s+s1}{\PYZsq{}}\PY{p}{:} \PY{l+s+s1}{\PYZsq{}}\PY{l+s+s1}{sepalwid}\PY{l+s+s1}{\PYZsq{}}\PY{p}{,} \PY{l+s+s1}{\PYZsq{}}\PY{l+s+s1}{sepal length (cm)}\PY{l+s+s1}{\PYZsq{}}\PY{p}{:} \PY{l+s+s1}{\PYZsq{}}\PY{l+s+s1}{sepallen}\PY{l+s+s1}{\PYZsq{}}\PY{p}{,} \PY{l+s+s1}{\PYZsq{}}\PY{l+s+s1}{petal length (cm)}\PY{l+s+s1}{\PYZsq{}}\PY{p}{:} \PY{l+s+s1}{\PYZsq{}}\PY{l+s+s1}{petallen}\PY{l+s+s1}{\PYZsq{}}\PY{p}{,} \PY{l+s+s1}{\PYZsq{}}\PY{l+s+s1}{petal width (cm)}\PY{l+s+s1}{\PYZsq{}}\PY{p}{:}\PY{l+s+s1}{\PYZsq{}}\PY{l+s+s1}{petalwid}\PY{l+s+s1}{\PYZsq{}}\PY{p}{\PYZcb{}}\PY{p}{)}

\PY{n}{model} \PY{o}{=} \PY{n}{sm}\PY{o}{.}\PY{n}{formula}\PY{o}{.}\PY{n}{ols}\PY{p}{(}\PY{l+s+s1}{\PYZsq{}}\PY{l+s+s1}{sepalwid \PYZti{} sepallen}\PY{l+s+s1}{\PYZsq{}}\PY{p}{,} \PY{n}{data}\PY{o}{=}\PY{n}{df\PYZus{}ren}\PY{p}{[}\PY{p}{[}\PY{l+s+s1}{\PYZsq{}}\PY{l+s+s1}{sepallen}\PY{l+s+s1}{\PYZsq{}}\PY{p}{,} \PY{l+s+s1}{\PYZsq{}}\PY{l+s+s1}{sepalwid}\PY{l+s+s1}{\PYZsq{}}\PY{p}{,} \PY{l+s+s1}{\PYZsq{}}\PY{l+s+s1}{petallen}\PY{l+s+s1}{\PYZsq{}}\PY{p}{,} \PY{l+s+s1}{\PYZsq{}}\PY{l+s+s1}{petalwid}\PY{l+s+s1}{\PYZsq{}}\PY{p}{]}\PY{p}{]}\PY{o}{.}\PY{n}{astype}\PY{p}{(}\PY{n+nb}{float}\PY{p}{)}\PY{p}{)}\PY{o}{.}\PY{n}{fit}\PY{p}{(}\PY{p}{)}
\PY{n}{anovatable} \PY{o}{=} \PY{n}{sm}\PY{o}{.}\PY{n}{stats}\PY{o}{.}\PY{n}{anova\PYZus{}lm}\PY{p}{(}\PY{n}{model}\PY{p}{,} \PY{n}{typ}\PY{o}{=}\PY{l+m+mi}{2}\PY{p}{)}

\PY{n+nb}{print}\PY{p}{(}\PY{n}{anovatable}\PY{p}{)}

\PY{n}{raw\PYZus{}residuals} \PY{o}{=} \PY{n}{model}\PY{o}{.}\PY{n}{resid}
\end{Verbatim}
\end{tcolorbox}

    \begin{Verbatim}[commandchars=\\\{\}]
             sum\_sq    df         F    PR(>F)
sepallen   0.816612   1.0  4.858043  0.029858
Residual  16.473288  98.0       NaN       NaN
    \end{Verbatim}

    Поскольку Р-уровень значимости меньше 0.05, то мы можем утверждать, что
наша модель приемлема и будет работать лучше, чем наивный прогноз по
средним значениям. Коэффициент детерминации R-squared = 0,047, значит
примерно 4,7\% факторов мы учли в своей модели. Что не является
достаточно хорошим показателем. Посмотрим, чему равен средний квадрат
ошибки полученной модели.

    \begin{tcolorbox}[breakable, size=fbox, boxrule=1pt, pad at break*=1mm,colback=cellbackground, colframe=cellborder]
\prompt{In}{incolor}{34}{\boxspacing}
\begin{Verbatim}[commandchars=\\\{\}]
\PY{n}{x}\PY{p}{,} \PY{n}{y} \PY{o}{=} \PY{n}{df}\PY{p}{[}\PY{p}{[}\PY{l+s+s1}{\PYZsq{}}\PY{l+s+s1}{sepal length (cm)}\PY{l+s+s1}{\PYZsq{}}\PY{p}{]}\PY{p}{]}\PY{o}{.}\PY{n}{astype}\PY{p}{(}\PY{n+nb}{float}\PY{p}{)}\PY{p}{,} \PY{n}{df}\PY{p}{[}\PY{l+s+s1}{\PYZsq{}}\PY{l+s+s1}{sepal width (cm)}\PY{l+s+s1}{\PYZsq{}}\PY{p}{]}\PY{o}{.}\PY{n}{astype}\PY{p}{(}\PY{n+nb}{float}\PY{p}{)}
\PY{n}{x} \PY{o}{=} \PY{n}{sm}\PY{o}{.}\PY{n}{add\PYZus{}constant} \PY{p}{(}\PY{n}{x}\PY{p}{)}

\PY{n}{model} \PY{o}{=} \PY{n}{sm}\PY{o}{.} \PY{n}{OLS} \PY{p}{(}\PY{n}{y}\PY{p}{,} \PY{n}{x}\PY{p}{)}\PY{o}{.} \PY{n}{fit} \PY{p}{(}\PY{p}{)}

\PY{n}{y\PYZus{}pred} \PY{o}{=} \PY{n}{model}\PY{o}{.}\PY{n}{predict}\PY{p}{(}\PY{n}{x}\PY{p}{)}

\PY{n+nb}{print}\PY{p}{(}\PY{n}{sklearn}\PY{o}{.}\PY{n}{metrics}\PY{o}{.}\PY{n}{mean\PYZus{}squared\PYZus{}error}\PY{p}{(}\PY{n}{y}\PY{p}{,} \PY{n}{y\PYZus{}pred}\PY{p}{)}\PY{p}{)}
\end{Verbatim}
\end{tcolorbox}

    \begin{Verbatim}[commandchars=\\\{\}]
0.16473288323267674
    \end{Verbatim}

    Интересно сравнить это значение с тем, если бы мы не удаляли переменную
PETALLEN из модели. Вернемся к моменту построения нашей модели и оставим
PETALLEN.

    \begin{tcolorbox}[breakable, size=fbox, boxrule=1pt, pad at break*=1mm,colback=cellbackground, colframe=cellborder]
\prompt{In}{incolor}{35}{\boxspacing}
\begin{Verbatim}[commandchars=\\\{\}]
\PY{n}{x}\PY{p}{,} \PY{n}{y} \PY{o}{=} \PY{n}{df}\PY{p}{[}\PY{p}{[}\PY{l+s+s1}{\PYZsq{}}\PY{l+s+s1}{sepal length (cm)}\PY{l+s+s1}{\PYZsq{}}\PY{p}{,} \PY{l+s+s1}{\PYZsq{}}\PY{l+s+s1}{petal length (cm)}\PY{l+s+s1}{\PYZsq{}}\PY{p}{]}\PY{p}{]}\PY{o}{.}\PY{n}{astype}\PY{p}{(}\PY{n+nb}{float}\PY{p}{)}\PY{p}{,} \PY{n}{df}\PY{p}{[}\PY{l+s+s1}{\PYZsq{}}\PY{l+s+s1}{sepal width (cm)}\PY{l+s+s1}{\PYZsq{}}\PY{p}{]}\PY{o}{.}\PY{n}{astype}\PY{p}{(}\PY{n+nb}{float}\PY{p}{)}
\PY{n}{x} \PY{o}{=} \PY{n}{sm}\PY{o}{.}\PY{n}{add\PYZus{}constant} \PY{p}{(}\PY{n}{x}\PY{p}{)}

\PY{n}{model} \PY{o}{=} \PY{n}{sm}\PY{o}{.} \PY{n}{OLS} \PY{p}{(}\PY{n}{y}\PY{p}{,} \PY{n}{x}\PY{p}{)}\PY{o}{.} \PY{n}{fit} \PY{p}{(}\PY{p}{)}
\PY{n+nb}{print}\PY{p}{(}\PY{n}{model}\PY{o}{.} \PY{n}{summary} \PY{p}{(}\PY{p}{)}\PY{p}{)}
\end{Verbatim}
\end{tcolorbox}

    \begin{Verbatim}[commandchars=\\\{\}]
                            OLS Regression Results
==============================================================================
Dep. Variable:       sepal width (cm)   R-squared:                       0.517
Model:                            OLS   Adj. R-squared:                  0.507
Method:                 Least Squares   F-statistic:                     51.97
Date:                Sun, 24 Dec 2023   Prob (F-statistic):           4.56e-16
Time:                        23:52:46   Log-Likelihood:                -17.726
No. Observations:                 100   AIC:                             41.45
Df Residuals:                      97   BIC:                             49.27
Df Model:                           2
Covariance Type:            nonrobust
================================================================================
=====
                        coef    std err          t      P>|t|      [0.025
0.975]
--------------------------------------------------------------------------------
-----
const                 1.1505      0.325      3.542      0.001       0.506
1.795
sepal length (cm)     0.5480      0.073      7.482      0.000       0.403
0.693
petal length (cm)    -0.3212      0.033     -9.719      0.000      -0.387
-0.256
==============================================================================
Omnibus:                        0.663   Durbin-Watson:                   2.009
Prob(Omnibus):                  0.718   Jarque-Bera (JB):                0.254
Skew:                          -0.025   Prob(JB):                        0.881
Kurtosis:                       3.242   Cond. No.                         80.8
==============================================================================

Notes:
[1] Standard Errors assume that the covariance matrix of the errors is correctly
specified.
    \end{Verbatim}

    Получили модель вида sepalwid = 1.1505 + 0.5480\emph{sepallen -
0.3212}petallen

Для оценки качества модели проведем анализ остатков на нормальность.
Построим гистограмму распределения для остатков.

    \begin{tcolorbox}[breakable, size=fbox, boxrule=1pt, pad at break*=1mm,colback=cellbackground, colframe=cellborder]
\prompt{In}{incolor}{36}{\boxspacing}
\begin{Verbatim}[commandchars=\\\{\}]
\PY{n}{raw\PYZus{}residuals} \PY{o}{=} \PY{n}{model}\PY{o}{.}\PY{n}{resid}

\PY{n}{sns}\PY{o}{.}\PY{n}{displot}\PY{p}{(}\PY{n}{pd}\PY{o}{.}\PY{n}{DataFrame}\PY{p}{(}\PY{n}{raw\PYZus{}residuals}\PY{p}{)}\PY{p}{,} \PY{n}{kde}\PY{o}{=}\PY{k+kc}{True}\PY{p}{)}
\PY{n}{plt}\PY{o}{.}\PY{n}{show}\PY{p}{(}\PY{p}{)}
\end{Verbatim}
\end{tcolorbox}

    \begin{center}
    \adjustimage{max size={0.9\linewidth}{0.9\paperheight}}{output_74_0.png}
    \end{center}
    { \hspace*{\fill} \\}
    
    В целом, гистограмма выглядит неплохо, но имеет небольшое смещение
вправо. Построим график ``квантиль-квантиль''.

    \begin{tcolorbox}[breakable, size=fbox, boxrule=1pt, pad at break*=1mm,colback=cellbackground, colframe=cellborder]
\prompt{In}{incolor}{37}{\boxspacing}
\begin{Verbatim}[commandchars=\\\{\}]
\PY{n}{fig} \PY{o}{=} \PY{n}{sm}\PY{o}{.}\PY{n}{qqplot}\PY{p}{(}\PY{n}{raw\PYZus{}residuals}\PY{p}{,} \PY{n}{line}\PY{o}{=}\PY{l+s+s1}{\PYZsq{}}\PY{l+s+s1}{r}\PY{l+s+s1}{\PYZsq{}}\PY{p}{)}
\PY{n}{plt}\PY{o}{.}\PY{n}{show}\PY{p}{(}\PY{p}{)}
\end{Verbatim}
\end{tcolorbox}

    \begin{center}
    \adjustimage{max size={0.9\linewidth}{0.9\paperheight}}{output_76_0.png}
    \end{center}
    { \hspace*{\fill} \\}
    
    Я бы сказал, что ситуация на этом графике даже лучше, чем на прошлой
модели, но она не лишена недостатков в виде отклонения на концах.

Подтверждаем гипотезу о нормальности распределения остатков и посмотрим
на диаграмму рассеяния остатков и предсказанных моделью значений.

    \begin{tcolorbox}[breakable, size=fbox, boxrule=1pt, pad at break*=1mm,colback=cellbackground, colframe=cellborder]
\prompt{In}{incolor}{38}{\boxspacing}
\begin{Verbatim}[commandchars=\\\{\}]
\PY{n}{sns}\PY{o}{.}\PY{n}{regplot}\PY{p}{(}\PY{n}{x} \PY{o}{=} \PY{n}{raw\PYZus{}residuals}\PY{p}{,} \PY{n}{y} \PY{o}{=} \PY{n}{model}\PY{o}{.}\PY{n}{predict}\PY{p}{(}\PY{p}{)}\PY{p}{)}

\PY{n}{plt}\PY{o}{.}\PY{n}{show}\PY{p}{(}\PY{p}{)}
\end{Verbatim}
\end{tcolorbox}

    \begin{center}
    \adjustimage{max size={0.9\linewidth}{0.9\paperheight}}{output_78_0.png}
    \end{center}
    { \hspace*{\fill} \\}
    
    Заметно отсутствие какой-либо зависимости между остатками и
предсказанными значениями.

Рассмотрим приемлимость модели в целом, для этого построим таблицу
ANOVA.

    \begin{tcolorbox}[breakable, size=fbox, boxrule=1pt, pad at break*=1mm,colback=cellbackground, colframe=cellborder]
\prompt{In}{incolor}{39}{\boxspacing}
\begin{Verbatim}[commandchars=\\\{\}]
\PY{n}{df\PYZus{}ren} \PY{o}{=} \PY{n}{df}\PY{o}{.}\PY{n}{rename}\PY{p}{(}\PY{n}{columns}\PY{o}{=}\PY{p}{\PYZob{}}\PY{l+s+s1}{\PYZsq{}}\PY{l+s+s1}{sepal width (cm)}\PY{l+s+s1}{\PYZsq{}}\PY{p}{:} \PY{l+s+s1}{\PYZsq{}}\PY{l+s+s1}{sepalwid}\PY{l+s+s1}{\PYZsq{}}\PY{p}{,} \PY{l+s+s1}{\PYZsq{}}\PY{l+s+s1}{sepal length (cm)}\PY{l+s+s1}{\PYZsq{}}\PY{p}{:} \PY{l+s+s1}{\PYZsq{}}\PY{l+s+s1}{sepallen}\PY{l+s+s1}{\PYZsq{}}\PY{p}{,} \PY{l+s+s1}{\PYZsq{}}\PY{l+s+s1}{petal length (cm)}\PY{l+s+s1}{\PYZsq{}}\PY{p}{:} \PY{l+s+s1}{\PYZsq{}}\PY{l+s+s1}{petallen}\PY{l+s+s1}{\PYZsq{}}\PY{p}{,} \PY{l+s+s1}{\PYZsq{}}\PY{l+s+s1}{petal width (cm)}\PY{l+s+s1}{\PYZsq{}}\PY{p}{:}\PY{l+s+s1}{\PYZsq{}}\PY{l+s+s1}{petalwid}\PY{l+s+s1}{\PYZsq{}}\PY{p}{\PYZcb{}}\PY{p}{)}

\PY{n}{model} \PY{o}{=} \PY{n}{sm}\PY{o}{.}\PY{n}{formula}\PY{o}{.}\PY{n}{ols}\PY{p}{(}\PY{l+s+s1}{\PYZsq{}}\PY{l+s+s1}{sepalwid \PYZti{} sepallen}\PY{l+s+s1}{\PYZsq{}}\PY{p}{,} \PY{n}{data}\PY{o}{=}\PY{n}{df\PYZus{}ren}\PY{p}{[}\PY{p}{[}\PY{l+s+s1}{\PYZsq{}}\PY{l+s+s1}{sepallen}\PY{l+s+s1}{\PYZsq{}}\PY{p}{,} \PY{l+s+s1}{\PYZsq{}}\PY{l+s+s1}{sepalwid}\PY{l+s+s1}{\PYZsq{}}\PY{p}{,} \PY{l+s+s1}{\PYZsq{}}\PY{l+s+s1}{petallen}\PY{l+s+s1}{\PYZsq{}}\PY{p}{,} \PY{l+s+s1}{\PYZsq{}}\PY{l+s+s1}{petalwid}\PY{l+s+s1}{\PYZsq{}}\PY{p}{]}\PY{p}{]}\PY{o}{.}\PY{n}{astype}\PY{p}{(}\PY{n+nb}{float}\PY{p}{)}\PY{p}{)}\PY{o}{.}\PY{n}{fit}\PY{p}{(}\PY{p}{)}
\PY{n}{anovatable} \PY{o}{=} \PY{n}{sm}\PY{o}{.}\PY{n}{stats}\PY{o}{.}\PY{n}{anova\PYZus{}lm}\PY{p}{(}\PY{n}{model}\PY{p}{,} \PY{n}{typ}\PY{o}{=}\PY{l+m+mi}{2}\PY{p}{)}

\PY{n+nb}{print}\PY{p}{(}\PY{n}{anovatable}\PY{p}{)}

\PY{n}{raw\PYZus{}residuals} \PY{o}{=} \PY{n}{model}\PY{o}{.}\PY{n}{resid}
\end{Verbatim}
\end{tcolorbox}

    \begin{Verbatim}[commandchars=\\\{\}]
             sum\_sq    df         F    PR(>F)
sepallen   0.816612   1.0  4.858043  0.029858
Residual  16.473288  98.0       NaN       NaN
    \end{Verbatim}

    Поскольку Р-уровень значимости меньше 0.05, то мы можем утверждать, что
наша модель приемлема и будет работать лучше, чем наивный прогноз по
средним значениям. Коэффициент детерминации R-squared = 0,517, значит
примерно 52\% факторов мы учли в своей модели, что значительно лучше
значения для прошлой модели. Посмотрим, чему равен средний квадрат
ошибки полученной модели.

    \begin{tcolorbox}[breakable, size=fbox, boxrule=1pt, pad at break*=1mm,colback=cellbackground, colframe=cellborder]
\prompt{In}{incolor}{40}{\boxspacing}
\begin{Verbatim}[commandchars=\\\{\}]
\PY{n}{x} \PY{o}{=} \PY{n}{df}\PY{p}{[}\PY{p}{[}\PY{l+s+s1}{\PYZsq{}}\PY{l+s+s1}{sepal length (cm)}\PY{l+s+s1}{\PYZsq{}}\PY{p}{,} \PY{l+s+s1}{\PYZsq{}}\PY{l+s+s1}{petal width (cm)}\PY{l+s+s1}{\PYZsq{}}\PY{p}{]}\PY{p}{]}\PY{o}{.}\PY{n}{astype}\PY{p}{(}\PY{n+nb}{float}\PY{p}{)}
\PY{n}{y} \PY{o}{=} \PY{n}{df}\PY{p}{[}\PY{l+s+s1}{\PYZsq{}}\PY{l+s+s1}{sepal width (cm)}\PY{l+s+s1}{\PYZsq{}}\PY{p}{]}\PY{o}{.}\PY{n}{astype}\PY{p}{(}\PY{n+nb}{float}\PY{p}{)}

\PY{n}{x} \PY{o}{=} \PY{n}{sm}\PY{o}{.}\PY{n}{add\PYZus{}constant}\PY{p}{(}\PY{n}{x}\PY{p}{)}

\PY{n}{model} \PY{o}{=} \PY{n}{sm}\PY{o}{.}\PY{n}{OLS}\PY{p}{(}\PY{n}{y}\PY{p}{,} \PY{n}{x}\PY{p}{)}\PY{o}{.}\PY{n}{fit}\PY{p}{(}\PY{p}{)}

\PY{n}{y\PYZus{}pred} \PY{o}{=} \PY{n}{model}\PY{o}{.}\PY{n}{predict}\PY{p}{(}\PY{n}{x}\PY{p}{)}

\PY{n+nb}{print}\PY{p}{(}\PY{n}{sklearn}\PY{o}{.}\PY{n}{metrics}\PY{o}{.}\PY{n}{mean\PYZus{}squared\PYZus{}error}\PY{p}{(}\PY{n}{y}\PY{p}{,} \PY{n}{y\PYZus{}pred}\PY{p}{)}\PY{p}{)}
\end{Verbatim}
\end{tcolorbox}

    \begin{Verbatim}[commandchars=\\\{\}]
0.11735902671139867
    \end{Verbatim}

    Средний квадрат ошибки здесь меньше, что, учитывая еще значение
R-squared, говорит о том, что эта модель лучше работает для предсказания
значения переменной SEPALWID.


    % Add a bibliography block to the postdoc
    
    
    
\end{document}
